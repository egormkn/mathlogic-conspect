\section*{Базовые понятия}
\addcontentsline{toc}{section}{Базовые понятия}
\label{sec-1}
\subsection*{Формальные системы и модели}
\label{sec-1-1}
Сделано мной для меня самого, be careful

Мы работаем с формальными системами.
Формальная система определяется сигнатурой, грамматикой,
набором аксиом и набором правил вывода.
\begin{enumerate}
\item Сигнатура ФС -- это (Pr, F, C, Links, Misc, arity):
\begin{itemize}
\item $\mathrm Pr$ -- описывает предикаты (число + заглавная буква латинского алфавита)
\item $\mathrm F$ -- множество функций (заглавные буквы латинского алфавита)
\item $\mathrm C$ -- описывает константы
\item Links -- множество связок ($\lbrace$«$\to$», «$\cup$», «пробел»$\rbrace$)
\item Misc -- дополнительные элементы ($\lbrace$<<$($>>, <<$)$>>, <<пробел>>$\rbrace$)
\item $arity\colon Foo \cup Pr \cup C \to \mathbb N$ возвращает арность
\end{itemize}
\item Грамматика описывает то, как мы можем строить выражения
в соответствии с нашей сигнатурой.
\item Аксиомы -- выражения в нашей грамматике.
\item Правила вывода -- пары вида (List, List), где List --
список утверждений. Первый элемент – посылки, второй --
то, что из них следует.
\end{enumerate}

Иногда нам хочется что-то посчитать и мы прикручиваем к
формальной системе модель -- корректную структуру с оценкой.
Структура -- это сигнатура с интерпретацией и носителем.
\begin{enumerate}
    \item Сигнатура структуры -- (R, F, C, arity):
    \begin{itemize}
        \item $Pr$ -- множество символов для предикатов
        \item $F$ -- функциональных символов
        \item $C$ -- символов констант
        \item $arity$ – функция, определяющая арность $Pr \cup F \to \mathbb N$.
    \end{itemize}
    \item Интерпретация -- это приписывание символам значения
    и правил действия (отображения из $Pr \cup F \cup C$ в носитель)
    \item Носитель -- это объединение множеств, в котором обязательно
    присутствует $V$ -- множество истинностных значений. Если же
    мы рассматриваем только нульместные предикаты, на этом
    можно остановиться, otherwise часто вводится P -- предметное
    множество, в которое отображаются элементы из $F, C$.
\end{enumerate}
TODO Эта реализация структуры не определяет ничего в районе
аксиоматики, но аксиоматически заданные структуры существуют
– например в ФА есть Пеано.

Если все аксиомы тавтологии, то структура корректна.
В таком случае она называется моделью.

Оценку иногда определяют раньше/позже чем модель, мне
удобно думать о ней, как об отдельной сущности, потому что
она связывает модель с ФС.

Оценка -- это функция оценки и функция тавтологии.
\begin{enumerate}
    \item Функция оценки -- отображение из (множества всех формул,
        сгенеренных грамматикой) $\times$ (какие-нибудь допаргументы)
        в $V$ модели. Дополнительные аргументы -- например оценки
        элементов связки.
    \item Функция тавтологии -- отображение из множества формул
        грамматики в $\{0, 1\}$ -- является ли формула тавтологией.
        Тавтология использует функцию оценки. Например, тавтология
        -- это выражение, оценка которого на любых аргументах
        возвращает $\sigma \in V$ -- какой-то элемент $V$.
\end{enumerate}

Когда говорится <<сигнатура модели>> -- имеется в виду ровно она.
Когда говорится <<сигнатура ФС>> -- имеется в виду скорее всего
объединение сигнатур, а может только сигнатура самой ФС. Первый
вариант тут предпочтительней.
