\section{Ticket 15: кардиналы}
\label{sec-17}
\subsection{Кардинальные числа}
\label{sec-17-1}
Будем называть множества равномощными, если найдется биекция.
Будем называть A не превышающим по мощности B, если найдется
инъекция A \to B (|A| \le |B|)
Будем называть А меньше по мощности, чем B, если |A| \le |B| \& |A| \ne  |B|
Кардинальное число - число, оценивающее мощность множества.
Кардинальное число ℵ - это ординальное число a, такое что
\forall  x \le a |x| \le |a|
ℵ₀ = w по определению; ℵ_1 = минимальный кардинал, следующий за ℵ₀
Кардинальное число ℶ - это ординальное число а, такое что
ℶ_i = P(ℶ_i₋_1)
ℶ₀ = ℵ₀
Континуум-гипотеза формулируется таким образом: |P(ℵ₀)| = ℵ_1 или ℶ_1 = ℵ_1
В 40 году Гёдель доказал недоказуемость отрицания Континуум-гипотезы
в терминах ZFC, в 60 Коэн сделал то же самое но без отрицания. Это все
в условиях непротиворечивости ФА. То есть в ZFC нельзя доказать или
опровергнуть континуум-гипотезу.
Сложение кардинальных чисел - |A| + |B| = |A∪B| если в них
нету общих элементов, иначе max(|A|, |B|), поскольку мы можем
построить двумерную таблицу из перес. элементов.
Остальное есть на вики и вряд ли нужно вообще.
\subsection{Диагональный метод Кантора}
\label{sec-17-2}
Докажем, что для любого множества |x| < |P(x)|
Воспользуемся диагональным методом Kантора
Пусть |x|=|P(x)|
Выпишем таблицу, в которой  столбцу p и строке q соответствует
1, если в множестве X лежит p, а в множестве P(X) лежит
множество, содержащее в себе p. Построим ключевое мн-во t:
элемент лежит в t, если на i-й диагональной позиции не стоит 1
и наоборот. То есть это множество всех таких элементов из X,
которым по биекции соответствует множество о чем угодно, но не
о самом элементе (не включающее элемент).
t состоит из подмножеств X, тогда оно должно лежать в P(X).
Докажем, что строка t не присутствует в таблице, сравнив ее с
каждой другой строкой - от каждой n-й строки отличается в n-й
столбце по построению.
Противоречие - t нет в таблице, но t \in P(X).
\subsection{Теорема Лёвингейма-Скулема}
\label{sec-17-3}
\begin{itemize}
\item Назовем мощностью модели мощность ее носителя (\lor или P или \lor ∪P).
M - модель, |M| - ее мощность, ну ясно.
\item Элементарная подмодель
Пусть M - модель фс первого порядка с носителем D. Пусть определено
D_1 ⊂ D, тогда структура M_1 построенная на D_1 так, что в ее интерпретации
лежит все, что и в интерпретации M, кроме элементов, взаимодействующих
с M $\backslash$ M_1 (сужение области определения на D_1), называется \textbf{элементарной}
\textbf{подмоделью}, если:
\begin{enumerate}
\item Любая функция ФС, над которой рассматривается M, замкнута на
D_1 (то есть если a \in D_1, b \in D_1, \dots  то f(a, b, \dots ) \in D_1)
\item Любая формула A(x_1\dots x_n) теории при любых аргументах из D_1,
истинная в M истинна и в M_1.
\end{enumerate}
\item Элементарная подмодель теории - модель теории
Рассмотрим формулу А, она общазначима в М, значит и в М_1, тогда M_1 корректна.
\item Счетно-аксиоматизируемая теория - множество аксиом ФС имеет мощность ℵ₀
\item ФА и ТМ счетно-аксоиматизируемые
\item Пусть M - модель, T - мн-во формул теории. Тогда \exists M_1 : |M_1| = max(|T|, ℵ₀)
Нужно построить необходимое предметное множество и доказать,
что модель на нем - это подмодель.
\begin{enumerate}
\item Построение множества
Пусть у нас есть множество D', тогда D'' = D' ∪ P, где P - некоторое
множество формул добавленное при рассмотрении формул D' по одной.
A(y, x_1\dots x_n) - n-местная формула из Т. Фиксируем x_1\dots x_n из D'.
\begin{itemize}
\item Если А = И или А = Л (тождественно) при любом y \in D - пропустим формулу
\item Если А = И или А = Л при каких-то y \in D' - пропустим формулу
\item \exists y: A(y,..) = И, но при этом \forall y\in D' A(y,..) = Л - тогда добавим один из
тех у, на которых формула истинна, в D''. Добавим еще констант, которые
нужны для вычисления А. Типа если В D' не хватает переменных для того,
чтобы показать что A может принимать истинностное значение, сгенерим
и добавим такое.
\end{itemize}
Переход от предыдущего множества к текущему увеличивает его не более чем на
ℵ₀ * |T| * |D'| - max(ℵ₀, T)
Рассмотрим D₀, D₀ ⊂ D такое, что в него входят те элементы носителя,
соответствующие константам, упоминающимся в Т. Если оно пустое -- добавим
какую-нибудь константу из D. Оно ляжет в начало счетной последовательности
D₀ ⊂ D_1 ⊂ \dots  (каждый переход описан выше). D* = ∪D_i.
D* - нужное нам множество. |D*| = max(ℵ₀, T)
\item Проверка структуры
Индукция по структуре.
\begin{itemize}
\item База. Предикат.
P(f_1(x_1\dots x_n), \dots , fₖ(x_1\dots x_n)). Если x_1\dots x_n взяты из D*, то они были
добавлены на некотором шаге, значит \exists t | x_i \in Dₜ. Тогда на шаге Dₜ₊_1
лежат результаты функций f_1\dots fₖ. по построению. Тогда оценка формулы
сохраняется.
\item Переход
Связки X\&Y, X\lor Y, X\to Y, \lnot X работают на сужении модели и оценка сохр.
\begin{itemize}
\item \exists yB(y, x_1\dots x_n). Фиксируем x_1\dots x_n из D*.
\begin{enumerate}
\item A была тождественно истинна или ложна - все ОК
\item А принимала значения разных знаков
Каждый x_i добавлен на каком-то шаге, тогда возьмем максимальный
шаг t, в Dₜ₊_1 уже лежат все эти x_i.
Тогда по построению Dₜ₊_1 мы добавили нужный y такой что B(y, x_1\dots x_n)
определено и выполнено в M.
Значит B выполнено в M* по индукции, тогда A истинна в M*.
\end{enumerate}
\end{itemize}
\item \forall yB(y, x_1\dots x_n).
\begin{enumerate}
\item Тождественно - ОК
\item Принимает значения разных знаков
Если оно истинно в M, тогда оно истинно в M* по 1 пункту перехода.
Если \forall yB(y, x_1\dots x_n) было ложно на t-шаге, тогда на t+1 шаге мы
здоровски исправили ситуацию, положив в мир y на котором оно истинно.
Если оно было истинно, то по пункту 2 пошло дальше.
\end{enumerate}
Таким образом, D* - подмодель нашего множества, |D*| = ℵ₀ + |T|
\end{itemize}
\end{enumerate}
\end{itemize}
\subsection{Парадокс Скулема}
\label{sec-17-4}
Мнимый парадокс Сколема формулируется так:
Возьмем теорию, прикрутим модель с аксиоматикой ZF. Модель
будет счетно-аксиоматизируемой потому что ZF.
Утверждается, что в ZF $\vdash \exists x(|x| = ℶ_1)$ - это доказывает диагон.
метод Кантора.
Тогда получается что по теореме Лёвингейма-Скулема у нашей
модели есть подмодель размером ℵ₀ + ℵ₀(счетно-акс) = ℵ₀, но
мы можем взять то самое x | |x| = ℶ_1, и его занумеровать, выходит.

Формальный подход не допускает этого конфликта ввиду одного
простого факта:
Рассмотрим отношение существования несчетного мн-ва в R.
$ZF \vdash \lnot \exists f$(f - биецкия между w и P(w)) \& \exists f(f - биецкия между w и w∪w)
// второе гарантирует счетность
Первый аргумент конъюнкции - \lnot \exists f(\forall x\forall y(<x,y>\in f \leftrightarrow x\in A\&x\in B))
<x,y> - пара (типа \{x, y, \{x\}\})
Тогда это значит, что в носителе модели нет такого f, что он
бы представлял собой объединение пар.
Собственно, по теореме Лёвингейма-Скулема у нас любая подмодель
будет иметь счетный носитель. Нет никакого противоречия, потому
что мы все еще работаем со счетным количеством множеств, а
отсутствие биекции все так же выражается отсутствием множества
в носителе.

