\section{Первая теорема о неполноте}
\label{sec-13}
\subsection{Непротиворечивость, \texorpdfstring{$\omega$}{ω}-непротиворечивость}
\label{sec-13-1}

\begin{definition}
    Теория \emph{непротиворечива}, если в ней нельзя вывести
    одновременно $a$ и $\lnot a$ (что аналогично невозможности
    вывести $a\land \lnot a)$.
\end{definition}

\begin{lemma}
    В противоречивой теории доказуема любая формула
\end{lemma}
\begin{proof}
    Противоречивость эквивалентна доказуемости $a \land \lnot a$. По \eqref{provability},
    \[a \to \lnot a \to b\]
    Кроме того, по \eqref{impltoand},
    \[(a \to \lnot a \to b) \to (a \land \lnot a \to b),\]
    из чего немедленно следует утверждение леммы.
\end{proof}

\begin{definition}
    Теория \emph{$\omega$-непротиворечива}, если из $\forall \phi(x) \vdash \phi(\overline{x})$ следует
    $\nvdash \exists p ~ \lnot \phi(p)$. Проще говоря, если мы взяли
    формулу, то невозможно вывести одновременно $\exists x ~ \lnot A(x)$
    и $A(0), A(1), \dotsc$
\end{definition}

\begin{lemma}[о $\omega$- и обычной непротиворечивости]
    Если теория $\omega$-непротиворечива, то она непротиворечива
\end{lemma}
\begin{proof}
    Рассмотрим следующую формулу:
\begin{gather*}
    \phi = (x=x \to x=x) \\
    \shortintertext{Такая формула очевидно доказуема $(A \to A)$}
    \vdash \phi[x:=k] k \in N_0 \\
    \shortintertext{Но недоказуемо $\exists x\lnot (x=x\to x=x)$}
\end{gather*}
Однако в противоречивой теории доказуемо всё. Значит, наша теория непротиворечива.
\end{proof}
\subsection{Первая теорема о неполноте}
\label{sec-13-2}
Определим отношение $W_1(x, p)$, истинное тогда и только тогда,
когда $x$ - геделев номер формулы $\phi$ с единственным свободным
аргументом $x$, а $p$ - геделев номер доказательства $\phi("\phi")$. Это
отношение выразимо в ФA, потому что мы просто пихаем это в наш
Proof, а его мы выразили через рекурсивные функции, а они
представимы.

Пусть его выражает $w_1(x, p)$;

Рассмотрим формулу $\sigma  = \forall p ~ \lnot w_1(x, p)$ -- для любого доказательства
оно не является доказательством самоприменения $\phi$, то есть
самоприменение $\phi$ недоказуемо.
То есть если $\sigma (\overline{\Godel{a}})$ истинно, то $a(\overline{\Godel{a}})$ недоказуемо.
В нашем случае если $\sigma (\overline{\Godel{a}})$ истинно, то $\sigma (\overline{\Godel{\sigma}})$ недоказуемо.
\begin{enumerate}
    \item Если формальная арифметика непротиворечива, то недоказуемо $\sigma (\overline{\Godel{\sigma}})$
    \begin{enumerate}
        \item Пусть $\vdash \sigma (\overline{\Godel{\sigma}})$, тогда найдется геделев номер ее док-ва p,
            тогда $W_1(\Godel{\sigma}, p)$, то есть $\vdash w_1(\overline{\Godel{\sigma}}, \overline{p})$.
        \item С другой стороны,
        \begin{gather*}
            \vdash \sigma (\overline{\Godel{\sigma}})\\
            \vdash \forall p ~ \lnot w_1(\overline{\Godel{\sigma}}, p)\\
            \forall p ~ \lnot w_1(\overline{\Godel{\sigma}}, p) \to \lnot w_1(\overline{\Godel{\sigma}}, \overline{p})\\
            \lnot w_1(\overline{\Godel{\sigma}}, \overline{p})
        \end{gather*}
        Тогда ФА противоречива.
    \end{enumerate}
    \item Если формальная арифметика $\omega$-непротиворечива, то недоказуемо $\lnot \sigma (\overline{\Godel{\sigma}})$
    Пусть $\vdash \lnot \sigma (\overline{\Godel{\sigma}})$,
    то есть $\vdash \lnot \forall p\lnot w_1(\overline{\Godel{\sigma}}, p)$, что значит
    $\exists p.w_1(\overline{\Godel{\sigma}}, p)$
    Найдется такой $q$, что $\vdash w_1(\overline{\Godel{\sigma}}, \overline{q})$, потому что если бы не нашелся,
    это бы значило доказуемость для каждого $q \lnot w_1(\overline{\Godel{\sigma}}, \overline{q})$, тогда по
    $\omega$-непротиворечивости было бы не доказуемо $\exists p\lnot \lnot w_1(\overline{\Godel{\sigma}}, p)$
    То $q$, что мы нашли -- это номер доказательства  $\sigma (\overline{\Godel{\sigma}})$, что и
    утверждает выражение $\vdash w_1(\overline{\Godel{\sigma}}, \overline{q})$. Но мы предполагали, что $\vdash \lnot \sigma (\overline{\Godel{\sigma}})$.
    Противоречие.
\end{enumerate}

\subsubsection*{Нормальное доказательство общезначимости}

Я не знаю, зачем нам второй пункт, но из первого следует, что если
наша теория $\omega$-непротиворечива, то она непротиворечива (по лемме выше),
значит в ней недоказуемо $\sigma (\overline{\Godel{\sigma}})$, то есть $\forall p ~ \lnot w_1(\overline{\Godel{\sigma}}, p)$, то есть
по корректности последнее выражение И, но это и есть в точности определение
$\sigma (\overline{\Godel{\sigma}})$.

\subsubsection*{Ненормальное д-во общезначимости}

Итого мы доказали, что если формальная арифметика $\omega$-непротиворечива,
то в ней не доказуемо ни $\sigma (\overline{\Godel{\sigma}})$ ни $\lnot \sigma (\overline{\Godel{\sigma}})$. Одно из них точно тавтология
(в формуле нет свободных переменных). Тогда ФА неполна при условии
$\omega$-непротиворечивости.

\subsubsection*{Другое доказательство общезначимости}

$\lnot \sigma (\overline{\Godel{\sigma}})$ недоказуема
\[
    \llbracket \sigma (\overline{\Godel{\sigma}}) \rrbracket = \llbracket \forall p\lnot w_1(\overline{\Godel{\sigma}}, p) \rrbracket =
    \begin{cases}
    \true & \text{если $\llbracket \lnot w_1(\overline{\Godel{\sigma}}, a)\rrbracket = \true$ для какого-то $a$} \\
    \false & \text{иначе}
    \end{cases}
\]

Это значит, что\\
И если $\llbracket w_1(\overline{\Godel{\sigma}}, a) \rrbracket = \false$\\
$\llbracket w_1(\overline{\Godel{\sigma}}, a) \rrbracket$ = \false, докажем от противного
Пусть $\llbracket \sigma (\overline{\Godel{\sigma}}) \rrbracket = \false$,\\
$\llbracket \forall p\lnot w_1(\overline{\Godel{\sigma}}, p) \rrbracket = \false$\\
$\llbracket \lnot \forall p\lnot w_1(\overline{\Godel{\sigma}}, p) \rrbracket = \true$\\
$\llbracket \exists p.w_1(\overline{\Godel{\sigma}}, p) \rrbracket = \true$\\
$\llbracket w_1(\overline{\Godel{\sigma}}, a) \rrbracket = \true$ для какого-то а\\
то есть a доказывает $\sigma (\overline{\Godel{\sigma}})$
\myworries{Какая-то шляпа, если честно}

???

тогда по определению $w_1$ существует
доказательство $\sigma (\overline{\Godel{\sigma}})$,
\subsection{Пример \texorpdfstring{$\omega$}{ω}-противоречивой, но непротиворечивой теории (при усл. непрот. ФА)}
\label{sec-13-3}
Добавим в ФА аксиому Г: $\lnot \sigma (\overline{\Godel{\sigma}})$
Тогда по контрпозиции 1п2 она $\omega$-противоречива.
Если бы мы могли доказать противоречивость нашей системы, то
ФА была бы противоречива, тогда хз
$\lnot \sigma (\overline{\Godel{\sigma}}) \vdash \sigma (\overline{\Godel{\sigma}})\land \lnot \sigma (\overline{\Godel{\sigma}})$
$\vdash \sigma (\overline{\Godel{\sigma}})$
Но мы предположили что $\lnot \sigma (\overline{\Godel{\sigma}})$
\subsection{Форма Россера}
\label{sec-13-4}
Если формальная арифметика непротиворечива, то в ней найдется
такая формула $\phi$, что $\nvdash \phi$ и $\nvdash \lnot \phi$
