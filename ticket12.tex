\section{Ticket 12: 2-ая теорема Гёделя о неполноте}
\label{sec-14}
\subsection{$Consis$, Условия выводимости Гильберта-Бернайса}
\label{sec-14-1}
Определим $Consis$ как утверждение, показывающее непротиворечивость ФА - отсутствие $\phi : \vdash \phi, \lnot \phi$. Поскольку
в противоречивой теории выводится что угодно, возьмем что-то недоказуемое, например $1 = 0$.\\
$Consis = \forall p(\lnot Proof(\overline{\Godel{1 = 0}}, p))$

Определим отношение $Sub(a, b, c)$ истинным, если $a$, $b - \Godel{a}$, $\Godel{b} \land c = \Godle{a[x := b]}$ или же $a \lor b$ не геделев номер и $c = 0$

Пусть $Sub(a, b, c)$ выражает $\tau(a, b, c)$
\begin{lemma}
Лемма о самоприменении
$a(x)$ - формула, тогда $\exists b$ такой что
\begin{enumerate}
\item $\vdash a(\overline{\Godel{b}}) \to b$
\item $\vdash \beta \to a(\overline{\Godel{b}})$
\end{enumerate}
\end{lemma}
\begin{proof}
$b_0(x) = \forall t(\tau(x, x, t) \to a(t))$\\
$b = b_0(\overline{\Godel{b_0}})$\\
\begin{enumerate}
\item $a(\overline{\Godel{b}}) \vdash a(\overline{\Godel{b}})$\\
$a(\overline{\Godel{b}}) \vdash \tau(\overline{\Godel{b_0}}, \overline{\Godel{b_0}}, \overline{\Godel{b}}) \to a(\overline{\Godel{b}})$    акс 1 + MP\\
$a(\overline{\Godel{b}}) \vdash \top \to (\tau(\overline{\Godel{b_0}}, \overline{\Godel{b_0}}, \overline{\Godel{b}}) \to a(\overline{\Godel{b}}))$\\
$a(\overline{\Godel{b}}) \vdash \top \to \forall t(\tau(\overline{\Godel{b}}, \overline{\Godel{b_0}}, t) \to a(t))$\\
$a(\overline{\Godel{b}}) \vdash \forall t(\tau(\overline{\Godel{b_0}}, \overline{\Godel{b_0}}, t) \to a(t))$\\
$a(\overline{\Godel{b}}) \vdash b$
\item $b \vdash \forall t(\tau(\overline{\Godel{b_0}}, \overline{\Godel{b_0}}, t) \to a(t))$    тут почти $a \vdash a$ написано\\
$b \vdash \tau(\overline{\Godel{b_0}}, \overline{\Godel{b_0}}, \overline{\Godel{b}})$             по выразимости\\
$b \vdash \tau(\overline{\Godel{b_0}}, \overline{\Godel{b_0}}, \overline{\Godel{b}}) \to a(\overline{\Godel{b}})$    сняли квантор с 1\\
$b \to a(\overline{\Godel{b}})$
\end{enumerate}
\end{proof}

Условия Гильберта-Бернайса \myworries{Лемма, теорема? хз чёт кароч}
\end{itemize}
Пусть πg(x, p) выражает Proof(x, p)
πr(x) = \exists t πg(x, t) тогда если
\begin{enumerate}
\item $\vdash a$ , то $\vdash πr(`a\textasciitilde{})$
\item $\vdash πr(`a\textasciitilde{}) \to πr(`πr(`a\textasciitilde{})\textasciitilde{})$
\item $\vdash πr(`a\textasciitilde{}) \to πr(`(a \to b)\textasciitilde{}) \to πr(`b\textasciitilde{})$
\end{enumerate}
\subsection{Вторая теорема о неполноте}
\label{sec-14-2}
\subsubsection{Рукомашеское доказательство без условий Г-Б}
\label{sec-14-2-1}
\begin{itemize}
\item Если арифметика непротиворечива, в ней нет д-ва Consis
рассмотрим Consis \to σ(`σ\textasciitilde{}).
Тогда если Consis доказуемо, то σ(`σ\textasciitilde{}) недоказуемо.
То есть это формулировка 1.1 Гёделя о неполноте.
Тогда если у нас будет Consis, мы сможем доказать
σ(`σ), тогда 1.1 фейлится. Значит Consis недоказуемо.

\item Доказательство того, что Consis недостаточно формален
Заменим Consis в д-ве на
Proof1(a, x) = Proof(a, x)\&\lnot Proof(`(1=0),x)
Consis1 = \forall x\lnot Proof1(`(1=0),x)
Если арифметика непротиворечива, то Proof1(a, x) = Proof(a, x)
Если арифметика противоречива, то Consis1 доказуема как и все
остальное.
Ну давайте менять.

Поменяли. Смотрим. хехехе, давайте докажем Consis1:
\lnot (π(x) \& \lnot π(x))              доказуемо в ИВ
\top
\top \to \lnot (π(x) \& \lnot π(x))          1 акс, MP
\lnot (π(x) \& \lnot π(x))
\forall x(\lnot (π(x) \& \lnot π(x)))

Тогда выходит, что мы можем доказать противоречивость арифметики.
Но это не так, бага вот в чем:
Замена consis на consis1 неоправдана - в consis1 есть
формула 1=0, на которой ее результат не вычисляется, а
постулируется.
Чтобы выражать Consis абстрактно, существуют условия выводимости
Гильберта-Бернайса.

Докажем, что consis1 не удовлетворяет 3 свойству Г-Б
Пусть Proof1(x,p) выражает π1.
$\vdash π1(`a\textasciitilde{}) \to π1(`a\to b\textasciitilde{}) \to π1(`b\textasciitilde{})$ оценим при a=(2=0), b=(1=0)
?       \to (   true  \to  false)
?       \to false
Если эта формула верна, то $\vdash π1(`a\textasciitilde{}$)
Тогда если π1(`a\textasciitilde{}), то Proof(2=0, x)\&\lnot Proof(`1=0, x) = И
Это значит что теория противоречива, потому что в ней выводимо 2=0,
но она непротиворечива, потому что недоказуемо 1=0. \to ←
\end{itemize}
\subsubsection{Доказательство 2 теоремы Гёделя о неполноте}
\label{sec-14-2-2}
Пусть π удовлетворяет условиям Г-Б
Consis = \lnot π(1=0)
ФА непротиворечива
Тогда ⊬ Consis

\begin{enumerate}
\item По лемме о самоприменении
\begin{enumerate}
\item \lnot π(γ) \to γ
\item γ \to \lnot π(γ)
\item \lnot γ \to π(γ)                            контрпозиция
\item π(γ) \to \lnot γ
\end{enumerate}
\item π(γ) \to π(\lnot γ)
\begin{enumerate}
\item $π(γ) \vdash π(`π(γ)\textasciitilde{})$                     ГБ2
\item $\vdash π(π(γ) \to \lnot γ)$                       ГБ1 от 1.4
\item $\vdash π(π(γ)) \to π(π(γ) \to \lnot γ) \to π(\lnot (γ))$   ГБ3
\item $π(γ) \vdash π(\lnot γ)$                         2MP (2.1, 2.2)
\end{enumerate}
\item $\vdash π(α \to \beta \to γ) and \vdash π(α) \to π(\beta) => \vdash π(α) \to π(γ)$
\begin{enumerate}
\item π(α \to \beta \to γ) \to π(α) \to π(b \to γ)        ГБ3
\item π(\beta \to γ) \to π(\beta) \to π(γ)                ГБ3
\item π(α) \to π(\beta \to γ)                       MP 1, given
\item π(α) \to π(\beta)                           given
\item π(α) \to π(γ)                           занести под дедукцию, ГБ3
\end{enumerate}
\item $\vdash π(γ) \to π(1=0)$
\begin{enumerate}
\item γ \to \lnot γ \to (1=0)                        10i в ИИВ, выводима в предикатах
\item $\vdash π(γ \to \lnot γ \to (1=0))$                   ГБ1
\item π(γ) \to π(\lnot γ)                          2
\item $\vdash π(γ) \to π(1=0)$                       MP 4.2 4.3
\end{enumerate}
\item ⊬ Consis
$\vdash \lnot π(1=0) \to \lnot π(γ)$                        контрапозиция 4
$\vdash Consis \to \lnot π(γ)$                         the same
$] \vdash Consis$, тогда $\vdash \lnot π(γ)$
$\vdash \lnot π(γ) \to γ => \vdash γ => \vdash π(γ)$            1.1, ГБ1
$\vdash \lnot π(γ), \vdash π(γ)  \to ←$
\end{enumerate}

