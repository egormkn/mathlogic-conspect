\section{Ticket 12: 2-ая теорема Гёделя о неполноте}
\label{sec-14}
\subsection{Consis, Условия выводимости Гильберта-Бернайса}
\label{sec-14-1}
Определим $Consis$ как утверждение, показывающее непротиворечивость ФА - отсутствие $\phi$: $\vdash \phi, \lnot \phi$. Поскольку
в противоречивой теории выводится что угодно, возьмем что-то недоказуемое, например $1 = 0$.\\
$Consis = \forall p(\lnot Proof(\overline{\Godel{1 = 0}}, p))$

Определим отношение $Sub(a, b, c)$ истинным, если $a$, $b - \Godel{a}$, $\Godel{b} \land c = \Godel{a[x := b]}$ или же $a \lor b$ не геделев номер и $c = 0$

Пусть $Sub(a, b, c)$ выражает $\tau(a, b, c)$
\begin{lemma}
Лемма о самоприменении
$a(x)$ - формула, тогда $\exists b$ такой что
\begin{enumerate}
\item $\vdash a(\overline{\Godel{b}}) \to b$
\item $\vdash \beta \to a(\overline{\Godel{b}})$
\end{enumerate}
\end{lemma}
\begin{proof}
$b_0(x) = \forall t(\tau(x, x, t) \to a(t))$\\
$b = b_0(\overline{\Godel{b_0}})$
\begin{enumerate}
\item $a(\overline{\Godel{b}}) \vdash a(\overline{\Godel{b}})$\\
$a(\overline{\Godel{b}}) \vdash \tau(\overline{\Godel{b_0}}, \overline{\Godel{b_0}}, \overline{\Godel{b}}) \to a(\overline{\Godel{b}})$    акс 1 + MP\\
$a(\overline{\Godel{b}}) \vdash \top \to (\tau(\overline{\Godel{b_0}}, \overline{\Godel{b_0}}, \overline{\Godel{b}}) \to a(\overline{\Godel{b}}))$\\
$a(\overline{\Godel{b}}) \vdash \top \to \forall t(\tau(\overline{\Godel{b}}, \overline{\Godel{b_0}}, t) \to a(t))$\\
$a(\overline{\Godel{b}}) \vdash \forall t(\tau(\overline{\Godel{b_0}}, \overline{\Godel{b_0}}, t) \to a(t))$\\
$a(\overline{\Godel{b}}) \vdash b$
\item $b \vdash \forall t(\tau(\overline{\Godel{b_0}}, \overline{\Godel{b_0}}, t) \to a(t))$    тут почти $a \vdash a$ написано\\
$b \vdash \tau(\overline{\Godel{b_0}}, \overline{\Godel{b_0}}, \overline{\Godel{b}})$             по выразимости\\
$b \vdash \tau(\overline{\Godel{b_0}}, \overline{\Godel{b_0}}, \overline{\Godel{b}}) \to a(\overline{\Godel{b}})$    сняли квантор с 1\\
$b \to a(\overline{\Godel{b}})$
\end{enumerate}
\end{proof}
\begin{lemma}
Условия Гильберта-Бернайса \myworries{Лемма, теорема? хз чёт кароч}\\
Пусть $\pi g(x, p)$ выражает $Proof(x, p)$\\
$\pi r(x) = \exists t \pi g(x, t)$ тогда если
\begin{enumerate}
\item $\vdash a$, то $\vdash \pi r(\overline{\Godel{a}})$
\item $\vdash \pi r(\overline{\Godel{a}}) \to \pi r(\overline{\Godel{\pi r(\overline{\Godel{a}})}})$
\item $\vdash \pi r(\overline{\Godel{a}}) \to \pi r(\overline{\Godel{a \to b}}) \to \pi r(\overline{\Godel{b}})$
\end{enumerate}
\end{lemma}
\subsection{Вторая теорема о неполноте}
\label{sec-14-2}
\begin{theorem}
$Consis$ не доказуем
\end{theorem}
\subsubsection{Рукомашеское доказательство без условий Гильберта-Бернайса}
\label{sec-14-2-1}
\begin{proof}
Докажем несколько различных фактов
\begin{itemize}
\item Если арифметика непротиворечива, в ней нет доказательства $Consis$:\\
Рассмотрим $Consis \to \sigma(\overline{\Godel{\sigma}})$. Тогда если $Consis$ доказуемо, то $\sigma(\overline{\Godel{\sigma}})$ недоказуемо. Тоесть это формулировка 1-ой теоремы Гёделя о неполноте.\\
Тогда если у нас будет $Consis$, мы сможем доказать $\sigma(\overline{\Godel{\sigma}})$, тогда 1-ая теорема Гёделя о неполноте не работает. Значит $Consis$ недоказуемо.
\item Доказательство того, что $Consi$s недостаточно формален:\\
Заменим $Consis$ в д-ве на:\\
$Proof1(a, x) = Proof(a, x) \land \lnot Proof(\Godel{1=0}, x)$\\
$Consis1 = \forall x \lnot Proof1(\Godel{1=0}, x)$\\
Если арифметика непротиворечива, то $Proof1(a, x) = Proof(a, x)$\\
Если арифметика противоречива, то $Consis1$ доказуема как и все остальное.
Ну давайте менять.

Поменяли. Смотрим. хехехе, давайте докажем $Consis1$:\\
$\lnot (\pi(x) \& \lnot \pi(x))$ (Доказуемо в ИВ)\\
$\top$\\
$\top \to \lnot (\pi(x) \& \lnot \pi(x))$ (1 акс, MP)\\
$\lnot (\pi(x) \& \lnot \pi(x))$\\
$\forall x(\lnot (\pi(x) \& \lnot \pi(x)))$

Тогда выходит, что мы можем доказать противоречивость арифметики. Но это не так, бага вот в чем:
Замена $Consis$ на $Consis1$ неоправдана -- в $Consis1$ есть формула $1=0$, на которой ее результат не вычисляется, а
постулируется. Чтобы выражать $Consis$ абстрактно, существуют условия выводимости Гильберта-Бернайса.

Докажем, что $Consis1$ не удовлетворяет 3-ему свойству Г-Б:\\
Пусть $Proof1(x,p)$ выражает $\pi 1$.
$\vdash \pi 1(\overline{\Godel{a}}) \to \pi 1(\overline{\Godel{a \to b}}) \to \pi 1(\overline{\Godel{b}})$ оценим при $a=(2=0)$,$b=(1=0)$\\
$? \to (true \to false)$\\
$? \to false$

Если эта формула верна, то $\vdash \pi 1(\overline{\Godel{a}})$
Тогда если $\pi 1(\overline{\Godel{a}})$, то $Proof(2=0, x) \land \lnot Proof(`1=0, x) = И$\\
Это значит что теория противоречива, потому что в ней выводимо $2=0$, но она непротиворечива, потому что недоказуемо $1=0$.
\end{itemize}
\end{proof}
\subsubsection{Доказательство 2 теоремы Гёделя о неполноте}
\label{sec-14-2-2}
\begin{proof}
Пусть $\pi$ удовлетворяет условиям Гильебрта-Бернайса:\\
$Consis = \lnot \pi(1=0)$\\
ФА непротиворечива\\
Тогда $\not \vdash Consis$

\begin{enumerate}
\item По лемме о самоприменении
\begin{enumerate}
\item $\lnot \pi(\gamma) \to \gamma$
\item $\gamma \to \lnot \pi(\gamma)$
\item $\lnot \gamma \to \pi(\gamma)$ (контрпозиция)
\item $\pi(\gamma) \to \lnot \gamma$
\end{enumerate}
\item $\pi(\gamma) \to \pi(\lnot \gamma)$
\begin{enumerate}
\item $\pi(\gamma) \vdash \pi(\overline{\Godel{\pi(\gamma)}})$ (ГБ 2)
\item $\vdash \pi(\pi(\gamma) \to \lnot \gamma)$ (ГБ 1 от 1.4)
\item $\vdash \pi(\pi(\gamma)) \to \pi(\pi(\gamma) \to \lnot \gamma) \to \pi(\lnot (\gamma))$ (ГБ 3)
\item $\pi(\gamma) \vdash \pi(\lnot \gamma)$  (2 MP (2.1, 2.2))
\end{enumerate}
\item $\vdash \pi(\alpha \to \beta \to \gamma) and \vdash \pi(\alpha) \to \pi(\beta) => \vdash \pi(\alpha) \to \pi(\gamma)$
\begin{enumerate}
\item $\pi(\alpha \to \beta \to \gamma) \to \pi(\alpha) \to \pi(b \to \gamma)$ (ГБ 3)
\item $\pi(\beta \to \gamma) \to \pi(\beta) \to \pi(\gamma)$ (ГБ 3)
\item $\pi(\alpha) \to \pi(\beta \to \gamma)$ (MP 1, given)
\item $\pi(\alpha) \to \pi(\beta)$ (given)
\item $\pi(\alpha) \to \pi(\gamma)$ (занести под дедукцию, ГБ 3)
\end{enumerate}
\item $\vdash \pi(\gamma) \to \pi(1=0)$
\begin{enumerate}
\item $\gamma \to \lnot \gamma \to (1=0)$ (10i в ИИВ, выводима в предикатах)
\item $\vdash \pi(\gamma \to \lnot \gamma \to (1=0))$ (ГБ1)
\item $\pi(\gamma) \to \pi(\lnot \gamma)$ 2
\item $\vdash \pi(\gamma) \to \pi(1=0)$ (MP 4.2, 4.3)
\end{enumerate}
\item $\not \vdash Consis$
$\vdash \lnot \pi(1=0) \to \lnot \pi(\gamma)$ (контрапозиция 4)\\
$\vdash Consis \to \lnot \pi(\gamma)$ (the same)\\
$] \vdash Consis$, тогда $\vdash \lnot \pi(\gamma)$\\
$\vdash \lnot \pi(\gamma) \to \gamma \Rightarrow \vdash \gamma \Rightarrow \vdash \pi(\gamma)$ (1.1, ГБ1)\\
$\vdash \lnot \pi(\gamma), \vdash \pi(\gamma)$
\end{enumerate}
\end{proof}
