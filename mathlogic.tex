% Created 2015-01-21 Ср 21:06
%\documentclass[draft,12pt]{article}
\documentclass[fleqn,12pt]{article}
\usepackage{euler}
%\usepackage[normalem]{ulem}
\usepackage{amsmath}
\usepackage{epigraph}
\usepackage{mathtools}
\usepackage{amssymb}
\usepackage{proof} % M. P. и его друзья
\usepackage[pdftitle={Курс матлогики по Штукенбергу Д. Г.}]{hyperref}
\usepackage{amsthm}
\usepackage{comment}
\usepackage{stmaryrd} % ⟦⟧
\usepackage{enumitem} % всякие модные enumerate-ы (latin letters, roman numbers etc)
\usepackage{xcolor} % red color for worries
\usepackage{pict2e} % templates
%\usepackage{enumerate}
%\usepackage{changepage}
\usepackage[left=2cm,right=1.6cm,top=2cm,bottom=2cm,bindingoffset=0cm]{geometry}
\usepackage{microtype} % модная микротипографика, нифига не работающая с XeLaTeX (ну, почти)
\usepackage{fontspec}
\setmainfont[Ligatures=TeX]{Palatino Linotype}
\setmonofont{DejaVu Sans Mono}
\usepackage{polyglossia}
\setdefaultlanguage[babelshorthands=true]{russian}

%\hyphenpenalty=5000


% some magic for Godel numerals
\newbox\gnBoxA
\newdimen\gnCornerHgt
\setbox\gnBoxA=\hbox{$\ulcorner$}
\global\gnCornerHgt=\ht\gnBoxA
\newdimen\gnArgHgt
\def\Godel #1{%
\setbox\gnBoxA=\hbox{$#1$}%
\gnArgHgt=\ht\gnBoxA%
\ifnum     \gnArgHgt<\gnCornerHgt \gnArgHgt=0pt%
\else \advance \gnArgHgt by -\gnCornerHgt%
\fi \raise\gnArgHgt\hbox{$\ulcorner$} \box\gnBoxA %
\raise\gnArgHgt\hbox{$\urcorner$}}

\renewcommand{\le}{\leqslant} % ⩽
\renewcommand{\leq}{\leqslant} % ⩽
\renewcommand{\ge}{\geqslant} % ⩾
\renewcommand{\geq}{\geqslant} % ⩾
\renewcommand{\phi}{\varphi}
\renewcommand{\epsilon}{\varepsilon}
\DeclareMathOperator{\plog}{plog}
\DeclareMathOperator{\Int}{Int}
\DeclareMathOperator{\Proof}{Proof}

\newcommand{\ltemplate}{\begin{picture}(5,7)
\put(.5,4){\line(1,-1){5}}
\put(.2,4.1){\line(1,-1){5.2}} % just to make it more bold lol
\put(.5,4){\line(1,1){5}}
\put(.2,3.9){\line(1,1){5.2}} % just to make it more bold lol
\end{picture}}
\newcommand{\rtemplate}{\begin{picture}(5,7)
\put(4.5,4){\line(-1,-1){5}}
\put(4.8,4.1){\line(-1,-1){5.2}} % just to make it more bold lol
\put(4.5,4){\line(-1,1){5}}
\put(4.8,3.9){\line(-1,1){5.2}} % just to make it more bold lol
\end{picture}}
\newcommand{\template}[1]{\ltemplate #1\rtemplate}

\newcommand{\defeq}{\coloneqq}
\newcommand{\s}[1]{\texttt{#1}}
\newcommand{\xl}{$\lambda$}
\newcommand{\+}{\lambda}
\newcommand{\bredmath}{\ \longrightarrow_\beta\ }
\newcommand{\bred}{$\bredmath$}
\newcommand{\mbred}{$\ \longrightarrow\!\!\!\!\rightarrow_\beta\ $}
\newcommand{\lid}[1]{\textit{#1}}
\newcommand{\concat}{\hat{\ \ }}

\newcommand\myworries[1]{\textcolor{red}{#1}}

\def\ra{\rightarrow}

\tolerance 1000 % чтобы не очковал переносить

\renewcommand{\theenumii}{\asbuk{enumii}}
\AddEnumerateCounter{\asbuk}{\@asbuk}{ы}

\DeclareRobustCommand{\divby}{%
  \mathrel{\vbox{\baselineskip.65ex\lineskiplimit0pt\hbox{.}\hbox{.}\hbox{.}}}%
}
\newcommand{\perc}{\mathbin{\%}}
%\newenvironment{epigraph}%
%{\begin{list}{}{\setlength{\leftmargin}{0.3\textwidth}}\item[]}%
%{\end{list}}

\author{Daniyar Itegulov, Aleksei Latyshev, Ignat Loskutov}
\date{\today}
\title{Курс математической логики по Штукенбергу Д.~Г.}

\begin{document}
\theoremstyle{definition}
\newtheorem*{definition}{Определение}%[section]
\newtheorem*{example}{Пример}
%\theoremstyle{theorem}
\newtheorem{theorem}{Теорема}[section]
\newtheorem{axiom}{Аксиома}[section]
\newtheorem{lemma}[theorem]{Лемма}

\maketitle
\tableofcontents

%Должно отображаться корректно: $x_1$,x₂x₃x_n, θ, \exists , ∑, ∉
\epigraph{%
    лан, всё не книжку верстаем))))000}
  {некто Игнат Лоскутов о качестве вёрстки}
\epigraph{%
    данияр лолка пиздос)))))}
  {аноним о всяких там хех латехерах}

Mykhail Volkhov, 2538, 2014Sep-2015Jan\\
Я не отвечаю за верность написанного - много информации
я придумал сам, много достал из недостоверных источников.
\renewcommand{\thesection}{\Roman{section}}
\section*{Базовые понятия}
\addcontentsline{toc}{section}{Базовые понятия}
\label{sec-1}
\subsection*{Формальные системы и модели}
\label{sec-1-1}
Сделано мной для меня самого, be careful

Мы работаем с формальными системами.
Формальная система определяется сигнатурой, грамматикой,
набором аксиом и набором правил вывода.
\begin{enumerate}
\item Сигнатура ФС -- это (Pr, F, C, Links, Misc, arity):
\begin{itemize}
\item $\mathrm Pr$ -- описывает предикаты (число + заглавная буква латинского алфавита)
\item $\mathrm F$ -- множество функций (заглавные буквы латинского алфавита)
\item $\mathrm C$ -- описывает константы
\item Links -- множество связок ($\lbrace$«$\to$», «$\cup$», «пробел»$\rbrace$)
\item Misc -- дополнительные элементы ($\lbrace$<<$($>>, <<$)$>>, <<пробел>>$\rbrace$)
\item $arity\colon Foo \cup Pr \cup C \to \mathbb N$ возвращает арность
\end{itemize}
\item Грамматика описывает то, как мы можем строить выражения
в соответствии с нашей сигнатурой.
\item Аксиомы -- выражения в нашей грамматике.
\item Правила вывода -- пары вида (List, List), где List --
список утверждений. Первый элемент – посылки, второй --
то, что из них следует.
\end{enumerate}

Иногда нам хочется что-то посчитать и мы прикручиваем к
формальной системе модель -- корректную структуру с оценкой.
Структура -- это сигнатура с интерпретацией и носителем.
\begin{enumerate}
    \item Сигнатура структуры -- (R, F, C, arity):
    \begin{itemize}
        \item $Pr$ -- множество символов для предикатов
        \item $F$ -- функциональных символов
        \item $C$ -- символов констант
        \item $arity$ – функция, определяющая арность $Pr \cup F \to \mathbb N$.
    \end{itemize}
    \item Интерпретация -- это приписывание символам значения
    и правил действия (отображения из $Pr \cup F \cup C$ в носитель)
    \item Носитель -- это объединение множеств, в котором обязательно
    присутствует $V$ -- множество истинностных значений. Если же
    мы рассматриваем только нульместные предикаты, на этом
    можно остановиться, otherwise часто вводится P -- предметное
    множество, в которое отображаются элементы из $F, C$.
\end{enumerate}
TODO Эта реализация структуры не определяет ничего в районе
аксиоматики, но аксиоматически заданные структуры существуют
– например в ФА есть Пеано.

Если все аксиомы тавтологии, то структура корректна.
В таком случае она называется моделью.

Оценку иногда определяют раньше/позже чем модель, мне
удобно думать о ней, как об отдельной сущности, потому что
она связывает модель с ФС.

Оценка -- это функция оценки и функция тавтологии.
\begin{enumerate}
    \item Функция оценки -- отображение из (множества всех формул,
        сгенеренных грамматикой) $\times$ (какие-нибудь допаргументы)
        в $V$ модели. Дополнительные аргументы -- например оценки
        элементов связки.
    \item Функция тавтологии -- отображение из множества формул
        грамматики в $\{0, 1\}$ -- является ли формула тавтологией.
        Тавтология использует функцию оценки. Например, тавтология
        -- это выражение, оценка которого на любых аргументах
        возвращает $\sigma \in V$ -- какой-то элемент $V$.
\end{enumerate}

Когда говорится <<сигнатура модели>> -- имеется в виду ровно она.
Когда говорится <<сигнатура ФС>> -- имеется в виду скорее всего
объединение сигнатур, а может только сигнатура самой ФС. Первый
вариант тут предпочтительней.

\section*{Определения (нужно знать идеально)}
\addcontentsline{toc}{section}{Определения}
\label{sec-2}
Определения тут зачастую дублируют то, что написано в самом
конспекте, поэтому удаление этого блока сэкономит бумагу при
печати.
\subsection*{ИВ}
\label{sec-2-1}
Формальная система с алгеброй Яськовского $J_{0}$ в качестве модели, множество истинностных значений $\lbrace0, 1\rbrace$.
Формальная теория нулевого порядка, кванторов нету, предикаты -- это пропозициональные переменные.
Аксиомы:
\begin{enumerate}
\item $\alpha \to \beta \to \alpha$
\item $(\alpha \to \beta) \to (\alpha \to \beta \to \gamma) \to (\alpha \to \gamma)$
\item $\alpha \to \beta \to \alpha \land \beta$
\item $\alpha \land \beta \to \alpha$
\item $\alpha \land \beta \to \beta$
\item $\alpha \to \alpha \lor \beta$
\item $\beta \to \alpha \lor \beta$
\item $(\alpha \to \beta) \to (\gamma \to \beta) \to (\alpha \lor \gamma \to \beta)$
\item $(\alpha \to \beta) \to (\alpha \to \lnot \beta) \to \lnot \alpha$
\item $\lnot \lnot \alpha \to \alpha$
\end{enumerate}
\subsection*{Общезначимость, доказуемость, выводимость}
\label{sec-2-2}
\begin{itemize}
\item Общезначимость формулы -- ее свойство в теории с моделью.
Общезначимость можно определить как угодно, в принципе.
Например в ИВ общезначимость -- это что оценка формулы на любых значениях свободных переменных отображает в 1.
В модели крипке -- существование формулы во всех мирах и т.д.
\item Доказуемость -- свойство формулы в теории, значащее, что существует
доказательство для этой формулы. Доказательство для теории тоже определяется
по разному (последовательность утверждений, каждое из которых есть аксиома
или следует по правилу вывода из предыдущих в ИВ, дерево с выводами в $S_{\infty}$)
\item Выводимость -- в общем случае часто используется как аналог доказуемости,
в ИВ это доказуемость из всего, что и ранее + из посылок.
\end{itemize}
\subsection*{Теорема о дедукции для ИВ}
\label{sec-2-3}
Теорема, утверждающая, что из $\Gamma, \alpha \vdash \beta$ следует $\Gamma \vdash \alpha \to \beta$ и наоборот.\\
Доказывается вправо поформульным преобразованием, влево
добавлением 1 формулы. Работает в ИВ, ИИВ, предикатах.
\subsection*{Теорема о полноте исчисления высказываний}
\label{sec-2-4}
\theorem[о полноте исчисления высказываний]{
Исчисление предикатов полно}.\\
Общий ход д-ва: строим док-ва для конкретных наборов перменных,
$2^n$, где $n$ -- количество возможных переменных. Потом их мерджим.
\subsection*{ИИВ}
\label{sec-2-5}
Берем ИВ, выкидываем 10 аксиому, добавляем $\alpha \rightarrow \neg \alpha \rightarrow \beta$.\\
Она доказывается и в ИВ:
\begin{lemma}
$\alpha, \alpha \vee \neg \alpha, \neg \alpha \vdash \beta$
\end{lemma}
\begin{tabular}{lll}
(1) & $\alpha$& Допущение\\
(2) & $\neg \alpha$& Допущение\\
(3) & $\alpha \rightarrow \neg \beta \rightarrow \alpha$& Сх. акс. 1\\
(4) & $\neg \beta \rightarrow \alpha$& M.P. 1,3\\
(5) & $\neg \alpha \rightarrow \neg \beta \rightarrow \neg \alpha$& Сх. акс. 1\\
(6) & $\neg \beta \rightarrow \neg \alpha$& M.P. 2,5\\
(7) & $(\neg \beta \rightarrow \alpha) \rightarrow (\neg \beta \rightarrow \neg \alpha) \rightarrow (\neg \neg \beta)$& Сх. акс. 9\\
(8) & $(\neg \beta \rightarrow \neg \alpha) \rightarrow (\neg \neg \beta)$& M.P. 4,7\\
(9) & $\neg \neg \beta$& M.P. 6,8\\
(10) & $\neg \neg \beta \rightarrow \beta$& Сх. акс. 10\\
(11) & $\beta$& M.P. 9,10\\
\end{tabular}\\
А еще в ИИВ главная фишка -- недоказуемо $\alpha \lor \lnot \alpha$ (можно подобрать такую модель).
\subsection*{Теорема Гливенко}
\label{sec-2-6}
\begin{theorem}[Гливенко]
Если в ИВ доказуемо $\alpha$, то в ИИВ доказуемо $\lnot \lnot \alpha$
\end{theorem}

Общий ход д-ва: говорим, что если в ИИВ доказуема $\delta_{i}$,
то в ней же доказуема $\lnot \lnot \delta_{i}$. Доказываем руками двойное
отрицание 10 аксиомы и то же самое для MP.
\subsection*{Порядки}
\label{sec-2-7}
\begin{definition}
    Частичный порядок – рефлексивное, антисимметричное, транзитивное
отношение.
\end{definition}
\begin{definition}
    Частично упор. мн-во -- множество с частичным порядком на элементах.
\end{definition}
\begin{definition}
    Линейно упорядоч. мн-во -- множество с частичным порядком, в котором
    два любых элемента сравнимы.
\end{definition}
\begin{definition}
    Фундированное мн-во -- частично упорядоч. множество, в котором каждое
    непустое подмножество имеет минимальный элемент.
\end{definition}
\begin{definition}
    Вполне упорядоченное множество -- фундированное множество с линейным
    порядком.
\end{definition}
\subsection*{Решетки (все свойства)}
\label{sec-2-8}
\begin{definition}
    Решетка -- это $(L, +, *)$ в алгебраическом смысле
    и $(L, \le)$ в порядковом.
\end{definition}

Решетку можно определить как алгебраическую структуру через
аксиомы: коммутативность, ассоциативность, поглощение.

Решетку можно определить как упорядоченное множество через
множество с частичным порядком на нем, тогда операции $+$, $*$ определяются
как $\sup$ и $\inf$:
\begin{align*}
    \sup p &= \min \{u \mid u \ge all\ s \in p\} \\
    \inf p &= \max \{u \mid u \le all\ s \in p\} \\
    a + b  &= \sup \{a, b\} \\
    a * b  &= \inf \{a, b\}
\end{align*}
Если для двух элементов всегда можно определить $a + b$ и $a * b$, то такое
множество назывется решеткой.

\begin{definition}
    Дистрибутивная решетка -- решетка, в которой работает дистрибутивность:
    $a * (b + c) = (a * b) + (a * c)$
\end{definition}

\begin{definition}
    Импликативная решетка -- всегда существует псевдодополнение b ($b \to a$)
    $a \to b = \max \lbrace c \mid c \times a \le b \rbrace$
\end{definition}
Имеет свойствa, что в ней всегда есть максимальный элемент $a \to a$ и что
она дистрибутивна.

\subsection*{Булевы/псевдобулевы алгебры}
\label{sec-2-9}
\begin{itemize}
\item Булеву алгебру можно определить так:
\begin{enumerate}
\item $(L, +, *, -, 0, 1)$ с выполненными аксиомами -- коммутативность, ассоциативность,
    поглощение, две дистрибутивности и $a * -a = 0$,
    $a + -a = 1$.
\item Импликативная решетка над фундированным множеством.

Тогда мы в ней определим $1$ как $a \to a$ (традиционно для импликативной),
отрицание как $-a = a \to 0$, и тогда последняя аксиома из
предыдущего определения будет свойством:
\[a * -a = a * (a \to 0) = a * (\max c: c * a \le 0) = a * 0 = 0\]
Насчет второй аксиомы -- должно быть 1. То есть лучше как-то
через аксиомы определять, видимо.
\[a + -a = a + (a \to 0) = a + (\max c: c * a \le 0) = a + 0 = a\] // не 1
\end{enumerate}
\item Псевдобулева алгебра -- это импликативная решетка над фундированным
множеством с $\lnot a = (a \to 0)$
\end{itemize}
\subsection*{Топологическая интерпретация ИИВ}
\label{sec-2-10}
Булеву алгебру и алгебру Гейтинга можно интерпретировать
на множестве $\mathbb{R}^{n}$. Тогда заключения о общезначимости формулы
можно делать более наглядно.
Давайте возьмем в качестве множества алгебры все открытые
подмножества $\mathbb{R}^{n}$. Определим операции следующим образом:
\begin{enumerate}
\item $a + b \defeq a \cup b$
\item $a * b \defeq a \cap b$
\item $a \to b \defeq Int(a^c \cup b)$
\item $-a \defeq \Int(a^c)$
\item $0 \defeq \emptyset$
\item $1 \defeq \bigcup\{\text{всех мн-в в $L$}\}$
\end{enumerate}
\subsection*{Модель Kрипке}
\label{sec-2-11}
$Var = \{P, Q, \dotsc\}$
Модель Крипке – это $\langle W, \leq, v\rangle$, где
\begin{itemize}
\item $W$ -- множество <<миров>>
\item $\leq$ -- частичный порядок на W (отношение достижимости)
\item $v \colon W \times Var \to \{0, 1, \_\}$ -- оценка перменных на $W$, монотонна
(если $v(x, P) = 1$, $x \leq y$, то $v(y, P) = 1$ -- формулу нельзя un'вынудить)
\end{itemize}

Правила:
\begin{itemize}
    \item $W, x \vDash  P \Leftrightarrow v(x, P) = 1 \text{, если $P \in Var$}$
\item $W, x \vDash  (A \land B) \Leftrightarrow W, x \vDash  A \land W, x \vDash  B$
\item $W, x \vDash  (A \lor B) \Leftrightarrow W, x \vDash  A \lor W, x \vDash  B$
\item $W, x \vDash  (A \to B) \Leftrightarrow \forall  y \ge x (W, y \vDash  A \Rightarrow W, y \vDash  B)$
\item $W, x \vDash  \lnot A \Leftrightarrow \forall  y \in x (W, x \lnot \vDash  A)$
\end{itemize}

В мире разрешается быть не вынужденной переменной и ее отрицанию
одновремеменно.
Формула называется тавтологией в ИИВ с моделью Крипке, если она
истинна (вынуждена) в любом мире любой модели Крипке.
\subsection*{Вложение Крипке в алгебры Гейтинга}
\label{sec-2-12}
Возьмем модель Крипке, возьмем какое-то объединение поддеревьев
со всеми потомками, каждое такое объединение пусть будет входить
в алгебру Гейтинга. $\le$ -- отношение <<быть подмножеством>>.
Определим $0$ как $\emptyset$ (пустое объединение поддеревьев);
Определим операции:
\begin{align*}
    + &= \cup,\\
    * &= \cap,\\
    a \to b &= \bigcup \{z \in H \mid z \le x^c \cup  y\}
\end{align*}
Так созданное множество с операциями является импликативной
решеткой, в которой мы определим $-a = a \to 0$, получим булеву алгебру.
\subsection*{Полнота ИИВ в алгебрах Гейтинга и моделях Крипке}
\label{sec-2-13}
ИИВ полно относительно алгебр Гейтинга и моделей Крипке.

Общий ход доказательства первого сводится к вложению
в Гейтинга алгебры Линденбаума-Тарского, а второго -
к построению дизъюнктивного множества всех доказуемых
формул, являющегося миром Крипке.
\subsection*{Нетабличность ИИВ}
\label{sec-2-14}
Не существует полной модели, которая может быть выражена таблицей
(конечной -- алгебра Гейтинга и Крипке не табличны, так как и там и
там связки определяются иначе).

От противного соорудим табличную модель и покажем, что она не полна,
приведя пример большой дизъюнкции из импликаций, для которой можно
построить модель Крипке в которой она не общезначима.
\subsection*{Предикаты}
\label{sec-2-15}
Теория первого порядка, расширяющая исчисление высказываний.
Добавляются две новые аксиомы
$\forall x.A \to A[x:=\eta]$, где $\eta$ свободна для подстановки в A
$A[x:=\eta] \to \exists x.A, -//-$

Правила вывода:
\[\infer{A \to \forall x . B}{A \to B}\]
 x не входит сводобно в А

\[\infer{\exists x . A \to B}{A \to B}\]
 x не входит свободно в В
\subsection*{Теорема о дедукции в предикатах}
\label{sec-2-16}
Аналогично 1 теореме о дедукции в ИВ, но в доказательстве должны
отсутствовать применения правил для кванторов по переменным входящих
свободно в выражение $\gamma$
\[\Gamma, \gamma \vdash a \Rightarrow \Gamma \vdash \gamma \to a\]
\subsection*{Теорема о полноте исчисления предикатов}
\label{sec-2-17}
Исчисление предикатов полно (заметим, что относительно любой модели).
Суть в том, что если предикаты непротиворечивы, то у них есть модель.
Если у них есть модель, то типа там можно по контрпозиции показать $\vDash a$.
\subsection*{Теории первого порядка, определение структуры и модели}
\label{sec-2-18}
Теория первого порядка -- это формальная система с кванторами по
функциональным символам, но не по предикатам. Рукомахательное
определение – это фс с логикой первого порядка в основе, в которой
абстрактные предикаты и функциональные символы определяются точно
(а может такое определение даже лучше).

Структура по ДГ:\\
Структурой теории первого порядка мы назовем упорядоченную тройку
$\langle D, F, P\rangle$, где $F$ — списки оценок для 0-местных, 1-местных и т.д.
функций, и $P$ = $P_{0}, P_{1} \ldots$ — списки оценок для 0-местных,
1-местных и т.д. предикатов, $D$ — предметное множество.

Понятие структуры — развитие понятия оценки из исчисления предикатов.
Но оно касается только нелогических составляющих теории; истинностные
значения и оценки для связок по-прежнему определяются исчислением
предикатов, лежащим в основе теории. Для получения оценки формулы
нам нужно задать структуру, значения всех свободных индивидных
переменных, и (естественным образом) вычислить результат.

Структура по-моему:\\
Все то же самое определение из ИВ. Мы просто забиваем на предикаты
в ИВ (не определям их), расширяем нашу сигнатуру (добавляя конкретные
предикаты и функциональные символы), определяем для нее интерпретацию.

И как всегда,..\\
Модель – это корректная структура (любое доказуемое утверждение должно
быть в ней общезначимо).
\subsection*{Аксиоматика Пеано}
\label{sec-2-19}
Множество $N$ удовлетворяет аксиоматике Пеано, если:
\begin{enumerate}
\item $0 \in N$
\item $x \in N, succ(x) \in N$
\item $\nexists x \in N : (succ(x) = 0)$
\item $(succ(a) = c \land succ(b) = c) \to a = b$
\item $P(0) \land \forall n.(P(n) \to P(succ(n))) \to \forall n.P(n)$
\end{enumerate}
\subsection*{Формальная арифметика -- аксиомы}
\label{sec-2-20}
Формальная арифметика -- это теория первого порядка, у которой
сигнатура определена как: (циферки, логические связки, алгебр.
связки, '), а интерпретацию сейчас будем определять.
Интерпретация определяет два множества -- $V, P$ -- истинностные и
предметные значения. Пусть множество $V = \lbrace 0, 1 \rbrace$ по-прежнему.
P = \{всякие штуки, которые мы можем получать из логических связок и 0\}

Определим оценки логических связок естественным образом.

Определим алгебраические связки так:
\begin{align*}
    +(a, 0) &= a \\
    +(a, b')& = (a + b)' \\
    *(a, 0) &= 0 \\
    *(a, b')& = a * b + a \\
\end{align*}
\subsubsection*{Аксиомы}
\label{sec-2-20-1}
\begin{enumerate}
\item $a = b \to a' = b'$
\item $a = b \to a = c \to b = c$
\item $a' = b' \to a = b$
\item $\lnot (a' = 0)$
\item $a + b' = (a + b)'$
\item $a + 0 = a$
\item $a * 0 = 0$
\item $a * b' = a * b + a$
\item $\phi[x:=0] \land \forall x.(\phi \to \phi[x:=x']) \to \phi$ // $\phi$ содержит св.п x
\end{enumerate}
\subsection*{Рекурсивные функции}
\label{sec-2-21}
$Z(x) = 0$\\
$N(x) = x + 1$\\
$U^n_i(x_1,\dotsc, x_n) = x_i$\\
$S\ltemplate f, g_1, \dotsc, g_n\rtemplate (x_1,\dotsc,x_m) = f(g_1(x_1\ldots{}x_m),\ldots{}g_n(x_1,\dotsc,x_m))$\\
$R\ltemplate f, g\rtemplate(x_1\ldots{}x_n, n) = \begin{cases}
    f(x_1\ldots{}x_n) & n = 0 \\
    g(x_1\ldots{}x_n, n, R\ltemplate f, g\rtemplate(x_1\ldots{}x_n, n - 1)) & n > 0
\end{cases}$\\
$\mu\ltemplate f\rtemplate(x_1, \dotsc, x_n)$ -- минимальное k, такое что $f(x_1\ldots{}x_n, k) = 0$
\subsection*{Функция Аккермана}
\label{sec-2-22}
\begin{align*}
    A(0, n) &= n + 1 \\
    A(m, 0) &= A(m - 1, 1) \\
    A(m, n) &= A(m - 1, A(m, n - 1))
\end{align*}
\subsection*{Существование рек.ф-й не явл. ф-ей Аккермана (определение конечной леммы)}
\label{sec-2-23}
Пусть $f(n_1,\dotsc,n_k)$ -- примитивная рекурсинвная функция, $k \ge 0$.
\[\exists J:f(n_1\ldots{}n_k)<A(J, \sum(n_1,\ldots{}n_k))\]
Доказывается индукцией по рекурсивным функциям.
\subsection*{Представимость}
\label{sec-2-24}
Функция $f:N^n\to N$ называется представимой в формальной арифметике, если
существует отношение $a(x_1\ldots{}x_{n+1})$, ее представляющее, причем выполнено
следующее:
\begin{enumerate}
\item $f(a,b,\ldots{}) = x \Leftrightarrow \vdash a(\overline a, \overline b,\dotsc, \overline x$)
\item $\exists !x.f(a,b,\ldots{}x)$ (вот это свойство вроде бы не обязательно, но ДГ его писал).
\end{enumerate}
\subsection*{Выразимость}
\label{sec-2-25}
Отношение n называется выразимым, если существует предикат N его
выражающий, такой что
\begin{enumerate}
    \item $n(x_1, \dotsc, x_n) \text{истинно} \Rightarrow \vdash N(\overline{x_1}, \dotsc, \overline{x_n}$)
    \item $n(x_1, \dotsc, x_n) \text{ложно} \Rightarrow \vdash \lnot N(\overline{x_1}, \dotsc, \overline{x_n}$)
\end{enumerate}
\subsection*{Лемма о связи представимости и выразимости}
\label{sec-2-26}
Если $n$ выразимо, то $C_n$ представимо.
$C_n = 1$ если $n$, и нулю если $!n$
\subsection*{Бета-функция Гёделя, Г-последовательность}
\label{sec-2-27}
\[\beta(b, c, i) = k_i\]
Функция, отображающая конечную последовательность из $N (a_i)$ в $k_i$.
Работает через магию, математику, простые числа и Гёделеву
последовательность, которая подходит под условия китайской
теоремы об остатках.
\[\beta(b, c, i) = b \% ((i + 1) * c + 1)\]
\subsection*{Представимость рек.ф-й в ФА (знать формулы для самых простых)}
\label{sec-2-28}
Рекурсивные функции представимы в ФА
\begin{enumerate}
\item $z(a, b) = (a = a) \land (b = 0)$
\item $n(a, b) = (a = b')$
\item $u^n_i = (x_1 = x_1) \land \dotsb \land (x_n = x_n) \land (x_{n+1} = x_i)$
\item $s(a_1\ldots{}a_m, b) = \exists b_1\ldots{}\exists b_n(G_1(a_1\ldots{}a_n, b_1) \land \ldots{} \land Gn(a_1\ldots{}a_m, b_n)$
\item $\begin{aligned}[t]
        r(x_1,\dotsc,x_n, k, a&) =\\
    \exists b\exists c(&\exists k(\beta(b, c, 0, k) \land \phi(x_1, \dotsc, x_n, k)) \land\\
        &B(b, c, x_{n+1}, a)\land\\
        &\forall k(k<x_{n+1} \to \exists d\exists e(B(b,c,k,d) \land B(b,c,k',e) \land G(x_1, \dotsc, x_n,k,d,e))))\\
    \end{aligned}$
\item $m\ltemplate F\rtemplate(x_1,\dotsc, x_{n+1}) = F(x_1, \dotsc, x_n, x_{n+1}, 0) \land \forall y((y < x_{n+1}) \to \lnot F(x_1,\dotsc, x_n, y, 0))$
\end{enumerate}
\subsection*{Гёделева нумерация (точно)}
\label{sec-2-29}
\begin{center}
\begin{tabular}{lrl}
$a$ & $\Godel{a}$ & описание\\
\hline
$($        & $3$ & \\
$)$        & $5$ & \\
$,$        & $7$ & \\
$\lnot$    & $9$ & \\
$\to$      & $11$ & \\
$\lor$     & $13$ & \\
$\land$    & $15$ & \\
$\forall$  & $17$ & \\
$\exists$  & $19$ & \\
$x_k$      & $21 + 6 \cdot k$ & переменные\\
$f^n_k$    & $23 + 6 \cdot 2^k \cdot 3^n$ & n-местные функцион. символы (', +, *)\\
$P^n_k$    & $25 + 6 \cdot 2^k \cdot 3^n$ & n-местные предикаты (=)\\
\hline
\end{tabular}
\end{center}

\subsection*{Выводимость и рекурсивные функции (че там с Тьюрингом)}
\label{sec-2-30}
Основные тезисы по вопросу:
\begin{itemize}
    \item $\operatorname{Emulate}(\mathrm{input}, \mathrm{prog}) =
        \plog(R\template{f,g}(\template{`S, input, 0},  , \mathrm{pb}, \mathrm{pc},
            \mathrm{tb}, \mathrm{tc}, \operatorname{steps}(-//-)), 1) == F$
        \item $\begin{multlined}[t]
                \operatorname{Proof}(\mathrm{term}, \mathrm{proof}) = \operatorname{Emulate}(\mathrm{proof}, \mathrm{MY\_PROOFCHECKER})\\
                \&\& (\plog(\mathrm{proof}, \mathrm{len}(\mathrm{proof})) = \mathrm{term})
            \end{multlined}$
    \item Любая представимая в ФА ф-я является рекурсивной
        \begin{multline*}
            f(x_1, \dotsc, x_n) = \\
            \plog(μ\template{S\template{G_\phi, U_{n+1, 1}, \dotsc, U_{n+1, n},
            \plog(U_{n+1, n+1}, 1),
            \plog(U_{n+1, n+1}, 2)}}(x_1,\dotsc, x_n), 1)
        \end{multline*}

    $G_\phi$ тут принимает $n + 2$ аргумента: $x_1\ldots{}x_n, p, b$ и возвращает 0 если
    p -- доказательство $\phi(x_1\ldots{}xₙ, p)$, представляющего f.
\end{itemize}
\subsection*{Непротиворечивость}
\label{sec-2-31}
Теория непротиворечива, если в ней нельзя одновременно
вывести $a$ и $\lnot a$.
Одновременная выводимость $\lnot a$ и $a$ эквивалентна выводимости
$a \land \lnot a$
\subsection*{\texorpdfstring{$\omega$}{w}-непротиворечивость}
\label{sec-2-32}
Теория $\omega$-непротиворечива, если из $\forall \phi(x) \vdash \phi(\overline x)$ следует
$\nvdash  \exists p\lnot \phi(p)$. Проще говоря, если мы взяли
формулу, то невозможно вывести одновременно $\exists x\lnot A(x)$
и $A(0), A(1), \dotsc$
\subsection*{Первая теорема Гёделя о неполноте}
\label{sec-2-33}
\begin{enumerate}
    \item Если формальная арифметика непротиворечива, то недоказуемо $\sigma(\Godel{\overline \sigma})$
    \item Если формальная арифметика $\omega$-непротиворечива, то недоказуемо $\lnot \sigma(\Godel{\overline \sigma})$
\end{enumerate}
\subsection*{Первая теорема Гёделя о неполноте в форме Россера}
\label{sec-2-34}
Если формальная арифметика непротиворечива, то в ней найдется
такая формула $\phi$, что $\nvdash \phi$ и $\nvdash \lnot \phi$
\subsection*{Consis}
\label{sec-2-35}
Consis -- утверждение, формально доказывающее непротиворечивость ФА

То есть $\vdash Consis => ФА$ непротиворечива
\subsection*{Условия Гильберта-Бернайса}
\label{sec-2-36}
Пусть $\pi g(x, p)$ выражает $\Proof(x, p)$.
$\pi (x) = \exists t.\pi g(x, t)$ действительно показывает,
что выражение доказуемо, если
\begin{enumerate}
\item $\vdash a => \vdash \pi(\Godel{\overline a})$
\item $\vdash \pi \left(\Godel{\overline a}\right) \to \pi \left(\Godel{\overline{\pi (\Godel{\overline a})}}\right)$
\item $\vdash \pi \left(\Godel{\overline a}\right) \to
    \pi \left(\Godel{\overline{(a \to b)}}\right) \to \pi \left(\Godel{\overline{b}}\right)$
\end{enumerate}
\subsection*{Лемма о самоприменении}
\label{sec-2-37}
$a(x)$ -- формула, тогда $\exists b$ такой что
\begin{enumerate}
    \item $\vdash a\left(\Godel{\overline b}\right) \to b$
    \item $\vdash \beta \to a\left(\Godel{\overline b}\right)$
\end{enumerate}
\subsection*{Вторая теорема Гёделя о неполноте ФА}
\label{sec-2-38}
Если теория непротиворечива, в ней $\nvdash Consis$
\subsection*{Теория множеств}
\label{sec-2-39}
Теория множеств -- теория первого порядка, в которой
есть единственный предикат $\in$ (в ФА был =), есть связка
$\leftrightarrow$, есть пустое множество, операции пересечения и
объединения.
$x \cap y = z$, тогда $\forall t(t \in z \leftrightarrow t \in x \land t \in y)$
$x \cup y = z$, тогда $\forall t(t \in z \leftrightarrow t \in x \lor t \in y)$
$D_j(x) \forall a \forall b(a \in x \land b \in x \land a \ne  b \to a \cap b = \emptyset)$
\subsection*{ZFC}
\label{sec-2-40}
\subsubsection*{Аксиома равенства}
\label{sec-2-40-1}
$\forall x \forall y \forall z((x = y \land y \in z) \to x \in z)$
Eсли два множества равны, то любой элемент лежащий в первом,
лежит и во втором
\subsubsection*{Аксиома пары}
\label{sec-2-40-2}
$\forall x \forall y (\lnot (x=y) \to \exists p(x \in p \land y \in p \land \forall z(z \in p \to (x = z \lor y = z))))$
$x \ne  y$, тогда сущ. $\lbrace x, y \rbrace$
\subsubsection*{Аксиома объединений}
\label{sec-2-40-3}
$\forall x(\exists y(y\in x) \to \exists p \forall y(y \in p \leftrightarrow \exists s(y \in s \land s \in x)))$
Eсли x не пусто, то из любого семейства множеств можно
образовать <<кучу-малу>>, то есть такое множество p,
каждый элемент y которого принадлежит по меньшей мере
одному множеству s данного семейства s x
\subsubsection*{Аксиома степени}
\label{sec-2-40-4}
$\forall x \exists p \forall y(y \in p \leftrightarrow y \in x)$
P(x) -- множество степени x (не путать с 2ˣ -- булеаном)
Это типа мы взяли наш x, и из его элементов объединением и
пересечением например понаобразовывали кучу множеств, а потом
положили их в p.
\subsubsection*{Схема аксиом выделения}
\label{sec-2-40-5}
$\forall x \exists b\forall y(y \in b \leftrightarrow (y \in x \land \phi(y)))$
Для нашего множества x мы можем подобрать множество побольше,
на котором для всех элементов, являющихся подмножеством x
выполняется предикат.
\subsubsection*{Аксиома выбора (не входит в ZF по дефолту)}
\label{sec-2-40-6}
Если $a = Dj(x)$ и $a \ne  0$, то $x \in a \ne  0$
\subsubsection*{Аксиома бесконечности}
\label{sec-2-40-7}
$\exists N(\emptyset \in N \land \forall x(x \in N \to x \cup \{x\} \in N))$
\subsubsection*{Аксиома фундирования}
\label{sec-2-40-8}
$\forall x(x = \emptyset \lor \exists y(y \in x \land y \cap x = \emptyset))$
$\forall x(x \ne  \emptyset \to \exists y(y \in x \land y \cap x = \emptyset))$
Равноценные формулы.

Я бы сказал, что это звучит как-то типа
<<не существует бесконечно вложенных множеств>>
\subsubsection*{Схема аксиом подстановки}
\label{sec-2-40-9}
$\forall x \exists !y.\phi(x,y) \to \forall a\exists b\forall c(c \in b \leftrightarrow (\exists d.(d \in a \land \phi(d, c))))$
Пусть формула $\phi$ такова, что для при любом $x$ найдется единственный $y$
такой, чтобы она была истинна на $x$, $y$, тогда для любого $a$
найдется множество $b$, каждому элементу которого $c$ можно сопоставить
подмножество $a$ и наша функция будет верна на нем и на $c$
Типа для хороших функций мы можем найти множество с отображением из
его элементов в подмножество нашего по предикату.

\subsection*{Ординальные числа, операции}
\label{sec-2-41}
\begin{itemize}
\item Определение вполне упорядоченного множества (фундированное
с линейныи порядком).
\item Определение транзитивного множества
Множество X транзитивно, если
$\forall a \forall b((a \in b \land b \in x) \to a \in x)$
\item Ординал -- транзитивное вполне упорядоченное отношением $\in$ мн-во
\item Верхняя грань множества ординалов S
$C | \{C = min(X) \land C \in X \mid X = \{z \mid \forall (y\in S)(z \ge y)\}\}$
$C = Upb(S)$
$Upb(\{\emptyset\}) = \{\emptyset\}$
\item Successor ordinal (сакцессорный ординал?)
Это $b = a' = a \cup \{a\}$
\item Предельны ординал
Ординал, не являющийся ни 0 ни successor'ом.
\item Недостижимый ординал
$\epsilon$ -- такой ординал, что $\epsilon  = w^\epsilon $

$\epsilon_0$ = $\operatorname{Upb}(w, w^w, w^{w^w}, w^{w^{w^w}}, \dotsc)$ -- минимальный из $\epsilon$
\item Канторова форма -- форма вида ∑(a*w$^{\text{b}}$+c), где b -- ординал, последовательность
строго убывает по b. Есть слабая канторова форма, где вместо $a (a \in N)$
пишут $a$ раз $w^b$. В канторовой форме приятно заниматься сложениями и
прочим, потому что всякие upb -- слишком ниочем.

\begin{align*}
x + 0      &= x \\
x + c'     &= (x + c)' \\
x + \lim(a) &= \operatorname{Upb}\{x + c \mid c < a\} \\
x * 0      &= 0 \\
x * c'     &= x * c + x \\
x * \lim(a) &= \operatorname{Upb}\{x * c \mid c < a\} \\
x ^ 0      &= 1 \\
x ^ {c'}     &= (x ^ c) * x \\
x ^ {\lim(a)} &= \operatorname{Upb}\{x ^ c \mid c < a\}
\end{align*}
\end{itemize}
\subsection*{Кардинальные числа, операции}
\label{sec-2-42}
\begin{definition}
    Будем называть множества равномощными, если найдется биекция.
\end{definition}
\begin{definition}
    Будем называть A не превышающим по мощности B, если найдется
    инъекция $A \to B (|A| \le |B|)$
\end{definition}
\begin{definition}
    Будем называть $А$ меньше по мощности, чем $B$, если $|A| \le |B| \land |A| \ne  |B|$
\end{definition}
\begin{definition}
    Кардинальное число -- число, оценивающее мощность множества.
\end{definition}
\begin{definition}
    Кардинальное число $\aleph$ -- это ординальное число a, такое что
    $\forall x \leq a \vert x \vert \leq \vert a \vert$

    $\aleph_0 = w$ по определению; $\aleph_1 = {}$ минимальный кардинал, следующий за $\aleph_0$
\end{definition}
\begin{definition}
    Кардинальное число $\beth$ -- это ординальное число а, такое что
    $\beth_i = P(\beth_{i-1})$\\
    $\beth_0 = \aleph_0$\\
    $+: |A| + |B| = \max(|A|, |B|) \text{(если нет общих элементов)} = |A \cup B|$
\end{definition}
\subsection*{Диагональный метод, теорема Лёвенгейма-Скулема}
\label{sec-2-43}
Диагональный метод -- метод доказательства $\vert$2$^{\text{X}}\vert > \vert X \vert$
\subsection*{Парадокс Скулема}
\label{sec-2-44}
Мнимый парадокс, базирующийся на теореме Лёвенгейма-Скулема
и том факте, что в формальной арифметике существуют несчетные
множества. Заковырка в том, что <<существует счетное мн-во>> выражается
в ФА <<не существует биекции>>. И тогда прийти к противоречию
нельзя.
\subsection*{Теорема Генцена о непротиворечивости ФА}
\label{sec-2-45}
Ну типа мы можем обернуть ФА в теорию покруче, доказать что в ней
невозможно доказать $0=1$, а потом доказать, что если $S_\infty$ непротиворечива,
то и $S$ непротиворечива.

\setcounter{section}{0}
\renewcommand{\thesection}{\arabic{section}}
\section{Ticket 1: ИВ}
\label{sec-3}
\subsection{Определения (исчисление, высказывание, оценкa\ldots{})}
\label{sec-3-1}
Формальная система с алгеброй Яськовского $J_{0}$ в качестве модели, множество
истинностных значений $\lbrace 0, 1 \rbrace$. Формальная теория нулевого порядка, кванторов
нету, предикаты - это пропозициональные переменные.
\subsection{Общезначимость, доказуемость, выводимость}
\label{sec-3-2}
\begin{itemize}
\item Общезначимость формулы -- ее свойство в теории с моделью. Общезначимость
можно определить как угодно, в принципе. Например в ИВ общезначимость --
это что оценка формулы на любых значениях свободных переменных отображает
в 1. В модели крипке - существование формулы во всех мирах и т.д.
\item Доказуемость - свойство формулы в теории, значащее, что существует
доказательство для этой формулы. Доказательство для теории тоже определяется
по разному (последовательность утверждений, каждое из которых есть аксиома
или следует по правилу вывода из предыдущих в ИВ, дерево с выводами в $S\infty$)
\item Выводимость - в общем случае часто используется как аналог доказуемости,
в ИВ это доказуемость из всего, что и ранее + из посылок.
\end{itemize}
\subsection{Схемы аксиом и правило вывода}
\label{sec-3-3}
Аксиомы:
\begin{enumerate}
\item $\alpha \to \beta \to \alpha$
\item $(\alpha \to \beta) \to (\alpha \to \beta \to \gamma) \to (\alpha \to \gamma)$
\item $\alpha \to \beta \to \alpha \land \beta$
\item $\alpha \land \beta \to \alpha$
\item $\alpha \land \beta \to \beta$
\item $\alpha \to \alpha \lor \beta$
\item $\beta \to \alpha \lor \beta$
\item $(\alpha \to \beta) \to (\gamma \to \beta) \to (\alpha \lor \gamma \to \beta)$
\item $(\alpha \to \beta) \to (\alpha \to \lnot \beta) \to \lnot \alpha$
\item $\lnot \lnot \alpha \to \alpha$
\end{enumerate}

Правило вывода M.P.:
\[\infer{\beta}{\alpha & (\alpha \rightarrow \beta)}\]
\subsection{Теорема о дедукции}
\label{sec-3-4}
\begin{theorem}
	$\Gamma, \alpha \vdash \beta \Leftrightarrow \Gamma \vdash \alpha \to \beta$
\end{theorem}
\begin{proof}

$\Rightarrow$ 
Если нужно переместить последнее предположение вправо,
то рассматриваем случаи -- аксиома или предположение,
MP, это самое выражение.
\begin{enumerate}
\item $A$ \\
$A\to \alpha \to A$ \\
$\alpha \to A$
\item (там где-то сзади уже было $\alpha \to A$, $\alpha \to A \to B$) \\
$(\alpha \to A)\to (\alpha \to A \to B)\to (\alpha \to B)$ \\
$(\alpha \to A\to B)\to (a\to B)$ \\
$\alpha \to B$
\item $\alpha\to \alpha$ умеем доказывать
\end{enumerate}
$\Leftarrow$ Если нужно переместить влево, то перемещаем, добавляем \\
$A\to B$ (последнее) \\
$A$    (перемещенное) \\
$B$
\end{proof}

\subsection{Корректность исчисления высказываний относительно алгебры Яськовского}
\label{sec-3-5}
\begin{itemize}
\item Индукцией по доказательству -- если аксиома, то она
тавтология, все ок. Если модус поненс, то таблица
истинности для импликации и все ок
\end{itemize}

\section{Ticket 2: полнота ИВ}
\label{sec-4}
\subsection{Полнота исчисления высказываний относительно алгебры Яськовского}
\label{sec-4-1}
Кстати полноту можно доказывать маханием руками как для предикатов,
и я не могу утверждать, что при таком подходе ИВ не будет полно
относительно любой модели.
\subsubsection{Контрапозиция}
\label{sec-4-1-1}
\begin{lemma}
    $(\alpha \to \beta) \to (\lnot \beta \to \lnot \alpha)$
\end{lemma}
\begin{proof}
    Докажем, что $(\alpha \to \beta), \lnot \beta \vdash \lnot \alpha$:\\
    \begin{tabular}{lll}
    (1) & $\alpha \to \beta$& Допущение\\
    (2) & $(\alpha \to \beta) \to (\alpha \to \lnot \beta) \to \lnot \alpha$& Сх. акс. 9\\
    (3) & $(\alpha \to \lnot \beta) \to \lnot \alpha$& M.P. 1,2\\
    (4) & $\lnot \beta \to \alpha \to \lnot \beta$& Сх. акс. 1\\
    (5) & $\lnot \beta$& Допущение\\
    (6) & $\alpha \to \lnot \beta$& M.P. 5,4\\
    (7) & $\lnot \alpha$& M.P. 6,3\\
    \end{tabular}\\
    После применения теоремы о дедукции 2 раза получим как раз то, что нужно
\end{proof}
\subsubsection{Правило исключененного третьего}
\label{sec-4-1-2}
С помощью контрапозиции доказываем два утверждения:\\
$\lnot (A|\lnot A)\to \lnot A$ (один раз контрапозицию от этого обратную, там $A\to (A|\lnot A)$ акс) \\
$\lnot (A|\lnot A)\to \lnot \lnot A$
Потом девятую аксиому и снимаем двойное отрицание
\subsubsection{Всякие очевидные вещи типа если выводится из А и из Б то из А и Б тоже}
\label{sec-4-1-3}
\subsubsection{Правило со звездочкой (14 доказательств)}
\label{sec-4-1-4}
\begin{enumerate}
\item $\alpha, \beta \vdash \alpha \lor \beta$ \\
$\alpha$ \\
$\alpha \to \alpha \lor \beta$ \\
$\alpha \lor \beta$
\item $\alpha, \lnot \beta \vdash \alpha \lor \beta$ \\
$\alpha$ \\
$\alpha \to \alpha \lor \beta$ \\
$\alpha \lor \beta$
\item $\lnot \alpha, \beta \vdash \alpha \lor \beta$ \\
$\beta$ \\
$\beta \to \alpha \lor \beta$ \\
$\alpha \lor \beta$
\item $\lnot \alpha, \lnot \beta \vdash \lnot (\alpha \lor \beta)$ \\
$\lnot \alpha$ \\
$\lnot \beta$ \\
$(\alpha \lor \beta \to \alpha) \to (\alpha \lor \beta \to \lnot \alpha) \to \lnot (\alpha \lor \beta)$ \\
$\lnot \alpha \to \alpha \lor \beta \to \lnot \alpha$ \\
$\alpha \lor \beta \to \lnot \alpha$ \\
$\lnot \alpha, \lnot \beta, \alpha \lor \beta \vdash \alpha$ \\
$\lnot \alpha$ \\
$\lnot \beta$ \\
$\alpha \lor \beta$ \\
$\alpha \to \alpha$ \\
\ldots{} //д-во $\lnot \beta, \lnot \alpha \vdash \beta \to \alpha$ \\
$\beta \to \alpha$ \\
$(\alpha \to \alpha) \to ((\beta \to \alpha) \to (\alpha \lor \beta \to \alpha))$ \\
$(\beta \to \alpha) \to (\alpha \lor \beta \to \alpha)$ \\
$\alpha \lor \beta \to \alpha$ \\
$\alpha$ \\
$\alpha \lor \beta \to \alpha$ \\
$(\alpha \lor \beta \to \lnot \alpha) \to \lnot (\alpha \lor \beta)$ \\
$\lnot (\alpha \lor \beta)$
\item $\alpha, \beta \vdash \alpha \land \beta$ \\
$\alpha$ \\
$\beta$ \\
$\alpha \to \beta \to \alpha \land \beta$ \\
$\beta \to \alpha \land \beta$ \\
$\alpha \land \beta$
\item $\alpha, \lnot \beta \vdash \lnot (\alpha \land \beta)$ \\
$\lnot \beta$ \\
$((\alpha \land \beta) \to \beta) \to ((\alpha \land \beta) \to \lnot \beta) \to \lnot (\alpha \land \beta)$ \\
$\alpha \land \beta \to \beta$ \\
$(\alpha \land \beta \to \lnot \beta) \to \lnot (\alpha \land \beta)$ \\
$\lnot \beta \to \alpha \land \beta \to \lnot \beta$ \\
$\alpha \land \beta \to \lnot \beta$ \\
$\lnot (\alpha \land \beta)$
\item $\lnot \alpha, \beta \vdash \lnot (\alpha \land \beta)$ \\
аналогично
\item $\lnot \alpha, \lnot \beta \vdash \lnot (\alpha \land \beta)$ \\
аналогично
\item $\alpha, \beta \vdash \alpha \to \beta$ \\
$\beta$ \\
$\beta \to \alpha \to \beta$ \\
$\alpha \to \beta$
\item $\alpha, \lnot \beta \vdash \lnot (\alpha \to \beta)$ \\
$\alpha$ \\
$\lnot \beta$ \\
$\lnot \beta \to ((\alpha \to \beta) \to \lnot \beta)$ \\
$(\alpha \to \beta) \to \lnot \beta$ \\
$\alpha, \lnot \beta, \alpha \to \beta \vdash \beta$ \\
$\alpha$ \\
$\alpha \to \beta$ \\
$\beta$ \\
$(\alpha \to \beta) \to \beta$ \\
$((\alpha \to \beta) \to \beta) \to ((\alpha \to \beta) \to \lnot \beta) \to \lnot (\alpha \to \beta)$ \\
$((\alpha \to \beta) \to \lnot \beta) \to \lnot (\alpha \to \beta)$ \\
$\lnot  \beta \to (\alpha \to \beta) \to \lnot \beta$ \\
$(\alpha \to \beta) \to \lnot \beta$ \\
$\lnot (\alpha \to \beta)$
\item $\lnot \alpha, \beta \vdash \alpha \to \beta$ \\
$\beta$ \\
$\beta \to \alpha \to \beta$ \\
$\alpha \to \beta$
\item $\lnot \alpha, \lnot \beta \vdash \alpha \to \beta$\\
Ну тут типо очевидно (на самом деле тут боль и страдания)
\item $\alpha \vdash \lnot \lnot \alpha$\\
Схема аксиом 9
\item $\lnot \alpha \vdash \lnot \alpha$\\
$\lnot \alpha$
\end{enumerate}

\section{Ticket 3: ИИВ}
\label{sec-5}
\subsection{ИИВ, структура, модель}
\label{sec-5-1}

Сигнатура -- $(R, F, C, r)$: $R$ -- множество символов для предикатов, $F$ -- функциональных символов, $C$ -- символов констант, $r$ -- функция, определяющая арность $x \in R \cup F$. 

Интерпретация -- это приписывание символам значения и правил действия.

Структура -- это носитель $M$ (множество истинностных значений), сигнатура и интерпретация над носителем.

Если все аксиомы верны, то структура корректна.
В таком случае она называется моделью.

Выкидываем 10 аксиому, добавляем $\alpha \rightarrow \neg \alpha \rightarrow \beta$.

Она доказывается и в ИВ.
\begin{lemma}\label{provability}
$\alpha, \alpha \vee \neg \alpha, \neg \alpha \vdash \beta$
\end{lemma}
\begin{tabular}{lll}
(1) & $\alpha$& Допущение\\
(2) & $\neg \alpha$& Допущение\\
(3) & $\alpha \rightarrow \neg \beta \rightarrow \alpha$& Сх. акс. 1\\
(4) & $\neg \beta \rightarrow \alpha$& M.P. 1,3\\
(5) & $\neg \alpha \rightarrow \neg \beta \rightarrow \neg \alpha$& Сх. акс. 1\\
(6) & $\neg \beta \rightarrow \neg \alpha$& M.P. 2,5\\
(7) & $(\neg \beta \rightarrow \alpha) \rightarrow (\neg \beta \rightarrow \neg \alpha) \rightarrow (\neg \neg \beta)$& Сх. акс. 9\\
(8) & $(\neg \beta \rightarrow \neg \alpha) \rightarrow (\neg \neg \beta)$& M.P. 4,7\\
(9) & $\neg \neg \beta$& M.P. 6,8\\
(10) & $\neg \neg \beta \rightarrow \beta$& Сх. акс. 10\\
(11) & $\beta$& M.P. 9,10\\
\end{tabular}


Таким образом мы умеем доказывать $\alpha \rightarrow \alpha \vee \neg \alpha \rightarrow \neg \alpha \rightarrow \beta$ применив 3 раза теорему о дедукции
\begin{lemma}
$\alpha \rightarrow \alpha \vee \neg \alpha \rightarrow \neg \alpha \rightarrow \beta, \alpha \vee \neg \alpha \vdash \alpha \rightarrow \neg \alpha \rightarrow \beta$
\end{lemma}
\begin{tabular}{lll}
(1) & $(\alpha \rightarrow \alpha \vee \neg \alpha) \rightarrow (\alpha \rightarrow \alpha \vee \neg \alpha \rightarrow (\neg \alpha \rightarrow \beta)) \rightarrow (\alpha \rightarrow (\neg \alpha \rightarrow \beta))$& Сх. акс. 2\\
(2) & $\alpha \vee \neg \alpha \rightarrow \alpha \rightarrow \alpha \vee \neg \alpha$& Сх. акс. 1\\
(3) & $\alpha \vee \neg \alpha$& Допущение\\
(4) & $\alpha \rightarrow \alpha \vee \neg \alpha$& M.P. 3,2\\
(5) & $(\alpha \rightarrow \alpha \vee \neg \alpha \rightarrow (\neg \alpha \rightarrow \beta)) \rightarrow (\alpha \rightarrow (\neg \alpha \rightarrow \beta))$& M.P. 4,1\\
(6) & $\alpha \rightarrow \alpha \vee \neg \alpha \rightarrow \neg \alpha \rightarrow \beta$& Допущение\\
(7) & $\alpha \rightarrow \neg \alpha \rightarrow \beta$& M.P. 6,5\\
\end{tabular}

\subsection{Опровергаемость исключенного третьего}
\label{sec-5-2}
Вводим в наше множество \emph{истинностных значений} дополнительный элемент \texttt{Н} (сокращение от слова <<Неизвестно>>). Отождествим \texttt{Н} с $\nicefrac{1}{2}$, так что $\texttt{Л} < \texttt{Н} < \texttt{И}$. Определим операции на этом множестве \emph{истинностных значений}:
\begin{itemize}
\item конъюнкция: минимум из двух значений (например $\texttt{И} \land \texttt{Н} = \texttt{Н}$).
\item дизъюнкция: максимум из двух значений (например $\texttt{И} \vee \texttt{Н} = \texttt{И}$).
\item импликация: $\texttt{И} \rightarrow \alpha = \alpha$, $\texttt{Л} \rightarrow \alpha = \texttt{И}$, $\texttt{Н} \rightarrow \texttt{Л} = \texttt{Л}$, $\texttt{Н} \rightarrow \texttt{Н} = \texttt{И}$, $\texttt{Н} \rightarrow \texttt{И} = \texttt{И}$.
\item отрицание: $\neg \texttt{Н} = \texttt{Л}$, а для остальных элементов все так же.
\end{itemize}

Назовем формулу \emph{3-тавтологией}, если она принимает значение
\texttt{И} при любых значениях переменных из множества $\{\texttt{И}, \texttt{Л}\, \texttt{Н}\}$. Теперь нужно всего-лишь проверить, что все аксиомы являются 3-тавтологиями и, что если посылка импликации является тавтологией, то и заключение является тавтологией. Второе очевидно по определению тавтологии, а аксиомы просто проверяются вручную.

Значит любая интуиционистски выводимая формула 3-тавтология. Теперь заметим, что формула $\alpha \vee \neg \alpha$ принимает значение \texttt{Н} при $\alpha = \texttt{Н}$. Следовательно она не 3-тавтология, а значит невыводима.
\subsection{Решетки}
\label{sec-5-3}
Просто \emph{решетка} -- это $(L, +, *)$ в алгебраическом смысле и $(L, \leq)$ в порядковом. Решетку можно определить как алгебраическую структуру через аксиомы: 
\begin{itemize}
\item Аксиомы идемпотентность\\
$\alpha + \alpha = \alpha$\\
$\alpha * \alpha = \alpha$
\item Аксиомы коммутативности\\
$\alpha + \beta = \beta + \alpha$\\
$\alpha * \beta = \beta * \alpha$
\item Аксиомы ассоциативности\\
$(\alpha + \beta) + \gamma = \alpha + (\beta + \gamma)$\\
$(\alpha * \beta) * \gamma = \alpha * (\beta * \gamma)$
\item Аксиомы поглощения\\
$\alpha + (\alpha * \beta) = \alpha$\\
$\alpha * (\alpha + \beta) = \alpha$
\end{itemize} 
Также решетку можно определить как упорядоченное множество с частичным порядком на нем. Тогда операции $+, *$ определяются как $\sup$ и $\inf$
\begin{gather*}
\sup(\phi) = \min \lbrace u \mid u \geq \forall x \in \phi \rbrace\\
\inf(\phi) = \max \lbrace u \mid u \leq \forall x \in \phi \rbrace\\
\alpha + \beta = \sup (\lbrace \alpha, \beta \rbrace)\\
\alpha * \beta = \inf (\lbrace \alpha, \beta \rbrace)\\
\end{gather*}
Если для любых двух элементов из множества $S$ можно определить эти две операции, то $S$ называется решеткой.

\emph{Дистрибутивная решетка} -- \emph{решетка}, в которой добавляется дистрибутивность:
\[\alpha * (\beta + \gamma) = \alpha * \beta + \alpha * \gamma\]

\emph{Импликативная решетка} -- \emph{решетка}, в которой для любых двух элементов $\alpha$ и $\beta$ из множества существует псевдодополнение $\alpha$ относительно $\beta$ ($\alpha \rightarrow \beta$), которое определяется так:
\[\alpha \rightarrow \beta = max \lbrace \gamma \vert \gamma * \alpha \leq \beta \rbrace\]

Свойства \emph{импликативной решетки}:
\begin{itemize}
\item Существует максимальный элемент $\alpha \rightarrow \alpha$, обычно обозначаемый как $1$
\item Всякая \emph{импликативная решетка} дистрибутивна
\end{itemize}

\subsection{Алгебра Гейтинга, булева алгебра}
\label{sec-5-4}
\emph{Булева алгебра} -- $(L, +, *, -, 0, 1)$, с аксиомами:
\begin{itemize}
\item Аксиомы коммутативности\\
$\alpha + \beta = \beta + \alpha$\\
$\alpha * \beta = \beta * \alpha$
\item Аксиомы ассоциативности\\
$(\alpha + \beta) + \gamma = \alpha + (\beta + \gamma)$\\
$(\alpha * \beta) * \gamma = \alpha * (\beta * \gamma)$
\item Аксиомы поглощения\\
$\alpha + (\alpha * \beta) = \alpha$\\
$\alpha * (\alpha + \beta) = \alpha$
\item Аксиомы дистрибутивности\\
$\alpha + (\beta * \gamma) = (\alpha + \beta) * (\alpha + \gamma)$\\
$\alpha * (\beta + \gamma) = (\alpha * \beta) + (\alpha * \gamma)$
\item Аксиомы дополнительности\\
$\alpha * \neg \alpha = 0$\\
$\alpha + \neg \alpha = 1$
\end{itemize}

Также \emph{булеву алгебру} можно определить как импликативную решетку над фундированным множеством. Тогда $1$ в ней будет $\alpha \rightarrow \alpha$, $\neg \alpha = \alpha \rightarrow 0$. Тогда $\alpha * \neg \alpha = 0$ будет уже свойством, а $\alpha + \neg \alpha = 1$ все еще аксиомой.

\emph{Псевдобулева алгебра} (алгебра Гейтинга) -- это импликативная решетка над фундированным множеством с $\neg \alpha = \alpha \rightarrow 0$ (нет аксиомы $\alpha + \neg \alpha = 1$)
\subsection{Алгебра Линденбаума-Тарского}
\label{sec-5-5}
Пусть $V$ -- множество формул ИИВ\\
Порядок для решетки:\\
$\alpha \leq \beta \Leftrightarrow  \alpha \vdash \beta$\\
$\alpha \sim \beta \Leftrightarrow \alpha \vdash \beta$ и $\beta \vdash \alpha$\\
Определим операции и $0$, $1$:\\
$0$ -- $\alpha \land \neg \alpha = \perp$\\
$1$ -- $\alpha \rightarrow \alpha = T$\\
$\alpha \land \beta = \alpha * \beta$\\
$\alpha \vee \beta = \alpha + \beta$\\
$\neg \alpha = -\alpha$\\
Получившаяся алгебра называется \emph{алгеброй Линденбаума-Тарского} и является алгеброй Гейтинга, т.к. для нее выполняется аксиома $\alpha * \neg \alpha = 0$ (по определению).
\begin{lemma}
$\forall \beta \in V \perp \vdash \beta$ (Из лжи следует все)
\end{lemma}
\begin{proof}
$\alpha \land \neg \alpha \vdash \beta$\\
\begin{tabular}{lll}
(1) &$\alpha \land \neg \alpha$& Допущение\\
(2) &$\alpha \land \neg \alpha \rightarrow \alpha$& Сх. акс. 4\\
(3) &$\alpha \land \neg \alpha \rightarrow \neg \alpha$& Сх. акс. 5\\
(4) &$\alpha$& M.P. 1,2\\
(5) &$\neg \alpha$& M.P. 1,3\\
(6) &$\alpha \rightarrow \neg \alpha \rightarrow \beta$& Сх. акс. 10\\
(7) & $\neg \alpha \rightarrow \beta$& M.P. 4,6\\
(8) & $\beta$& M.P. 5,7\\
\end{tabular}
\end{proof}
\subsection{Теорема о полноте ИИВ относительно алгебры Гейтингa}
\label{sec-5-6}
Возьмем в качестве алгебры Гейтинга алгебру Линденбаума-Тарского - $\xi$. Она очевидно является моделью. 
\begin{theorem}
$\vDash \alpha \Rightarrow \vdash \alpha$
\end{theorem}
\begin{proof}
$\vDash \alpha \Rightarrow \llbracket \alpha \rrbracket ^ {\xi} = 1$\\
$\llbracket \alpha \rrbracket ^ {\xi} = 1 \Rightarrow 1 \leq \llbracket \alpha \rrbracket ^ {\xi}$ (По определению алгебры Л-Т)\\
$\beta \rightarrow \beta \vdash \alpha$ (По определению $\leq$ в алгебре Л-Т)\\
Т.к. $\beta \rightarrow \beta$ - тавтология, то и $\alpha$ - тавтология
\end{proof}
\subsection{Дизъюнктивность ИИВ}
\label{sec-5-7}
Используем алгебру Гёделя $\Gamma(A)$ ($\gamma$ - функция преобразования). Можно преобразовать любую алгебру Гейтинга, возьмем алгебру Л-Т. Алгебра Гёделя использует функцию преобразования: $\gamma(a)=b$ значит, что в алгебре $A$ элементу $a$ соответствует элемент $b$ из алгебры Гёделя. Порядок сохраняется естественным образом. Также добавим еще один элемент $\omega$ ($\gamma(1)=\omega$). Таким образом $\Gamma(A) = A \cup \lbrace \omega \rbrace$. Порядок в $\Gamma(A)$:
\begin{itemize}
\item $\forall a \in \Gamma(A) \setminus \lbrace 1 \rbrace$ $a \leq \omega$
\item $\omega \leq 1$
\end{itemize}
\begin{tabular}{lllll}
\cline{1-3}
\multicolumn{1}{|l|}{$a+b$}         & \multicolumn{1}{l|}{$b=1$} & \multicolumn{1}{l|}{$b=\gamma(v)$} &  &  \\ \cline{1-3}
\multicolumn{1}{|l|}{$a=1$}         & \multicolumn{1}{l|}{$1$}   & \multicolumn{1}{l|}{$1$}           &  &  \\ \cline{1-3}
\multicolumn{1}{|l|}{$a=\gamma(u)$} & \multicolumn{1}{l|}{$1$}   & \multicolumn{1}{l|}{$\gamma(u+v)$} &  &  \\ \cline{1-3}
                                    &                            &                                    &  & 
\end{tabular}
\begin{tabular}{lllll}
\cline{1-3}
\multicolumn{1}{|l|}{$a*b$}         & \multicolumn{1}{l|}{$b=1$}         & \multicolumn{1}{l|}{$b=\gamma(v)$} &  &  \\ \cline{1-3}
\multicolumn{1}{|l|}{$a=1$}         & \multicolumn{1}{l|}{$1$}           & \multicolumn{1}{l|}{$\gamma(a*v)$} &  &  \\ \cline{1-3}
\multicolumn{1}{|l|}{$a=\gamma(u)$} & \multicolumn{1}{l|}{$\gamma(u*b)$} & \multicolumn{1}{l|}{$\gamma(u*v)$} &  &  \\ \cline{1-3}
                                    &                                    &                                    &  & 
\end{tabular}\\
\begin{tabular}{lllll}
\cline{1-3}
\multicolumn{1}{|l|}{$a \rightarrow b$} & \multicolumn{1}{l|}{$b=1$} & \multicolumn{1}{l|}{$b=\gamma(v)$}             &  &  \\ \cline{1-3}
\multicolumn{1}{|l|}{$a=1$}             & \multicolumn{1}{l|}{$1$}   & \multicolumn{1}{l|}{$\gamma(a \rightarrow v)$} &  &  \\ \cline{1-3}
\multicolumn{1}{|l|}{$a=\gamma(u)$}     & \multicolumn{1}{l|}{$1$}   & \multicolumn{1}{l|}{$u \rightarrow v$}         &  &  \\ \cline{1-3}
                                        &                            &                                                &  & 
\end{tabular}
\begin{tabular}{lllll}
\cline{1-2}
\multicolumn{1}{|l|}{$a$} & \multicolumn{1}{l|}{$\neg a$} &  &  &  \\ \cline{1-2}
\multicolumn{1}{|l|}{$a=1$} & \multicolumn{1}{l|}{$\gamma(\neg a)$} &  &  &  \\ \cline{1-2}
\multicolumn{1}{|l|}{$a=\gamma(u)$} & \multicolumn{1}{l|}{$\neg u$} &  &  &  \\ \cline{1-2}
 &  &  &  & 
\end{tabular}
\begin{lemma}
Гёделева алгебра является Гейтинговой
\end{lemma}
\begin{proof}
Необходимо просто доказать аксиомы коммутативности, ассоциативности и поглощения.
\end{proof}

\begin{theorem}
$\vdash \alpha \vee \beta \Rightarrow$ либо $\vdash \alpha$, либо $\vdash \beta$
\end{theorem}
\begin{proof}
Возьмем $A$, построим $\Gamma(A)$. Если $\vdash \alpha \vee \beta$, то $\llbracket \alpha \vee \beta \rrbracket ^ {A} = 1$ и $\llbracket \alpha \vee \beta \rrbracket ^ {\Gamma(A)} = 1$.\\
Тогда по определению $+$ в алгебре Гёделя, $\llbracket \alpha \rrbracket ^ {\Gamma(A)} = 1$, либо $\llbracket \beta \rrbracket ^ {\Gamma(A)} = 1$. Тогда оно такое же и в алгебре Л-Т, а алгебра Л-Т полна.
\end{proof}
\subsection{Теорема Гливенко}
\label{sec-5-8}
\begin{theorem}
Если в ИВ доказуемо $\alpha$, то в ИИВ доказуемо $\neg\neg\alpha$.
\end{theorem}
\begin{proof}
Разберем все втречающиеся в изначальном доказательстве формулы
\begin{enumerate}
\item Заметим, что если в ИИВ доказуемо $\alpha$, то $\neg\neg\alpha$ так же доказуемо.

Докажем, что $\alpha \vdash \neg \neg \alpha$

\begin{tabular}{lll}
(1) &$\alpha$& Допущение\\
(2) &$\alpha \rightarrow \neg \alpha \rightarrow \alpha$& Сх. акс. 1\\
(3) &$\neg \alpha \rightarrow \alpha$& M.P. 1,2\\
(4) & $\neg \alpha \rightarrow (\neg \alpha \rightarrow \neg \alpha)$&Сх. акс. 1\\
(5) & $(\neg \alpha \rightarrow (\neg \alpha \rightarrow \neg \alpha)) \rightarrow 
  (\neg \alpha \rightarrow ((\neg \alpha \rightarrow \neg \alpha) \rightarrow \neg \alpha)) \rightarrow
  (\neg \alpha \rightarrow \neg \alpha)$&Сх. акс. 2\\
(6) & $(\neg \alpha \rightarrow ((\neg \alpha \rightarrow \neg \alpha) \rightarrow \neg \alpha)) \rightarrow
  (\neg \alpha \rightarrow \neg \alpha)$&M.P. 4,5\\
(7) & $(\neg \alpha \rightarrow ((\neg \alpha \rightarrow \neg \alpha) \rightarrow \neg \alpha))$ & Сх. акс. 1\\
(8) & $\neg \alpha \rightarrow \neg \alpha$ & M.P. 7,6\\
(9) & $(\neg \alpha \rightarrow \alpha) \rightarrow (\neg \alpha \rightarrow \neg \alpha) \rightarrow \neg \neg \alpha$& Сх. акс. 9\\
(10) & $(\neg \alpha \rightarrow \neg \alpha) \rightarrow \neg \neg \alpha$& M.P. 3,9\\
(11) & $\neg \neg \alpha$& M.P. 8,10\\
\end{tabular}

Значит, если $\alpha$ -- аксиома с 1-ой по 9-ую, то $\neg \neg \alpha$ также может быть доказано
\item Пусть $\alpha$ получилось по 10-ой аксиоме $\neg \neg \alpha \rightarrow \alpha$. Докажем, что $\vdash \neg \neg (\neg \neg \alpha \rightarrow \alpha)$

\begin{tabular}{lll}
(1) &$\alpha \rightarrow \neg \neg \alpha \rightarrow \alpha$& Сх. акс. 1\\
(2) &$\neg (\neg \neg \alpha \rightarrow \alpha) \rightarrow \neg \alpha$& Контрпозиция\\
(3) &$\neg \alpha \rightarrow \neg \neg \alpha \rightarrow \alpha$& Сх. акс. 10\\
(4) &$\neg (\neg \neg \alpha \rightarrow \alpha) \rightarrow \neg \neg \alpha$& Контрпозиция\\
(5) &$(\neg ( \neg \neg \alpha \rightarrow \alpha) \rightarrow \neg \alpha) \rightarrow (\neg (\neg \neg \alpha \rightarrow \alpha) \rightarrow \neg \neg \alpha) \rightarrow \neg \neg (\neg \neg \alpha \rightarrow \alpha)$& Сх. акс. 9\\
(6) &$(\neg (\neg \neg \alpha \rightarrow \alpha) \rightarrow \neg \neg \alpha) \rightarrow \neg \neg (\neg \neg \alpha \rightarrow \alpha)$& M.P. 2,5\\
(7) &$\neg \neg (\neg \neg \alpha \rightarrow \alpha)$& M.P. 4,6\\
\end{tabular}
\item Приведем конструктивное доказательство:
\begin{itemize}
\item Если $\alpha$ - аксиома, то $\neg \neg \alpha$ доказывается с помощью 1-го и 2-го пунктов
\item Если был применен M.P., то в изначальном доказательстве были $\alpha$, $\alpha \rightarrow \beta$, $\beta$. По индукционному предположению мы знаем, что $\neg \neg \alpha$, $\neg \neg (\alpha \rightarrow \beta)$. Нужно доказать $\neg \neg \beta$.\\
Давайте для начала докажем, что \[\neg \neg \alpha, \neg \neg (\alpha \rightarrow \beta), \neg \beta, \alpha, \alpha \rightarrow \beta \vdash \beta\].\\
\begin{tabular}{lll}
(1) &$\alpha$& Допущение\\
(2) &$\alpha \rightarrow \beta$& Допущение\\
(3) &$\beta$& M.P. 1,2\\
\end{tabular}\\
Значит мы знаем, что $\neg \neg \alpha, \neg \neg (\alpha \rightarrow \beta), \neg \beta, \alpha \vdash (\alpha \rightarrow \beta) \rightarrow \beta$. Теперь докажем, что \[\neg \neg \alpha, \neg \neg (\alpha \rightarrow \beta), \neg \beta, \alpha, (\alpha \rightarrow \beta) \rightarrow \beta \vdash \neg (\alpha \rightarrow \beta.)\]\\
\begin{tabular}{lll}
(1) &$((\alpha \rightarrow \beta) \rightarrow \beta) \rightarrow ((\alpha \rightarrow \beta) \rightarrow \neg \beta) \rightarrow \neg(\alpha \rightarrow \beta)$& Сх. акс. 9\\
(2) &$((\alpha \rightarrow \beta) \rightarrow \beta)$& Допущение\\
(3) &$\neg \beta \rightarrow (\alpha \rightarrow \beta) \rightarrow \neg \beta$& Сх. акс. 1\\
(4) &$\neg \beta$& Допущение\\
(5) &$(\alpha \rightarrow \beta) \rightarrow \neg \beta$& M.P. 4,3\\
(6) &$((\alpha \rightarrow \beta) \rightarrow \neg \beta) \rightarrow \neg(\alpha \rightarrow \beta)$ M.P. 2,1\\
(7) &$\neg(\alpha \rightarrow \beta)$ M.P. 5,6\\
\end{tabular}\\
Теперь мы знаем, что $\neg \neg \alpha, \neg \neg (\alpha \rightarrow \beta), \neg \beta \vdash \alpha \rightarrow \neg (\alpha \rightarrow \beta)$. Докажем, что \[ \neg \neg \alpha, \neg \neg (\alpha \rightarrow \beta), \neg \beta, \alpha \rightarrow \neg (\alpha \rightarrow \beta) \vdash \neg \alpha.\] \\
\begin{tabular}{lll}
(1) &$(\alpha \rightarrow \neg (\alpha \rightarrow \beta)) \rightarrow (\alpha \rightarrow \neg \neg (\alpha \rightarrow \beta)) \rightarrow \neg \alpha$& Сх. акс. 9\\
(2) &$\alpha \rightarrow \neg (\alpha \rightarrow \beta)$&Допущение\\
(3) &$\neg \neg (\alpha \rightarrow \beta) \rightarrow \alpha \rightarrow \neg \neg (\alpha \rightarrow \beta)$& Сх. акс. 1\\
(4) &$\neg \neg (\alpha \rightarrow \beta)$& Допущение\\
(5) &$\alpha \rightarrow \neg \neg (\alpha \rightarrow \beta)$& M.P. 4,3\\
(6) &$(\alpha \rightarrow \neg \neg (\alpha \rightarrow \beta)) \rightarrow \neg \alpha$& M.P. 2,1\\
(7) &$\neg \alpha$& M.P.5,6\\
\end{tabular}\\
Теперь мы знаем, что $\neg \neg \alpha, \neg \neg (\alpha \rightarrow \beta) \vdash \neg \beta \rightarrow \neg \alpha$. Наконец докажем, что \[ \neg \neg \alpha, \neg \neg (\alpha \rightarrow \beta), \neg \beta \rightarrow \neg \alpha \vdash \neg \neg \beta\] \\
\begin{tabular}{lll}
(1) &$(\neg \beta \rightarrow \neg \alpha) \rightarrow (\neg \beta \rightarrow \neg \neg \alpha) \rightarrow \neg \neg \beta$& Сх. акс. 9\\
(2) &$\neg \beta \rightarrow \neg \alpha$& Допущение\\
(3) &$\neg \neg \alpha \rightarrow \neg \beta \rightarrow \neg \neg \alpha$& Сх. акс. 1\\
(4) &$\neg \neg \alpha$& Допущение\\
(5) &$\neg \beta \rightarrow \neg \neg \alpha$& M.P. 4,3\\
(6) &$(\neg \beta \rightarrow \neg \neg \alpha) \rightarrow \neg \neg \beta$& M.P. 2,1\\
(7) &$\neg \neg \beta$& M.P. 5,6\\
\end{tabular}
\end{itemize}
\end{enumerate}
\end{proof}
\subsection{Топологическая интерпретация}
\label{sec-5-9}
Булеву алгебру и алгебру Гейтинга можно интерпретировать на множестве $\mathbb R^n$. Тогда заключения о общезначимости формулы можно делать более наглядно. Давайте возьмем в качестве множества алгебры все открытые подмножества $\mathbb R^n$. Определим операции следующим образом:
\begin{itemize}
\item $\alpha + \beta = \alpha \cup \beta$
\item $\alpha * \beta = \alpha \cap \beta$
\item $\alpha \rightarrow \beta = Int(\alpha^c\cup\beta)$
\item $-\alpha = Int(\alpha^c)$
\item $0 = \emptyset$
\item $1 = \cup \lbrace V \subset L \rbrace$
\end{itemize}

\section{Модели Крипке}
\label{sec-6}
\subsection{Модели Крипке}
\label{sec-6-1}
$W$ -- множество миров\\
$V$ -- множество вынужденных переменных\\
Введем отношение частичного порядка на $W$ - $\leq$ (отношение достижимости). И введем оценку переменной $v: W \times V \rightarrow \lbrace 0, 1 \rbrace$. $v$ должна быть монотонна (Если $v(x, P) = 1$ и $x \leq y$, то $v(y, P) = 1$). Если пременная $x$ истинна в мире $w$, то мы пишем $w \Vdash x$.\\
\emph{Модель Крипке} -- это $\langle W, \leq, v\rangle$.\\
Теперь можно определить истинность любой формулы (в данном мире) индукцией по построению формулы. Правила:
\begin{itemize}
\item $w \Vdash A \land B \Leftrightarrow w \Vdash A$ и $w \Vdash B$;
\item $w \Vdash A \vee B \Leftrightarrow w \Vdash A$ или $w \Vdash B$;
\item $w \Vdash A \rightarrow B \Leftrightarrow$ в любом мире $u \geq w$, в котором истинна $A$, так же истинна и $B$;
\item $w \Vdash \neg A \Leftrightarrow$ ни в каком мире $u \geq w$ формула $A$ не является истинной;
\end{itemize}
\subsection{Корректность ИИВ относительно моделей Крипке}
\label{sec-6-2}
\begin{theorem}
Если формула выводима в ИИВ, то она истинна в моделях Крипке.
\end{theorem}
\begin{proof}
Проверим M.P. и аксиомы (что они истинны во всех мирах):
\begin{itemize}
\item M.P.: по определению импликации в моделях Крипке, если в мире истинно $A$, $A \rightarrow B$, то истинно и $B$
\item Аксиомы:
\begin{enumerate}
\item $A \rightarrow (B \rightarrow A)$\\
Пусть где-нибудь истинна $A$, в силу монотонности она истинна во всех б\'ольших мирах, так что $B \rightarrow A$ тоже будет истинно.
\item $(A \rightarrow B) \rightarrow ((A \rightarrow (B \rightarrow C)) \rightarrow (A \rightarrow C))$\\
Пусть где-нибудь истинно $A \rightarrow B$, тогда необходимо доказать, что истинно и $((A \rightarrow (B \rightarrow C)) \rightarrow (A \rightarrow C))$.
\begin{itemize}
\item Пусть истинны $A, B$. Тогда если истинно $A \rightarrow (B \rightarrow C)$, то истинно и $C$ по монотонности $A$ и $B$. $A, B, C$ истинны, значит $A \rightarrow C$ истинно.
\item Пусть не истинны ни $A$, ни $B$. Тогда $A \rightarrow (B \rightarrow C)$ не истинно и $C$ не истинно. Значит $A \rightarrow C$ не может быть истинно, т.к. ни $A$, ни $B$, ни $C$ не истинны.\\
\myworries{Сомнение насчет этого места}
\end{itemize}
\item Подобным образом доказываем все аксиомы
\end{enumerate}
\end{itemize}
\end{proof}
\subsection{Вложение Крипке в Гейтинга}
\label{sec-6-3}
Не нужно (Д.Г. обещал не спрашивать это)
\subsection{Полнота ИИВ в моделях Крипке}
\label{sec-6-4}
\begin{theorem}
ИИВ полно относительно моделей Крипке
\end{theorem}
\begin{proof}
Докажем в несколько шагов
\begin{enumerate}
\item \emph{Дизъюнктивное множество} $M$ -- такое множество, что если в $M \vdash a \vee b$, то $a \in M$ или $b \in M$.
\begin{lemma}
$M \vdash a \Rightarrow a \in M$
\end{lemma}
\begin{proof}
Пусть это не так. Рассмотрим $a \rightarrow a \vee \neg a$. Раз $M \vdash a$, то $M \vdash a \vee \neg a$. Т.к. $a \notin M$, то $\neg a \in M$ по определению дизъюнктивности $M$. Но тогда из $M \vdash a$ и $M \vdash \neg a$ мы можем доказать, что $M \vdash a \land \neg a$.
\end{proof}
\item Возьмем множество всех дизъюнктивных множеств с формулами из ИИВ. Мы можем это сделать, т.к. ИИВ дизъюнктивно. Для любого элемента $W_{i} \vdash a, a \in W_{i}$, значит в этом мире $a$ вынуждено. Построим дерево с порядком <<быть подмножеством>>. Докажем, что это множество - модель Крипке. Проверим 5 свойств:
\begin{enumerate}
\item $W, x \Vdash P \Leftrightarrow v(x, P) = 1$ если $P \in V$ ($V$ - множество вынужденных переменных). Монотонность выполняется по определению дерева
\item $W, x \Vdash (A \land B) \Leftrightarrow W, x \Vdash A$ и $W, x \Vdash B$\\
С помощью аксиомы $A \land B \rightarrow A$ доказываем $W \vdash A$, значит $A \in W$. Аналогично с $B$
\item $W, x \Vdash (A \vee B) \Leftrightarrow W, x \Vdash A$ или $W, x \Vdash B$\\
Очевидно по определению дизъюнктивности
\item $W, x \Vdash (A \rightarrow B) \Leftrightarrow \forall y \geq x (W, y \Vdash A \Rightarrow W, y \Vdash B)$\\
Мы знаем, что $W \vdash A \rightarrow B$. Пусть в $W$ есть $A$, тогда по M.P. докажем, что $B$. Пусть в $W$ есть $B$, тогда мы уже получили $B$.
\item $W, x \Vdash \neg A \Leftrightarrow \forall y \geq x (W, x \nVdash A)$

Если где-то оказалось $A$, то оно доказуемо, а значит мы сможем доказать и $A \land \neg A$
\end{enumerate}
\item $\Vdash A$, тогда $W_{i} \Vdash A$. Рассмотрим $W_{0} = \lbrace$все тавтологии ИИВ$\rbrace$. $W_{0} \Vdash A$, т.е. $\vdash A$.
\end{enumerate}
\end{proof}

\subsection{Нетабличность интуиционистской логики}
\label{sec-6-5}
\begin{theorem}
Не существует полной модели, которая может быть выражена таблицей
\end{theorem}
\begin{proof}
Докажем от противного. Построим табличную модель и докажем, что она не полна. В ИВ мы обычно пользуемся алгеброй $J_{0}$ Яськовского $V = \lbrace 0, 1 \rbrace,$ $0 \leq 1$.\\
Пусть имеется $V = \lbrace ... \rbrace$, $\vert V \vert = n$ - множество истиностных значений. Пусть его размер больше 2.
Тогда построим формулу $V_{(1 \leq j < i \leq n + 1)}(p_{i} \rightarrow p_{j})$ - такая большая дизъюнкция из импликаций
\begin{enumerate}
\item Она общезначима, т.к. всего таких импликаций у нас будет $C_{n}^{2} >= n$ (по принципу Дирихле встретятся два одинаковых значения и она будет верна, тогда все выражение будет верно)
\item Недоказуемость. Построим такую модель Крипке, в которой она будет не общезначима.\\
$J_{0}$ - алгебра Яськовского. Определим последовательность алгебр $L_{n}$ по следующим правилам: $L_{0} = J_{0}$, $L_{n} = \Gamma(L_{n - 1})$. Таким образом $L_{n}$ - упорядоченное множество $\lbrace 0, w_{1}, w_{2}, ..., 1 \rbrace$. Пусть $f$ - оценка в $L_{n}$, действующая по следующим правилам на нашу формулу: $f(a_{1}) = 0$, $f(a_{n+1}) = 1$, $f(a_{i}) = w_{i}$ при $j < i$ $:$ $f(a_{i} \rightarrow a_{j}) = f(a_{i}) \rightarrow f(a_{j}) = f(a_{j})$. Последнее выражение не может являться $1$, так что формула недоказуема. (ИИВ полно относительно алгебры Гейтинга)
\end{enumerate}
\end{proof}

\section{Ticket 5: Логика 2 порядка}
\label{sec-7}
\subsection{Основные определения}
\label{sec-7-1}
Смотрим коснпект ДГ
\subsection{Теорема о дедукции}
\label{sec-7-2}
\begin{theorem}
Если $\Gamma, \alpha \vdash \beta$, и в доказательстве отсутствуют применения правил для кванторов, использующих свободные переменные  из формулы $\alpha$, то $\Gamma \vdash \alpha \rightarrow \beta$
\end{theorem}
\begin{proof}
Будем рассматривать формулы в порядке сверху вниз. На $i$-ой строке встретили формулу $\delta_{i}$. Тогда докажем, что $\alpha \rightarrow \delta_{i}$. Разберем случаи:
\begin{enumerate}
\item $\delta_{i}$ - старая аксиома, совпадает с $\alpha$ или выводится по правилу M.P.\\
Тогда мы знаем, что делать из Теоремы о дедукции для ИВ
\item $\delta_{i}$ - новая аксиома\\
Тогда все то же самое, что и в старой аксиоме, но нужно так же проверить условие.
\item $\exists x (\psi) \rightarrow \phi$ - новое правило вывода
\begin{itemize}
\item Докажем вспомогательную лемму:
\begin{lemma}
$(\alpha \rightarrow (\beta \rightarrow \gamma)) \rightarrow (\beta \rightarrow (\alpha \rightarrow \gamma))$
\end{lemma}
\begin{proof}
Докажем, что $\alpha \rightarrow (\beta \rightarrow \gamma), \beta, \alpha \vdash \gamma$:\\
\begin{tabular}{lll}
(1) & $\alpha \rightarrow \beta \rightarrow \gamma$& Допущение\\
(2) & $\alpha$& Допущение\\
(3) & $\beta \rightarrow \gamma$& M.P. 2,1\\
(4) & $\beta$& Допущение\\
(5) & $\gamma$& M.P. 4,3\\
\end{tabular}
\end{proof}
\item По индукционному преположению мы знаем, что $\alpha \rightarrow \psi \rightarrow \phi$. Тогда докажем, что $\alpha \rightarrow \psi \rightarrow \phi, (\alpha \rightarrow \psi \rightarrow \phi) \rightarrow (\psi \rightarrow \alpha \rightarrow \phi) \vdash \alpha \rightarrow \exists x (\psi) \rightarrow \phi$:\\
\begin{tabular}{lll}
(1) & $(\alpha \rightarrow \psi \rightarrow \phi) \rightarrow (\psi \rightarrow \alpha \rightarrow \phi)$& Допущение\\
(2) & $\alpha \rightarrow \psi \rightarrow \phi$& Допущение\\
(3) & $\psi \rightarrow \alpha \rightarrow \phi$& M.P. 2,1\\
(4) & $\exists x (\psi) \rightarrow \alpha \rightarrow \phi$& Правило вывода 1\\
(5) & $(\exists x (\psi) \rightarrow \alpha \rightarrow \phi) \rightarrow (\alpha \rightarrow \exists x (\psi) \rightarrow \phi)$& Допущение\\
(6) & $\alpha \rightarrow \exists x (\psi) \rightarrow \phi$& M.P. 4,5\\
\end{tabular}
\end{itemize}
\item $\phi \rightarrow \forall x (\psi)$ - новое правило вывода
\begin{itemize}
\item Докажем вспомогательную лемму 1
\begin{lemma}
$(\alpha \land \beta \rightarrow \gamma) \rightarrow (\alpha \rightarrow \beta \rightarrow \gamma)$
\end{lemma}
\begin{proof}
Докажем, что $(\alpha \land \beta \rightarrow \gamma), \alpha, \beta \vdash \gamma$:\\
\begin{tabular}{lll}
(1) & $\alpha$& Допущение\\
(2) & $\beta$& Допущение\\
(3) & $\alpha \rightarrow \beta \rightarrow \alpha \land \beta$& Сх. акс. 1\\
(4) & $\beta \rightarrow \alpha \land \beta$ M.P. 1,3\\
(5) & $\alpha \land \beta$& M.P. 2,4\\
(6) & $\alpha \land \beta \rightarrow \gamma$& Допущение\\
(7) & $\gamma$& M.P. 5,6\\
\end{tabular}
\end{proof}
\item Докажем вспомогателньую лемму 2
\begin{lemma}
$(\alpha \rightarrow \beta \rightarrow \gamma) \rightarrow (\alpha \land \beta \rightarrow \gamma)$
\end{lemma}
\begin{proof}
Докажем, что $\alpha \rightarrow \beta \rightarrow \gamma, \alpha \land \beta \vdash \gamma$:\\
\begin{tabular}{lll}
(1) & $\alpha \land \beta \rightarrow \alpha$& Сх. акс. 4\\
(2) & $\alpha \land \beta$& Допущение\\
(3) & $\alpha$& M.P. 2,1\\
(4) & $\alpha \land \beta \rightarrow \beta$& Сх. акс. 5\\
(5) & $\beta$& M.P. 2,4\\
(6) & $\alpha \rightarrow \beta \rightarrow \gamma$& Допущение\\
(7) & $\beta \rightarrow \gamma$& M.P. 3,6\\
(8) & $\gamma$& M.P. 5,7\\
\end{tabular}
\end{proof}
\item По индукционному предположению мы знаем, что $\alpha \rightarrow \psi \rightarrow \phi$. Тогда докажем, что $\alpha \rightarrow \psi \rightarrow \phi \vdash \alpha \rightarrow \psi \rightarrow \forall (\phi)$.\\
\begin{tabular}{lll}
(1) & $(\alpha \rightarrow \psi \rightarrow \phi) \rightarrow (\alpha \land \psi \rightarrow \phi)$& Вспомогательная лемма 1\\
(2) & $\alpha \rightarrow \psi \rightarrow \phi$& Допущение\\
(3) & $\alpha \land \psi \rightarrow \phi$& M.P. 2,1\\
(4) & $\alpha \land \psi \rightarrow \forall (\phi)$& Правило вывода 2\\
(5) & $(\alpha \land \psi \rightarrow \forall (\phi)) \rightarrow (\alpha \rightarrow \psi \rightarrow \forall (\phi))$& Вспомогательная лемма 2\\
(6) & $\alpha \rightarrow \psi \rightarrow \forall (\phi)$& M.P. 4,5\\
\end{tabular}
\end{itemize}
\end{enumerate}
\end{proof}
\subsection{Корректность исчисления предикатов}
\label{sec-7-3}
Смотрим конспект ДГ

\section{Полнота исчисления предикатов}
\label{sec-8}
Тут можно почитать конспект Д.Г.
\subsection{Свойства противоречивости}
\label{sec-8-1}
Противоречивая теория – теория, в которой можно вывести p, ¬p.
\begin{lemma}
Теория противоречива $\Leftrightarrow$ в ней выводится $a \land \neg a$
\end{lemma}
\begin{proof}
$\Leftarrow$ Если выводится $a \land \neg a$, то противоречива -- очевидно через аксиомы\\
$\Rightarrow$ Если противоречива, то выводится $a \land \neg a$\\
\begin{tabular}{lll}
(1)& $\neg \alpha$& Допущение\\
(2)& $\alpha$& Допущение\\
(3)& $\alpha \to \neg \alpha \to (\alpha \land \neg \alpha)$& Сх. акс. 10\\
(4)& $\neg \alpha \to (\alpha \land \neg \alpha)$& M.P. 1,3\\
(5)& $\alpha \land \neg \alpha$& M.P. 2,4\\
\end{tabular}\\
\end{proof}
Заметим, что всякое подмножество непротиворечивого множества непротиворечиво.\\
Заметим, что всякое бесконечное прот. множество содержит конечное противоречивое подмножество ввиду конечности вывода.\\
Совместное множество – множество с моделью (все формулы множества верны в какой-либо интерпретации).
\subsection{Лемма о дополнении непротиворечивого множества}
\label{sec-8-2}
\begin{lemma}
Для всякого непротиворечивого множества $\Gamma$ замкнутых формул сигнатуры $\sigma$ существует множество $\Gamma’$, являющееся к тому же полным, имеющее ту же сигнатуру и содержащее $\Gamma$.
\end{lemma}
\begin{proof}
Для не более чем счетных сигнатур:\\
Давайте добавлять недостающие формулы в $\Gamma$ -- если есть формула $\alpha$, добавим $\alpha$ или $\neg \alpha$ в зависимости от того, является ли $\Gamma \cup \alpha$ или $\Gamma \cup \neg \alpha$ противоречивым или нет (выберем непротиворечивый вариант). Одно всегда верно, потому что:
\begin{enumerate}
\item $\Gamma \cup \alpha$, $\Gamma \cup \neg \alpha$ противоречивы обе $\Rightarrow$ Мы можем доказать, что $\Gamma$ изначально было противоречиво
\item $\Gamma \cup \alpha$, $\Gamma \cup \neg \alpha$ не противоречивы обе $\Rightarrow$ Тогда можно сказать, что $\alpha \to \neg \alpha \to \alpha \land \neg \alpha$.
\end{enumerate}
\end{proof}
\subsection{Условие о интерпретации непротиворечивого мн-ва}
\label{sec-8-3}
Будем называть интерпретацией непротиворечивого множества формул функцию оценки, тождественно равную $1$ на элементах из этого множества. Будем говорить, что $\Gamma \vDash \alpha$, если она тождественна в любой модели $\Gamma$.
\subsection{Несколько лемм}
\label{sec-8-4}
\begin{lemma}
$\Gamma \vdash \alpha \Rightarrow \Gamma \vDash \alpha$
\end{lemma}
\begin{proof}
Механическая проверка аксиом
\end{proof}
\begin{lemma}
Eсли у $\Gamma$ есть модель, то $\Gamma$ непротиворечиво
\end{lemma}
\begin{proof}
Пусть $\Gamma$ имеет модель, но противоречиво, тогда из $\Gamma$ выводится $\alpha, \neg \alpha$, по корректности $\Gamma \vDash \alpha, \neg \alpha$, но формула и ее отрицание не могут быть общезначимыми одновременно.
\end{proof}
\begin{lemma}
Пусть $\Gamma$ -- полное непротиворечивое множество бескванторных формул. Тогда существует модель для $\Gamma$.
\end{lemma}
\begin{proof}
Построим модель структурной индукцией по формулам.\\
Предметное множество -- строки, содержащие выражения.\\
Например $\llbracket c_1 \rrbracket = \text{<<$c_1$>>}$,
$\llbracket f_1 (c_1, f_2(c_2)) \rrbracket = \text{<<$f_1 (c_1, f_2(c_2))$>>}$

Мы не хотим заниматься подсчетом, а предпочитаем оставлять то, что нужно вычислить как отдельную функцию. Рассмотрим формулу -- предикат. Его оценка истина, если он принадлежит носителю, ложна если его отрицание в носителе (в предметном множестве). Элементы всегда входят противоречиво (элемент не вдохит со своим отрицанием. Связки определим естественным образом. Докажем, что $\gamma \in \Gamma \Leftrightarrow \gamma$ истинна ($\Gamma$ -- предметное множество)
\begin{itemize}
\item База:\\
Если атомарная формула лежит в $\Gamma$, то она истинна по определению.\\
Если атомарная формула истинна, то лежит в $\Gamma$
\item Переход:
\begin{enumerate}
\item $\alpha \land \beta$

Если $\alpha \land \beta$ лежит в $\Gamma$, то оно истинно по определению
\begin{itemize}
\item Пусть $\llbracket \alpha \land \beta \rrbracket = \texttt{И}$, тогда покажем, что $\alpha \land \beta \in \Gamma$.\\
По таблице истинности $\land$ ясно, что $\llbracket \alpha \rrbracket = \texttt{И}$ и $\llbracket \beta \rrbracket = \texttt{И}$. Тогда $\alpha$ и $\beta$ лежат в $\Gamma$ по индукционному предположению. Тогда с помощью $\alpha \to \beta \to \alpha \land \beta$ можно показать, что и $\alpha \land \beta \in \Gamma$.
\item Пусть $\llbracket \alpha \land \beta \rrbracket = \texttt{Л}$, тогда покажем, что $\neg(\alpha \land \beta) \in \Gamma$.\\
По таблице истинности $\land$ ясно, что $\llbracket \alpha \rrbracket = \texttt{Л}$ или $\llbracket \beta \rrbracket = \texttt{Л}$. Для определенности возьмем, что $\alpha$ -- ложь. Тогда $\neg \alpha$ лежат в $\Gamma$ по индукционному предположению.\\
Докажем, что $\neg \alpha \vdash \neg (\alpha \land \beta)$:\\
\begin{tabular}{lll}
(1) & $\neg\alpha$ & Предположение\\
(2) & $\neg\alpha \rightarrow \alpha\land\beta\rightarrow\neg\alpha$ & Сх. акс. 1\\
(3) & $\alpha\land\beta \rightarrow \neg\alpha$ & M.P. 1,2\\
(4) & $\alpha \land \beta \rightarrow \alpha$ & Сх. акс. 4\\
(5) & $(\alpha \land \beta \rightarrow \alpha) \rightarrow (\alpha \land \beta \rightarrow \neg\alpha) \rightarrow \neg(\alpha \land \beta)$ & Сх. акс. 9\\
(6) & $(\alpha \land \beta \rightarrow \neg\alpha) \rightarrow \neg(\alpha \land \beta)$ & M.P. 5,4\\
(7) & $\neg(\alpha \land \beta)$ & M.P. 6,3
\end{tabular}
\end{itemize}
\item $\alpha \vee \beta$
\begin{itemize}
\item $\llbracket \alpha \vee \beta \rrbracket = \texttt{И}$. Тогда по таблице истинности $\vee$ либо $\llbracket \alpha \rrbracket = \texttt{И}$, либо $\llbracket \beta \rrbracket = \texttt{И}$. Не умаляя общности скажем, что $\llbracket \alpha \rrbracket = \texttt{И}$. Тогда $\alpha \in \Gamma$ по предположению индукции. Легко можно доказать, что и $\alpha \vee \beta \in \Gamma$ с помощью $\alpha \to \alpha \vee \beta$.
\item $\llbracket \alpha \vee \beta \rrbracket = \texttt{Л}$. Тогда по таблице истинности $\vee$ и $\llbracket \alpha \rrbracket = \texttt{Л}$, и $\llbracket \beta \rrbracket = \texttt{Л}$. Тогда $\neg \alpha \in \Gamma$ и $\neg \beta \in \Gamma$ по предположению индукции. С помощью 9-ой схемы аксиом мы можем доказать, что и $\neg (\alpha \vee \beta) \in \Gamma$.
\end{itemize}
\item Аналогично нужно доказать все связки
\end{enumerate}
\end{itemize}
\end{proof}
\subsection{Построение \texorpdfstring{$\Gamma ^ *$}{Г*}}
\label{sec-8-5}
\begin{theorem}
Можно построить из нашего множества формул множество бескванторных формул
\end{theorem}
\begin{proof}
Для этого определим такую операцию избавления от 1 квантора: Построим новый язык, отличающийся от нашего контантами, там будут $d_i^j$, где нижний индекс -- это поколение, верхний – нумерационный. Возьмем непротиворечивое множество формул $\Gamma_i$ и пополним его, получив непротиворечивое множество формул $\Gamma_{i+1}$, такое что $\Gamma_i \subset \Gamma_{i+1}$. Возьмем формулу $\gamma \in \Gamma_{i}$. Рассмотрим случаи:
\begin{enumerate}
\item Не содержит кванторов\\
Тогда делать ничего не нужно
\item $\gamma = \forall x (a)$\\
Тогда возьмем все константы, использующиеся в $\Gamma_{i}$ -- это будут $c_i$, $d_a^j$, где $a \leq i$. Занумеруем их $\theta_1, \theta_2, \dots$. И добавим формулы $a_1=a[x:=\theta_1], \dots$ к $\Gamma_{i+1}$.
\item $\gamma = \exists x (a)$\\
Тогда возьмем новую константу $d_{i+1}^j$ и добавим $a[x:=d_{i+1}^j]$ к $\Gamma_{i+1}$.
\end{enumerate}

Заметим, что сами формулы с кванторами мы не выкидываем -- ведь в будущем появятся новые формулы, и процесс для уже использованных кванторных формул нужно будет повторить. Покажем, что полученные множества остаются непротиворечивыми. $\Gamma_i$ непротиворечиво, а $\Gamma_{i+1}$ противоречиво, тогда $\Gamma_{i+1} \vdash \alpha \land \neg \alpha$, тогда выпишем конечное доказательство, найдем посылки, новые в $\Gamma_{i+1}$, которых нету в $\Gamma_{i}$, выпишем их и впихнем направо по теореме о дедукции: $\Gamma_{i} \vdash \gamma_1 \to \gamma_2 \to \gamma_3 \to \dots \to \gamma_n \to \beta \land \neg \beta$ Новые посылки у нас получаются только из пунктов 2 и 3.

\begin{enumerate}
\item $\gamma_1 = a[x:=\theta_1]$ из $\forall x (a)$. Тогда рассмотрим доказательство:\\
\begin{tabular}{lll}
$(1)$ & $\forall x \alpha \rightarrow \alpha [x := \theta]$ & Сх. акс. $\forall$\\
$(2)$ & $\forall x \alpha$ & $\forall x \alpha$ из $\Gamma_g$\\
$(3)$ & $\alpha [x := \theta]$ & M.P. $2,1$\\
$(4 \dots k)$ & $\alpha [x := \theta] \rightarrow (\gamma_2 \rightarrow \dots \gamma_n \rightarrow \beta \land \neg \beta)$ & Исх. формула\\
$(k+1)$ & $\gamma_2 \rightarrow \dots \gamma_n \rightarrow \beta \land \neg \beta$ & M.P. $3,k$
\end{tabular}
\item $\gamma_1 = a[x:=d_{i+1}^k]$ из $\exists x(a)$ выберем переменную, не участвующую в выводе противоречия -- $z$.
    Заменим все вхождения $d^k$ в д-ве на $z$.
    Поскольку $d_{i+1}^k$ -- константа, мы можем делать такие замены.
    Поскольку $z$ -- константа, специально введенная для замены и раньше не встречавшаяся, то она
    отсутствует в $\gamma_2,\dots$ + мы можем правильно выбрать $b$, чтобы и в нем отсутствовала $d_{i+1}^k$.
    Значит мы можем применить правило для выведения $\exists$:\\
\begin{tabular}{lll}
$(1 \dots k)$ & $\alpha [x := y] \rightarrow (\gamma_2 \rightarrow \dots \gamma_n \rightarrow \beta \land \neg \beta)$ & Исх. формула\\
$(k+1)$ & $\exists y \alpha [x := y] \rightarrow (\gamma_2 \rightarrow \dots \gamma_n \rightarrow \beta \land \neg \beta)$ & Правило для $\exists$\\
$(k+2)$ & $\exists x \alpha$ & Т.к. $\exists x \alpha$ из $\Gamma_g$ \\
$(k+3 \dots l)$ & $\exists y \alpha [x := y]$ & Доказуемо \\
$(l+1)$ & $\gamma_2 \rightarrow \dots \gamma_n \rightarrow \beta \land \neg \beta$ & M.P. $l, k+1$
\end{tabular}
\end{enumerate}
Возьмем $\Gamma_0 = \Gamma$. $\Gamma^* = \cup \Gamma_i$. $\Gamma^*$ также не противоречиво, потому что д-во использует конечное количество предположений, добавленных на каком-то шаге $j$ максимум, значит множество $\Gamma_j$ тоже противоречиво, что невозможно по условию.
\end{proof}
\subsection{Доказательство того, что дополненное бескванторное подмножество \texorpdfstring{$\Gamma^*$}{Г*} -- модель для \texorpdfstring{$\Gamma$}{Г}}
\label{sec-8-6}
\begin{theorem}
Дополненное бескванторное подмножество $\Gamma^*$ -- модель для $\Gamma$
\end{theorem}
\begin{proof}
Выделим в $\Gamma^*$ бескванторное подмножество $G$. Пополним его по лемме 2 (лемма о дополнении непротиворечевиого множества) модель сделаем из него по лемме о бескванторной модели. Покажем, что это модель для всего $\Gamma^*$, а значит и для $\Gamma$. Рассмотрим $\gamma \in \Gamma^*$, покажем, что $[\gamma] = \texttt{И}$.
\begin{itemize}
\item База\\
Формула не содержит кванторов. Истинность гарантируется леммой о бескванторном множестве.
\item Переход\\
Пусть $G$ это модель для любой формулы из $\Gamma^*$ с $r$ кванторами, покажем что она остается моделью для $r+1$ квантора.
\begin{enumerate}
\item $\gamma = \forall x (a)$\\
Покажем, что формула истинна для любого $t \in D$. По построению подели есть такое $\theta$, что $t = "\theta$ (string). По построению $\Gamma^*$ начиная с шага $p+1$ мы добавляем формулы вида $a[x:=k]$, где $k$ -- конструкция из констант и ф.симв. Также каждая константа ($c_i$ или $d_i^j$) из $\theta$ добавлена на некотором шаге $s_k$. То есть будет шаг $l=max(max(s_k), p)$, на котором $\theta$ обретет смысл и в $\Gamma_{l+1}$ будет присутствовать $a[x:=\theta]$. В формуле $a$ на один квантор меньше, значит она истинна по предположению индукции.
\item $\gamma = \exists x (a)$\\
По построению $\Gamma^*$ как только добавили $a$ к $\Gamma_i$, так сразу в следующем мире $\Gamma_{i+1}$ появляется $a[x:=d_{i+1}^k]$. Значит формула истинна на значении $"d_{i+1}^k"$, то есть истинна.
\end{enumerate}
\end{itemize}
\end{proof}
\subsection{Следствие -- если \texorpdfstring{$\vDash \alpha$}{⊨ a}, то \texorpdfstring{$\vdash \alpha$}{⊢ a}}
\label{sec-8-7}
\begin{theorem}
$\vDash \alpha \Rightarrow \vdash \alpha$
\end{theorem}
\begin{proof}
\begin{itemize}
\item Пусть $\Gamma \nvdash a$, тогда по полноте множества $\Gamma$, $\Gamma \vdash \neg a$, но у $\Gamma$ есть модель, в которой $\Gamma \vDash \neg a$. То есть $\Gamma \nvDash a$. Но $\Gamma$ по построению то же, что и модель теории, то есть все рассуждения $\Gamma \vdash a$ равноценны в предикатах $\vdash a$.
\item Пусть $\nvdash a$, тогда пусть $\Gamma=\lbrace \neg a \rbrace$
\begin{enumerate}
\item $\Gamma$ непротиворечиво\\
Пусть $\Gamma$ противоречиво, значит $\forall b \Gamma \vdash b, \Gamma \vdash \neg b$;
\begin{enumerate}[label=(\alph*)]
\item $\neg a \vdash b, \neg a \vdash b$;
\item $\neg a \vdash a, \neg a \vdash \neg a$;
\item $\vdash \neg a \to a, \neg a \to \neg a$;
\item $\vdash (\neg a \to a) \to (\neg a \to \neg a) \to \neg \neg a$;
\item $\vdash \neg \neg a \to a$;
\item $\vdash a \to\leftarrow а$ недоказуемо по условию.;
\end{enumerate}
\item $\Gamma$ подходит под условие теоремы Гёделя о полноти исчисления предикатов, то есть у $\Gamma$ есть модель. Тогда в ней оценка $[\neg a] = 1$, значит оценка $[a] = 0$, то есть $\nvDash a$. Мы доказали мета-контрпозицию $\nvdash a \Rightarrow \nvDash a$.
\end{enumerate}
\end{itemize}
\end{proof}

\section{Формальная арифметика}
\label{sec-9}
\subsection{Структуры и модели, теория первого порядка}
\label{sec-9-1}
Теория первого порядка - это формальная система с кванторами по
функциональным символам, но не по предикатам. Рукомахательное
определение – это фс с логикой первого порядка в основе, в которой
абстрактные предикаты и функциональные символы определяются точно
(а может такое определение даже лучше).

Структура по ДГ:

Структурой теории первого порядка мы назовем упорядоченную тройку
$\langle D, F, P\rangle$, где F -- списки оценок для 0-местных, 1-местных и т.д.
функций, и $P = P_0, P_1,\dotsc$ -- списки оценок для 0-местных,
1-местных и т.д. предикатов, $D$ — предметное множество.

Понятие структуры — развитие понятия оценки из исчисления предикатов.
Но оно касается только нелогических составляющих теории; истинностные
значения и оценки для связок по-прежнему определяются исчислением
предикатов, лежащим в основе теории. Для получения оценки формулы
нам нужно задать структуру, значения всех свободных индивидных
переменных, и (естественным образом) вычислить результат.

Структура по-моему:

Все то же самое определение из ИВ. Мы просто забиваем на предикаты
в ИВ (не определям их), расширяем нашу сигнатуру (добавляя конкретные
предикаты и функциональные символы), определяем для нее интерпретацию.

Модель -- это корректная структура (любое доказуемое утверждение должно
быть в ней общезначимо).
\subsection{Аксиомы Пеано}
\label{sec-9-2}
Множество $N$ удовлетворяет аксиоматике Пеано, если:
\begin{enumerate}
\item $0 \in N$
\item $x \in N, succ(x) \in N$
\item $\nexists x \in N : (S(x) = 0)$
\item $(succ(a) = c \land succ(b) = c) \to a = b$
\item $P(0) \land \forall n.(P(n) \to P(succ(n))) \to \forall n.P(n)$
\end{enumerate}
\subsection{Формальная арифметика -- аксиомы, схемы, правила вывода}
\label{sec-9-3}
Формальная арифметика -- это теория первого порядка, у которой
сигнатура определена как: (циферки, логические связки, алгебр.
связки, '), а интерпретацию сейчас будем определять.
Интерпретация определяет два множества -- $V, P$ -- истинностные и
предметные значения. На самом деле нет никакого множества P,
мы определяем только $V$, потому что оно нужно для оценок. Все
элементы, которые мы хотели бы видеть, выражаются в сигнатуре.
Пусть множество $V = \lbrace 0, 1 \rbrace$ по-прежнему.
Определим оценки логических связок естественным образом.
Определим алгебраические связки так:
\begin{align*}
    +(a, 0 ) &= a \\
    +(a, b') &= (a + b)' \\
    *(a, 0 ) &= 0 \\
    *(a, b') &= a * b + a
\end{align*}

\textbf{Тут должно быть что-то на уровне док-ва $2+2=4$}
\subsubsection{Аксиомы}
\label{sec-9-3-1}
\begin{enumerate}
\item $a = b \to a' = b'$
\item $a = b \to a = c \to b = c$
\item $a' = b' \to a = b$
\item $\lnot (a' = 0)$
\item $a + b' = (a + b)'$
\item $a + 0 = a$
\item $a * 0 = 0$
\item $a * b' = a * b + a$
\item $\phi[x:=0] \land \forall x.(\phi \to \phi[x:=x']) \to \phi$
\end{enumerate}
\subsubsection{a = a}
\label{sec-9-3-1-1}
\begin{lemma}
$\vdash a = a$
\end{lemma}
\begin{proof}
$\vdash a = a$\\
% Работает — не трожь ©
\begin{tabular}{@{}lll}
(1)& $a = b \to a = c \to b = c$& Сх. акс. ФА 2\\
(2)& $T$& Сх. акс.\\
(3)& $(a = b \to a = c \to b = c) \to T \to (a = b \to a = c \to b = c)$& Сх. акс. 1\\
(4)& $T \to (a = b \to a = c \to b = c)$& M.P. 1,3\\
(5)& $T \to \forall a (a = b \to a = c \to b = c)$& ПВ $\forall$\\
(6)& $T \to \forall a \forall b (a = b \to a = c \to b = c)$& ПВ $\forall$\\
(7)& $T \to \forall a \forall b \forall c (a = b \to a = c \to b = c)$& ПВ $\forall$\\
(8)& $\forall a \forall b \forall c (a = b \to a = c \to b = c)$& M.P. 2,7\\
(9)& \begin{tabular}[t]{@{}l}$\forall a \forall b \forall c (a = b \to a = c \to b = c) \to$\\
\hspace{5cm}$\forall b \forall c (a + 0 = b \to a + 0 = c \to b = c)$\end{tabular}& \begin{tabular}[t]{@{}l}\\Сх. акс. ИП 1\end{tabular}\\
(10)& $\forall b \forall c (a + 0 = b \to a + 0 = c \to b = c)$& M.P. 8,9\\
(11)& \begin{tabular}[t]{@{}l}$\forall b \forall c (a + 0 = b \to a + 0 = c \to b = c) \to$\\
\hspace{5cm}$(\forall c (a + 0 = a \to a + 0 = c \to a = c))$\end{tabular}& \begin{tabular}[t]{@{}l}\\Сх. акс. ИП 1\end{tabular}\\
(12)& $\forall c (a + 0 = a \to a + 0 = c \to a = c)$& M.P. 10,11\\
(13)& \begin{tabular}[t]{@{}l}$(\forall c (a + 0 = a \to a + 0 = c \to a = c)) \to$ \\
\hspace{5cm}$(a + 0 = a \to a + 0 = a \to a = a)$\end{tabular}& \begin{tabular}[t]{@{}l}\\Сх. акс. ИП 1\end{tabular}\\
(14)& $a + 0 = a \to a + 0 = a \to a = a$& M.P. 12,13\\
(15)& $a + 0 = a$& Сх. акс. ФА 6\\
(16)& $a + 0 = a \to a = a$& M.P. 15,14\\
(17)& $a = a$& M.P. 15,16\\
\end{tabular}\\
\end{proof}

\section{Ticket 8: рекурс, Аккерман}
\label{sec-10}
\subsection{Рекурсивные функции}
\label{sec-10-1}
Рассмотрим примитивы, из которых будем собирать выражения:

\begin{enumerate}
\item $Z \colon \mathbb{N} \rightarrow \mathbb{N}$, $Z(x) = 0$
\item $N \colon \mathbb{N} \rightarrow \mathbb{N}$, $N(x) = x'$
\item Проекция. $U^n_i \colon \mathbb{N}^n \rightarrow \mathbb{N}$, $U^n_i (x_1, \dotsc, x_n) = x_i$
\item Подстановка. Если $f \colon \mathbb{N}^n \rightarrow \mathbb{N}$ и $g_1, \dotsc, g_n \colon \mathbb{N}^m \rightarrow \mathbb{N}$, 
    то $S\template{f,g_1,\dotsc, g_n} \colon \mathbb{N}^m \rightarrow \mathbb{N}$.

    При этом $S\template{f,g_1,\dotsc, g_n} (x_1,\dotsc, x_m) = f(g_1(x_1,\dotsc, x_m), \dotsc, g_n(x_1,\dotsc, x_m))$
\item Примитивная рекурсия. Если $f \colon \mathbb{N}^n     \to \mathbb{N}$ и 
                                 $g \colon \mathbb{N}^{n+2} \to \mathbb{N}$, то
    \[R\template{f, g}(x_1\ldots{}x_n, n) = \begin{cases}
    	f(x_1, \dotsc, x_n) & n = 0 \\
    	g(x_1, \dotsc, x_n, n, R\ltemplate f, g\rtemplate(x_1, \dotsc, x_n, n - 1)) & n > 0
    \end{cases}\]
\item Минимизация. Если $f \colon \mathbb{N}^{n+1} \rightarrow \mathbb{N}$,
    то $\mu \template{f}\colon \mathbb{N}^n \rightarrow \mathbb{N}$, при этом
  $\mu \template{f} (x_1,\dotsc,x_n)$ --- такое минимальное число $y$, что $f(x_1,\dotsc, x_n, y) = 0$.
  Если такого $y$ нет, результат данного примитива неопределен.
\end{enumerate}

Пример:
\[a + b = R\template{U^2_1, S\template{N, U^3_3}}(a, b)\]
\subsection{Характеристическая функция и рекурсивное отношение}
\label{sec-10-2}
\begin{itemize}
\item \emph{Характеристическая фукнция} -- функция от выражения, которая возвращает $1$ если выражение истинно, $0$ иначе.
\item \emph{Рекурсивное отношение} -- отношение, характеристическая функция
которого рекурсивна.
\end{itemize}
\subsection{Аккерман не примитивно-рекурсивен, но рекурсивен (второе)}
\label{sec-10-3}
Функция Аккермана -- это функция, удовлетворяющая следующим правилам:
\[
    A(m,n) = \begin{cases}
        n+1 & m = 0\\
        A(m-1,n) & m > 0, n = 0\\
        A(m-1,A(m,n-1)) & m > 0, n > 0
    \end{cases}
\]
Например:
\[A(2, 0) = A(1, 1) = A(0, A(1, 0)) = A(0, 2) = 3\]
\begin{lemma}
$A(m, n) \geq 1$
\end{lemma}
\begin{proof}
$A(m, n)$ определена только на натуральных числах\\
$A(0, 0) = 1, A(1, 0) = A(0, 1) = 2, A(0, 1) = 2$, а все остальное ещё больше
\end{proof}
\begin{lemma}
\label{lemma1a}
$A(1, n) = n + 2$
\end{lemma}
\begin{proof}
\begin{align*}
    A(1, n) &= A(0, A(1, n - 1)) \\
    &= A(0, A(0, A(1, n - 2))) \\
    &= A(0, A(0, A(0, \ldots{} A(1, 0)))) \\
    &= A(0, A(0, A(0, \ldots{} 2))) \\
    &= n + 2 && \text{($n$ раз инкрементируем двойку)}
\end{align*}
\end{proof}
\begin{lemma}
\label{lemma1b}
$A(2, n) = 2n + 3$
\end{lemma}
\begin{proof}
$A(2, n)
= A(1, A(1, \ldots{} A(2, 0)))
= A(1, A(1, \ldots{} 3))
= 2n + 3$ ($n$ раз к тройке прибавляем $A(0, 1) = 2$)
\end{proof}
\begin{lemma}
\label{lemma2}
$A(m, n) \geq n + 1$
\end{lemma}
\begin{proof}
В первом случае $A \geq n + 1 = n + 1$\\
Во втором $A$ может перейти в первый случай, который работает
хорошо, или в третий.\\
В третьем случае мы можем получить $A(0, n)$ если первый аргумент
был нулем, тогда все ок, можем получить $A(1, 0)$, тогда это второй
случай, для него условие выполнено.\\
Третий ссылается на второй, а второй на третий, но тут
нет противоречия, потому что мы знаем, что функция Аккермана
завершается.
\end{proof}
\begin{lemma}
\label{lemma3a}
$A(m, n) < A(m, n + 1)$
\end{lemma}
\begin{proof}
Проведем индукцию по $m$:
\begin{itemize}
\item База:\\
$A(0, n) = n + 1 < n + 2 = A(0, n + 1)$
\item Переход:\\
$A(k + 1, m) < A(k + 1, m) + 1$\\
$\leq A(k, A(k + 1, m))$ (По \ref{lemma2})\\
$\leq A(k + 1, m + 1)$   (3-е свойства ф-ии Аккермана)
\end{itemize}
\end{proof}
\begin{lemma}
\label{lemma3b}
$A(m, n + 1) \leq A(m + 1, n)$
\end{lemma}
\begin{proof}
Проведем индукцию по $n$:
\begin{itemize}
\item База:\\
$A(m, 0 + 1) = A(m, 1) = A(m + 1, 0)$ (ii)
\item Переход, предположение:
    \begin{align*}
        A(m, j + 1) &\leq A(m + 1, j) && \hspace{-3cm} \text{По \ref{lemma2}}\\
        (j + 1) + 1 &\leq A(m, j + 1)\\
        A(m, (j + 1) + 1) &\leq A(m, A(m, j + 1)) && \hspace{-3cm} \text{По монотонности}\\
        A(m, A(m, j + 1)) &\leq A(m, A(m + 1, j)) && \hspace{-3cm} \text{По монотонности + предположение}\\
        A(m, (j + 1) + 1) &\leq A(m, A(m + 1, j)) = A(m + 1, j + 1)\\
                          &                       && \hspace{-3cm} \text{По 3-му свойству ф-ии Аккермана}
    \end{align*}
\end{itemize}
\end{proof}
\begin{lemma}
\label{lemma3c}
$A(m, n) < A(m + 1, n)$
\end{lemma}
\begin{proof}
$A(m, n) < A(m, n + 1) \leq A(m + 1, n)$ (По \ref{lemma3a}, \ref{lemma3b})
\end{proof}
\begin{lemma}
\label{lemma4}
$A(m_1, n) + A(m_2, n) < A(\max(m_1, m_2) + 4, n)$
\end{lemma}
\begin{proof}
\begin{align*}
&A(m_1, n) + A(m_2, n)\\
    \le {}&A(\max(m_1, m_2), n) + A(\max(m_1, m_2), n)\\
    = {}&2 \cdot A(\max(m_1, m_2), n)\\
    < {}&2 \cdot A(\max(m_1, m_2), n) + 3\\
    = {}&A(2, A(\max(m_1, m_2), n)) && \text{По \ref{lemma1a}}\\
    < {}&A(2, A(\max(m_1, m_2) + 3, n)) && \text{Строгая монотоннасть по обоим арг.}\\
    < {}&A(\max(m_1, m_2) + 2, A(\max(m_1, m_2) + 3, n)) && \text{По \ref{lemma3c}}\\
    = {}&A(\max(m_1, m_2) + 3, n + 1)   && \text{3-е свойство ф-ии Аккермана}\\
    \le {}&A(\max(m_1, m_2) + 4, n) && \text{По \ref{lemma3b}}\\
\end{align*}
\end{proof}
\begin{lemma}
\label{lemma5}
$A(m, n) + n < A(m + 4, n)$
\end{lemma}
\begin{proof}
    \begin{align*}
    A&(m, n) + n \\
    < A&(m, n) + n + 1 \\
    = A&(n, m) + A(0, n) \\
    < A&(m + 4, n)
    \end{align*}
\end{proof}
\begin{theorem}
Функция аккерманна не притивно-рекурсивна
\end{theorem}
\begin{proof}
Пусть $f(n_1 \dotsc n_k)$ - примитивная рекурсивная функция, $k \geq 0$.\\
$\exists J : f(n_1 \dotsc n_k) < A(J, \sum(n_1 \dotsc n_k))$

Пусть $\overline n = (n_1, \dotsc, n_k)$\\
Индукция по рекурсивным функциям
\begin{itemize}
\item База:\\
$f(\overline n)$ - $N$ или $Z$ или $U^k_j$
\begin{enumerate}
\item $f(\overline n) = N, k = 1;$ Пусть $J = 1$, по (i) и лемме 3c\\
$f(n) = N(n) = n + 1 = A(0, n) < A(1, n) = A(J, n) = A(J, \sum(\overline n))$
\item $f(\overline n) = Z, k = 1;$\\
$f(n) = 0 < A(J, n)$ (потому что $A \geq 1$) $= A(J, \sum(\overline n))$
\item $f(\overline n) = U^k_j; k = k;$\\
Пусть $J=1$\\
    $f(n_1\dotsc, n_k) = U_{kj}(n_1\dotsc, n_k) = n_j$\\
Пусть $n_j = 0$, тогда $f(n) = 0 < A(J, \sum(\overline n))$ для любого нормального $J$
Пусть $n_j > 0$, тогда $f(n) = (n_j - 1) + 1 = A(0, n_j - 1) < A(1, n)
= A(J, \sum(\overline n))$
\end{enumerate}
\item Переход
\begin{enumerate}
\item Предположим, что $f(\overline n) = S\template{h, g_1\ldots{}g_m}(\overline n) = h(g_1(\overline n) \ldots g_m(\overline n))$\\
По предположению индукции существует $J_0$ для $h$, $J_1, \dotsc, J_m$ для $g_1\ldots{}g_m$.
\begin{align*}
&f(\overline n) = h(g_1(\overline n),..)\\
    \le {}&A(J_0, \sum\{i=1..m\}(\overline n)) && \text{По выбору $J_0$}\\
    < {}& (J_0, \sum(A(J_i, \sum(\overline n)))) && \text{По выбору $J_i$ и строгой монотонности}\\
// {}&J* = \max(J_1..J_m) + 4(m - 1)\\
    < {}&A(J_i, A(J*, \sum(\overline n))) && \text{По \ref{lemma4} примененной $m-1$ раз}\\
    < {}&A(J_i, A(J*+1, \sum(\overline n))) && \text{По монотонности}\\
    \leq {}&A(J_0, A(\max(J_0, J*) + 1, \sum(\overline n))) && \text{По монотонности}\\
    \leq {}&A(\max(J_0, J*) + 1, \sum(\overline n) + 1) && \text{3-е свойство ф-ии Аккермана}\\
    = {}&A(\max(J_0, J*) + 2, \sum(\overline n)) && \text{По \ref{lemma3b}}\\
\end{align*}
Тогда пусть $j=\max(J_0, J*) + 2$
\item Пусть $f(\overline n) = R\template{h,g}(\overline n)$\\
$f(n_1, \dotsc, n_k, 0) = h(n_1, \dotsc, n_k)$\\
$f(n_1, \dotsc, n_k, m+1) = g(n_1, \dotsc, n_k, m, f(n_1, \dotsc, n_k, m))$\\
По предположению имеем $J_0 (h), J_1 (g).$\\
Пусть $J = \max(J_0, J_1) + 4$
\begin{enumerate}
\item
\begin{align*}
&f(\overline n, 0)\\
\leq {}&f(\overline n, 0) + \sum(\overline n)\\
= {}&h(\overline n) + \sum(\overline n)\\
< {}&A(J_0, \sum(\overline n)) + \sum(\overline n)\\
    < {}&A(J_0 + 4, \sum(\overline n)) && \text{По \ref{lemma5}}\\
    < {}&A(J, \sum(\overline n)) && \text{По монотонности}\\
= {}&A(J, \sum(\overline n) + 0)\\
\end{align*} 
\item 
\begin{align*}
&f(\overline n, k + 1)\\
= {}&g(\overline n, k, f(\overline n, k))\\
    < {}&A(J_1, \sum(\overline n) + k + f(\overline n, k)) && \hspace{-1cm}\text{По выбору $J_1$}\\
    < {}&A(J_1, \sum(\overline n) + k + 1 + f(\overline n, k)) && \hspace{-1cm}\text{По монотонности}\\
    = {}&A(J_1, A(0, \sum(\overline n) + k) + f(\overline n, k))   &&\hspace{-1cm}\text{По 1-му свойству ф-ии Аккермана}\\
    < {}&A(J_1, A(0, \sum(\overline n) + k) + H(J, \sum(\overline n)+k))&& \text{По предположению}\\
    < {}&A(J_1, A(J, \sum(\overline n)+k)+A(J, \sum(\overline n) + k)) && \text{По монотонности $(J > 0)$}\\
= {}&A(J_1, 2 * [A(J, \sum(\overline n) + k)])\\
< {}&A(J_1, 2 * [A(J, \sum(\overline n) + k)] + 3)\\
    = {}&A(J_1, A(2, A(J, \sum(\overline n) + k))) && \hspace{-1cm}\text{По \ref{lemma1a}}\\
    < {}&A(J_1, A(J_1 + 1, A(J, \sum(\overline n) + k))) && \hspace{-1cm}\text{По строгой монотонности $(J_1 > 2)$}\\
    = {}&A(J_1 + 1, A(J, \sum(\overline n) + k) + 1) &&\hspace{-1cm}\text{По 3-му свойству ф-ии Аккермана}\\
\leq {}&A(J_1 + 2, A(J, \sum(\overline n) + k))\\
    < {}&A(J - 1, A(J, \sum(\overline n) + k)) &&\hspace{-2.5cm}\text{По монот. $J > \max(..) + 4$}\\
    = {}&A(J, \sum(\overline n) + (k + 1)) &&\hspace{-2.5cm}\text{По 3-му свойству ф-ии Аккермана, $J \ne 0$}\\
\end{align*}
\end{enumerate}
\end{enumerate}
\end{itemize}
\end{proof}
\begin{theorem}
Функция Аккермана рекурсивна
\end{theorem}
\begin{proof}
Можем сказать, что он рекурсивный, потому что мы можем
его написать на компьютере, а тьюринг выражается в рекурсивных функциях.
\end{proof}

\section{Ticket 9: представимость}
\label{sec-11}
\subsection{Функции, их представимость}
\label{sec-11-1}
Арифметическая функция -- это отображение $f: N_0^n \to N_0$ \\
Арифметическое отношение -- это $P \in N_0^n$ \\
Если $k \in N_0$, то $\overline k = 0'''''^\cdots$, где количество штрихов есть $k$.
\begin{itemize}
\item Арифметическое отношение $R \in N_0^n$ выразимо в ФА, если\\
$\exists a$ с $n$ свободными переменными:
$a(x_1,\dotsc, x_n)$, такая что
\begin{enumerate}
\item Eсли $R(k_1,\dotsc, k_n)$, то $\vdash a(\overline{k_1}, \dotsc, \overline{k_n})$
\item Eсли $\lnot R(k_1, \dotsc, k_n)$, то $\vdash \lnot a(\overline{k_1}, \dotsc, \overline{k_n})$
\end{enumerate}
\item $C_R$ - функция, равная $1$, если $R$, и равная $0$, если $\lnot R$
\item $\exists !y.\phi(y) = \exists y.\phi(y) \& \forall a\forall b(\phi(a) \& \phi(b) \to a = b)$
\item $f: N_0^n \to N_0$ представима в ФА, если $\exists a(x_1\ldots{}x_{n+1})$, что
$\forall x_1\dotsc x_{n+1}:$
\begin{enumerate}
\item $f(x_1, \dotsc, x_n) = x_{n+1} \Leftrightarrow \vdash a(\overline{x_1}, \dotsc, \overline{x_{n+1}})$
\item $\exists !b(a(\overline{x_1}, \dotsc, \overline{x_n}, b))$
\end{enumerate}
\end{itemize}
\subsection{Теорема о связи представимости и выразимости}
\label{sec-11-2}
\begin{theorem}
$R$ выразимо $\Leftrightarrow$ $C_r$ представимо
\end{theorem}
\begin{proof}
$\Rightarrow$ $a$ выражает $R$\\
$(a \to (x_n₊_1=0')) \& (\lnot a \to (x_n₊_1=0))$
представляет $C_r$\\
По выразимости $R \vdash a$; тогда $⊤\to a\to ⊤ => a\to ⊤$\\
По 10i, перенесенной к нам $a \to (\lnot a \to ⊥)$\\
правило с единственностью вроде понятно (хотя руками помахал, да)

$\Leftarrow$ $C_r \text{ представимо} \to R \text{ выразимо}$
Пусть представлять $C_r$ будет
$a(x_1, \dotsc, x_n,x_{n+1})$
Тогда определим, какая формула выражает $R$:
$a(\dotsc, 1)$
Из представимости:
\begin{itemize}
\item $\exists b.a(x_1\ldots{}x_{n+1})$
\item $\forall x\forall y(a(\ldots{}x) \& a(\ldots{} y) \to x = y)$
\item если $C_r(x_1\ldots{}x_n) = 1$, то $\vdash a(x_1\ldots{}x_n,1)$
\item если $C_r(\dots) = 0$, то $\vdash a(\dotsc, 0)$
\end{itemize}
Докажем выводимость
\begin{enumerate}
\item Покажем, что если $R(x_1\ldots{}x_n)$, то $\vdash a(x_1\ldots{}x_n, 1)$
Из представимости прямо ровно.
\item Покажем, что если $\lnot R(x_1\ldots{}.x_n), то \vdash \lnot a(x_1\ldots{}x_n, 1)$\\
По единственности

$\forall x\forall y(a(x_1\ldots{}x_n, x) \& a(x_1\ldots{}x_n, y) \to x = y)$\\
$a(x_1\ldots{}x_n, 0) \& a(x_1\ldots{}x_n, 1) \to (0 = 1)$ (спустя две акс. и 2 MP)\\
Делаем дедукцию\\
$a(x_1\ldots{}x_n, 0) \& a(x_1\ldots{}x_n, 1) \vdash ⊥$\\
$a(x_1\ldots{}x_n, 0) \& a(x_1\ldots{}x_n, 1) \to a(x_1\ldots{}x_n, 0)$\\
$a(x_1\ldots{}x_n, 0)$\\
$\lnot a(x_1\ldots{}x_n, 0)$ по представимости
$a(x_1\ldots{}x_n, 0) \to (\lnot a(x_1\ldots{}x_n, 0) \to \lnot a(x_1\ldots{}x_n, 1))$ (10i в ИИВ, доказуема в предикатах)\\
$\lnot a(x_1\ldots{}x_n, 1)$\\
Хотим $\lnot a(x_1\ldots{}x_n, 1)$
\end{enumerate}
\end{proof}
\subsection{beta-функция Гёделя, китайская теорема об остатках}
\label{sec-11-3}
$\beta(b, c, i) = b \% (1 + c * (1 + i))$\\
Где $\%(a, b) = d$, что $\forall m . (d + m * b = a), m \geq 0, 0 \leq d \leq b$

\subsubsection{Китайская теорема об остатках}
\label{sec-11-3-1}
\begin{theorem}
$n_1\ldots{}n_k$ - попарно взаимно простые целый числа\\
$r_1\ldots{}r_k$ - любые целые числа, что $0 \leq r_1 < n_1$\\
Тогда: $\exists b \forall i  r_1 = b \% n_k$
\end{theorem}
\begin{proof}
Без доказательства
\end{proof}

\subsubsection{Гёделева Г-последовательность}
\label{sec-11-3-2}
$\Gamma_1 = (i + 1) * c + 1$\\
$\Gamma(c) = 1 * c + 1, 2 * c + 1, 3 * c + 1, \ldots (n + 1) * c + 1$
\begin{theorem}
$\Gamma(c)$ подходит на роль $n_1 \ldots n_k$ в китайской теореме об остатках
\end{theorem}
\begin{proof}
Выделим последовательность размера $n$: $k_1 \ldots k_n$.\\
Чтобы это выполнялось возьмем $c = (max(k_1\ldots{}k_n))!$
\begin{enumerate}
\item В $\Gamma$ любые два элемента попарно взаимно простые
\label{sec-11-3-2-1}
Пусть $\Gamma_1 \vdots \Gamma_j$ имеют общий делитель $p > 1$. Мы можем его разложить на простые множители и взять какой-нибудь простой (любое число раскладывается на простые множители).\\
Тогда $(\Gamma_1 - \Gamma_j) \divby p, (c * (i - j)) \divby p$. Заметим, что $\lnot (c \divby p)$, потому что иначе $\Gamma_1 = 1 + c * (i + 1) \divby p$ и $c * (i + 1) \divby p$, а они отличаются на единицу. Тогда $(i - j) \divby p$, но $c = m! m > n$, а $i - j < n$, значит $c \divby p$.
\item Каждое $k_1 < Г_1$
\label{sec-11-3-2-2}
$k_1 \leq c < 1 + c * (i + 1) = \Gamma_1$
\end{enumerate}
\end{proof}
\begin{comment}
%%%%%%%%%%%%%%%%%%%%%%%%%%%%%%%%%%%%%%%%%%%%%%%%%%%%%%%%%%%%%%%%%%%%%%%%%%%%%%%%%%%%%%%%%%
\subsubsection{Лемма о β-функции}
\label{sec-11-3-3}
Увидим, что β(b,c,i) считает остаток от деления b на
(i + 1) * c + 1 - элемент Геделевой последовательности.
\begin{itemize}
\item <a_0\ldots{}a_n>\in N \to \exists b\exists c(aₖ=β(b,i,c)) - β-функция кодирует
последовательность натуральных чисел и может
доставать по индексу i

a_0\ldots{}a_n  - последовательность натуральных чисел
тогда существует такое c, что Г = 1*c+1, 2*c+1,\ldots{}
если c ≥ max(a_0\ldots{}a_n), то aₖ < (i+1)*c+1
Но по свойству Г элементы попарно взаимно просты
тогда сравнения
a_0 \% (0+1)*c+1
a_1 \% (1+1)*c+1
\ldots{}\ldots{}\ldots{}\ldots{}.
a_n \% (n+1)*c+1
имеют общее решение b по китайской теореме об остатках
тогда aᵢ = b \% (i+1)*c + 1
но это и есть β-функция
aᵢ = β(b,c,i)
\end{itemize}
\subsubsection{Представимость β-функции Гёделя в ФА}
\label{sec-11-3-4}
β-функция представима в ФА отношением
B(b,c,i,d) = \exists q((b = q * (1 + c * (i + 1)) + d) \& (d < 1 + c * (i + 1)))
Пусть 1 + c * (i + 1) = z
Докажем условия представимости:
\begin{enumerate}
\item Эквивалентность
\begin{enumerate}
\item β(b,c,i) = d, тогда $\vdash B(b,c,i,d)$
b = z * (1 + c * (i + 1))    (это и следующее - из леммы о β) P
d < 1 + c * (i + 1)                                           Q
P \to Q \to P \& Q
P \& Q
P \& Q \to \exists q.(P \& Q) [z:= q]
\exists q.(P \& Q)
\item Пусть $\vdash B(b,c,i,d)$, тогда
\exists q.(P \& Q)
подберем такое q (по лемме)
P \& Q \to P
P \& Q \to Q
P
Q
значит β(b,c,i) = d
\end{enumerate}
\item Единственность
Следует из леммы.
\end{enumerate}
\subsection{Теорема о представимости рекурсивных функций Z, N, U}
\label{sec-11-4}
\begin{enumerate}
\item Z
Z(a, b) = (b = 0)
\begin{itemize}
\item Z(a) = b верно, тогда b = 0
b = 0
\item (b = 0)
b = 0
тогда Z(0) = 0, все ок
\item \exists y.\phi(y) \& \forall a\forall b(\phi(a) \& \phi(b) \to a = b)
Тоже как-то несложно
\end{itemize}
\item N
N(a, b) = (a = b')
\begin{itemize}
\item N(a) = b, тогда a = b'
a = b'
\item a = b', тогда
N(a) = b
\item Третье не хочу
\end{itemize}
\item U_nᵢ
U_nᵢ(x_1\ldots{}x_n) = (x_1 = x_1) \& (x₂ = x₂) \& \ldots{} \& (x_n₊_1 = xᵢ)
\begin{itemize}
\item U(..) = xᵢ, тогда x_n₊_1 = x_1
x_1 = x_1 доказывается
\ldots{}
x_n = x_n доказывается
x_n₊_1 = x_1 по условию
объединяем все \&
\item (x_1 = x_1) \& \ldots{}.
вытаскиваем каждый элемент и тогда видим, что
проекция делает ровно то, что должна.
\item \exists q.(x_n₊_1 = q)
ХЗ
\item \forall a\forall b(x(\ldots{}a)\&x(\ldots{}.b) \to a = b)
Для конкретных a, b обявляем a = b - ⊤, тогда выводим
из него конъюнкцию и навешиваем два квантора
\end{itemize}
\end{enumerate}
\subsection{Теорема о представимости S}
\label{sec-11-5}
Eсли f и g_1\ldots{}g_n представимы, то S<f, g_1\ldots{}g_n> представима
Пусть F, G_1\ldots{}G_n представляют их.
S(a_1\ldots{}aₘ, b) = \exists b_1\ldots{}\exists b_n(G_1(a_1\ldots{}a_n, b_1) \& \ldots{} \& Gn(a_1\ldots{}aₘ, b_n)
\& F(b_1\ldots{}b_n, b))
\begin{itemize}
\item Пусть S(a_1\ldots{}a_n) = b, тогда существуют такие b_1\ldots{}b_n, что \#каждый аргумент\#
Поскольку f,g_1\ldots{}g_n представимы, то доказуемы по представимости
f(b_1\ldots{}b_n, b)
g_1(a_1\ldots{}a_n, b_1)
\ldots{}
g_n(a_1\ldots{}a_n, b_n)
g_1 \& g₂ \& \ldots{} \& g_n \& f    объединили \&     "P"
"P" \to \exists b_1."P[b_1:=b_1]" + MP
\ldots{}
Ну и навесили кванторы, да.
\item Пусть верна формула с кванторами. Тогда она и есть уже то, что надо
\item не могу, да и вообще нигде это свойство не доказывается
\end{itemize}
\subsection{Теорема о представимости R}
\label{sec-11-6}
Пусть f, g представимы F, G. Тогда R<f,g> представима.
f: N^n\to N, g:N^n⁺²\to N
r(x_1\ldots{}x_n, k, a) =
   \exists b\exists c(
        \exists k(β(b, c, 0, k) \& \phi(x_1\ldots{}x_n, k))
        \& B(b, c, x_n₊_1, a)
        \& \forall k(k<x_n₊_1 \to \exists d\exists e(B(b,c,k,d)\&B(b,c,k',e)\&G(x_1\ldots{}x_n,k,d,e))))
Единственная возможность осознать -- внимательно прочесть формулу.
Тут β-функция используется в качестве функии отображения нашего шага
вычисления рекурсии в результат, типа
0 - F(\ldots{})
1 - G(\ldots{})
\ldots{}
n - G(\ldots{})
\subsection{Теорема о представимости μ}
\label{sec-11-7}
f: N^{n+1}\to N представима F, тогда μ<f> представима M
Μ<F>(x_1\ldots{}x_n₊_1) = F(x_1\ldots{}x_n, x_n₊_1, 0) \& \forall y((y < x_n₊_1) \to \lnot F(x_1\ldots{}x_n, y, 0))
\begin{itemize}
\item μ<f>(x_1\ldots{}x_n) = x_n₊_1, тогда x_n₊_1 - минимальное k, такое что f(x_1..x_n,k) = 0
то есть имеем
F(x_1\ldots{}x_n, x_n₊_1, 0)
\forall x.(k < x \to \lnot F(x_1\ldots{}x_n, k, 0))
Просто объединим конъюнкцией
\item обратно ей же и разъединим
\end{itemize}
\end{comment}

\end{document}
\section{Ticket 10: Тьюринг}
\label{sec-12}
\subsection{Арифметические отношения, их выразимость}
\label{sec-12-1}
\begin{itemize}
\item Арифметическое отношение R \in N₀ⁿ выразимо в ФА, если
\exists a с n свободными переменными:
a(x_1\dots x_n), такая что
\begin{enumerate}
\item Eсли $R(k_1\dots k_n), то \vdash a(k_1\textasciitilde{}\dots k_n\textasciitilde{})$
\item Eсли \lnot R(k_1..k_n), то $\vdash \lnot a(k_1\textasciitilde{}\dots k_n\textasciitilde{})$
\end{enumerate}
\end{itemize}
\subsection{Гёделева нумерация}
\label{sec-12-2}
\begin{center}
\begin{tabular}{lrl}
a & `a & описание\\
\hline
( & 3 & \\
) & 5 & \\
, & 7 & \\
\lnot  & 9 & \\
\to & 11 & \\
\lor & 13 & \\
\& & 15 & \\
\forall  & 17 & \\
\exists  & 19 & \\
xₖ & 21 + 6 * k & переменные\\
fⁿₖ & 23 + 6 *ᵏ * 3ⁿ & n-местные функцион. символы (', +, *)\\
Pⁿₖ & 25 + 6 *ᵏ * 3ⁿ & n-местные предикаты (=)\\
\hline
\end{tabular}
\end{center}

Последовательность значков будем составлять так:
a_1\dots a_n - наши простые числа, соответствующие char'ам, тогда
p_1$^{\text{(a_1)}}$ * p₂$^{\text{(a₂)}}$,\dots p_n$^{\text{(a_n)}}$ - геделев нумерал стринга, составленного
из чаров.

Если a - выражение, то `a - выражение в Геделевой форме (на
практике пишут квадратные скобочки без нижних их половинок)
Тогда если a - выражение, `a\textasciitilde{} - это элемент предметного множества
ФА, соответствующий нолику с количеством черточек, равным `a.

Доказательство - это последовательность простых чисел, возведенная
в геделевы нумералы выражений, являющихся составляющими док-ва, по
порядку. Аналогично с составлением строки из символов.

Тогда определим следующие функции операций с нумералами:
\begin{itemize}
\item \plog(a, b) = max n : a \% bⁿ = 0
Иногда вместо b стоит P$_{\text{b}}$, где P$_{\text{b}}$ - простое число с индексом b.
Функция берет геделев нумерал и достает у него i-й элем. последов.
\item len = max n : a \% p_n
Возвращает длину строки/д-ва
\item s@t = p_1$^{\text{(\plog(s, 1))}}$ * \dots  * pₗₑ_n₍ₛ₎$^{\text{(\plog(s, len(s)))}}$ *
pₗₑ_n₍ₛ₎₊_1$^{\text{(\plog(t, 1))}}$ * \dots  * pₗₑ_n₍ₛ₎₊ₗₑ_n₍ₜ₎$^{\text{(\plog(t, len(t)))}}$
Конкатенация строк
\end{itemize}
\subsection{Машина Тьюринга}
\label{sec-12-3}
Машина тьюринга состоит из ленты, головки, регистра состояния и конечной
таблицы состояний
Более формально, это 7-кортеж: <Q, Γ, b, ∑, δ, q₀, F>
Конечный список состояний, конечный алфавит, пустой символ из алфавита,
символы, которые мы можем писать (из Γ $\backslash$ b), функция таблицы состояний,
начальное состояние из Q, конечное состояние из Q.
\begin{itemize}
\item Лента - бесконечный двусвязный список, в каждой ячейке которого
содержится символ из конечного алфавита, в котором также есть
пустой символ (тут и далее  ), которым изначально заполнена вся
лента
\item Головка может находиться над элементом, писать в него и читать из
него символ. Может двигаться влево-вправо (или двигать ленту, неважно)
\item Регистр состояния хранит состояние - элемент из конечного множества
состояний машины. Есть особые состояния - стартовое и конечные.
\item Таблица состяний - таблица, хранящая данные о функции смены
состояния - foo: Γ × Q \to Γ × Q × \{left, this, right\}.
Функция берет текущее состояние, читает символ на головке, потом
получает тройку, пишет новый символ, перемещается по третьему
элементу, выставляет новое состояние. Если состояние конечное, то
она останавливается.
\end{itemize}
Мы будем придерживаться нотации <\_, \_,  , \_, \_, S, F>.
\subsection{Проблема останова}
\label{sec-12-4}
Дано описание процедуры и входные данные. Функция P(a, b) определяет,
остановится ли a на входных данных b. Существует ли P?
\begin{itemize}
\item Проблема останова неразрешима на машине Тьюринга:
Пусть P существует.
Тогда S(x) = P(x, x) остановится ли функция на своем же коде
MyProg(x) = if S(x) then while(true)\{\} else 1
Рассмотрим MyProg(`MyProg)
Если оно остановится, то первое условие выполнено, тогда оно не остановится
И наоборот.
Значит, P не существует.
\end{itemize}
\subsection{Выводимость и рек. функции - Тьюринг}
\label{sec-12-5}
\subsubsection{Выражение машин Тьюринга через рекурсивные функции}
\label{sec-12-5-1}
Мы хотим доказать, что если у нас есть какая-нибудь процедура, которую
можно выразить в Тьюринге, то мы можем ее сделать и в формальной арифметике
(рекурсивные функции представимы).
Введем обозначение <st,tape,pos> = 2$^{\text{(st)}}$*3$^{\text{(tape)}}$*5$^{\text{(pos)}}$
Такая тройка -- основная характеристика машины в данный момент.
Будем называеть ее текущим полным состоянием, например.
st, tape, pos - геделевы нумералы, st - нумерал из 1 элемента с состоянием,
tape - string, обозначающий ленту (бесконечные   слева и справа не входят),
pos - позиция в ленте.
\begin{itemize}
\item p: <st, a> \to <st, a, dir>
принимает <st, a>, декодит, лезет в δ машины тьюринга, достает
новые значения, делает из них <,,>, отдает.
\item t: <st> \to 0 | 1
Определяет, терминально ли наше состояние (0 если терминально)
\item \epsilon  - пустой символ (у нас  )
\item pb, pc кодируют \beta-функцией последовательность инпутов
в последовательность аутпутов. \beta(p$_{\text{b}}$, p$_{\text{c}}$, x) = p(x)
\item tb, tc аналогично кодируют t
\item R<f,g>(<s$_{\text{st}}$, s$_{\text{tape}}$, s$_{\text{pos}}$>, \epsilon , pb, pc, tb, tc, y)
Запускает машину Тьюринга от стартового состояния, заранее
говоря ей, сколько шагов (y) она должна сделать.
Возвращает тройку <st, tape, pos>
\item Определим f, g
\begin{enumerate}
\item Дополнительные функции
\begin{itemize}
\item os(prev) = \plog(prev, 1)
Текущее состояние
\item ot(prev) = \plog(prev, 2)
Лента
\item op(prev) = \plog(prev, 3)
Позиция головки в ленте
\item nextstate(pb, pc, prev) = \beta(pb, pc, 2$^{\text{(os(prev))}}$ * 3$^{\text{(\plog(ot(prev), op(prev)))}}$
           Реализует функцию p
\item st(pb, pc, prev) = \plog(nextstate(pb, pc, prev), 1)
Новое состояние.
\item sym(pb, pc, prev) = \plog(nextstate(pb, pc, prev), 2)
Символ который нужно писать
\item dir(pb, pc, prev) = \plog(nextstate(pb, pc, prev), 3)
Направление для перехода головки
\item repl(pb, pc, prev) = (ot(prev) / (P$_{\text{op}}$)$^{\text{(\plog(ot(..), op(..)))}}$) * (P$_{\text{op}}$)$^{\text{sym}}$(..)
Возвращает ленту, в которой удален символ в позиции op,
и добавлен новый символ в эту же позицию.
\end{itemize}
\item f - возвращает полное состояние машины
f(<start$_{\text{state}}$>, \epsilon , pb, pc, tb, tc) = <start$_{\text{state}}$>
\item g - возвращает новое полное состояние из машины после перехода
(пометка: 0 - nothing, 1 - right, 2 - left
все фукнции вызываются с аргументом prev, <start$_{\text{state}}$> не используется)
g(<start$_{\text{state}}$>, \epsilon , pb, pc, tb, tc, y, prev) =
\begin{center}
\begin{tabular}{lll}
\hline
Condition & Result & Desrc\\
\hline
dir = 0 & <st, repl, op> & nothing\\
dir = 1 \& len(repl) = op & <st, repl@2$^{\text{(\epsilon )}}$, op + 1> & tape end\\
dir = 1 & <st, repl, op + 1> & move right\\
dir = 2 \& op = 0 & <st, 2$^{\text{(\epsilon )}}$@repl, op - 1> & tape start\\
dir = 2 & <st, repl, op - 1> & move left\\
\hline
\end{tabular}
\end{center}
\end{enumerate}

\item Определим steps - функцию, определяющую необх. кол-во шагов
steps(<start$_{\text{state}}$>, \epsilon , pb, pc, tb, tc) =
μ<\beta(tb, tc, \plog(R<f,g>, 1))>(<start$_{\text{state}}$>, \epsilon , pb, pc, tb, tc)
Она найдет такое минимальное к, что состояние \plog(R<f, g>(args, k), 1) терминально.
\end{itemize}
\subsubsection{Выражение программы по проверке доказательства в машине тьюринга}
\label{sec-12-5-2}
\begin{itemize}
\item Emulate(input, prog) = \plog(R<f,g>(<`S, input, 0>,  , pb, pc, tb, tc, steps(-//-)), 1) == F
Функция проверяет, правда ли получившееся терминальное состояние - ок.
Можем давать программу такую, что она заканчивается в терминальном F(inish)
или в терминальном FAIL
Дает в качестве аргумента функцию перехода, pb, pc выражают prog
\item Proof(term, proof) = Emulate(proof, MY$_{\text{PROOFCHECKER}}$) \&\& (\plog(proof, len(proof)) = term)
Проверяет, что доказательство p заканчивается корректно и его последний
элемент - то, что мы доказываем.
\item Любая представимая в ФА ф-я является рекурсивной
Пусть f представима
Пусть f(x_1\dots x_n) = b, тогда $\vdash \phi(x_1\textasciitilde{}\dots x_n\textasciitilde{}, b\textasciitilde{})$
Всегда можно построить рекурсивную функцию G$_{\text{\phi}}$(x_1\dots x_n, b, p),
утверждающую, что p - гёделев номер вывода предиката \phi(x_1\textasciitilde{}\dots x_n\textasciitilde{}, b\textasciitilde{})
Мы делаем это обычным перебором чисел, проверяем вывод нашей
программой из домашнего задания, выраженную в тьюринге, а потом в
рекурсивных функциях.
Тогда f в рекурсивных функциях выражается так:
f(x_1\dots x_n) = \plog(μ<S<G$_{\text{\phi}}$, U_n₊_1,_1,\dots  U_n₊_1,_n,
\plog(U_n₊_1,_n₊_1, 1),
\plog(U_n₊_1,_n₊_1, 2)>>(x_1\dots x_n), 1)
Такая функция берет \plog(1) от первого такого минимального геделева
номера k (геделева пара из двух элементов - <b, p>)), что
S<G$_{\text{\phi}}$, U\dots .>(x_1\dots x_n, k) = 0. Это значит, что
G$_{\text{\phi}}$(x_1\dots x_n, \plog(k, 1), \plog(k, 2)) = 0, это значит, что
G$_{\text{\phi}}$(x_1\dots x_n, b, p) = 0, то есть p - вывод \phi(x_1\textasciitilde{}\dots x_n\textasciitilde{}, b\textasciitilde{}).
Этот геделев номер - b
\end{itemize}
\section{Ticket 11: 1т о неполноте}
\label{sec-13}
\subsection{Непротиворечивость, ω-непротиворечивость}
\label{sec-13-1}
\begin{itemize}
\item Теория непротиворечива, если в ней нельзя вывести
одновременно a и \lnot a (что аналогично невозможности
вывести a\&\lnot a).
\item Теория ω-непротиворечива, если из %\forall \phi(x) \vdash \phi(x\textasciitilde{})$ следует
⊬ \exists p\lnot \phi(p). Проще говоря, если мы взяли
формулу, то невозможно вывести одновременно \exists x\lnot A(x)
и A(0), A(1), \dots
\item Лемма о w-\# и обычной непротиворечивости
Если теория w-непротиворечива, то она непротиворечива
\phi = x=x \to x=x
Такая формула очевидно доказуема (A \to A)
$\vdash \phi[x:=k] k \in N₀$
Но недоказуемо \exists x\lnot (x=x\to x=x)
А в противоречивой теории доказуемо все
\end{itemize}
\subsection{Прервая теорема о неполноте}
\label{sec-13-2}
Определим отношение W_1(x, p), истинное тогда и только тогда,
когда x - геделев номер формулы \phi с единственным свободным
аргументом x, а p - геделев номер доказательства \phi("\phi"). Это
отношение выразимо в ФA, потому что мы просто пихаем это в наш
Proof, а его мы выразили через рекурсивные функции, а они
представимы.
Пусть его выражает w_1(x, p);
Рассмотрим формулу σ = \forall p\lnot w_1(x, p) - для любого доказательства
оно не является доказательством самоприменения \phi, то есть
самоприменение \phi недоказуемо.
То есть если σ(`a\textasciitilde{}) истинно, то a(`a\textasciitilde{}) недоказуемо.
В нашем случае если σ(`a\textasciitilde{}) истинно, то σ(`σ\textasciitilde{}) недоказуемо.
\begin{enumerate}
\item Если формальная арифметика непротиворечива, то недоказуемо σ(`σ\textasciitilde{})
\begin{enumerate}
\item Пусть $\vdash σ(`σ\textasciitilde{})$, тогда найдется геделев номер ее док-ва p,
тогда W_1(`σ, p), то есть $\vdash w_1(`σ\textasciitilde{}, p\textasciitilde{})$.
\item С другой стороны,
$\vdash σ(`σ\textasciitilde{})$
$\vdash \forall p\lnot w_1('σ\textasciitilde{}, p)$
\forall p\lnot w_1(`σ\textasciitilde{}, p) \to \lnot w_1(`σ\textasciitilde{}, p\textasciitilde{})
\lnot w_1(`σ\textasciitilde{}, p\textasciitilde{})
Тогда ФА противоречива.
\end{enumerate}
\item Если формальная арифметика w-непротиворечива, то недоказуемо \lnot σ(`σ\textasciitilde{})
Пусть $\vdash \lnot σ(`σ\textasciitilde{})$, то есть $\vdash \lnot \forall p\lnot w_1(`σ\textasciitilde{}, p)$, что значит \exists p.w_1(`σ\textasciitilde{}, p)
Найдется такой q, что $\vdash w_1(`σ\textasciitilde{}, q\textasciitilde{})$, потому что если бы не нашелся,
это бы значило доказуемость для каждого q \lnot w_1(`σ\textasciitilde{}, q\textasciitilde{}), тогда по
ω-непротиворечивости было бы не доказуемо \exists p\lnot \lnot w_1(`σ\textasciitilde{}, p)
То q, что мы нашли - это номер доказательства  σ(`σ\textasciitilde{}), что и
утверждает выражение $\vdash w_1(`σ\textasciitilde{}, q\textasciitilde{})$. Но мы предполагали, что $\vdash \lnot σ(`σ\textasciitilde{})$.
Противоречие.
\end{enumerate}

Нормальное доказательство общезначимости:
Я не знаю, зачем нам второй пункт, но из первого следует, что если
наша теория w-непротиворечива, то она непротиворечива (по лемме выше),
значит в ней недоказуемо σ(`σ\textasciitilde{}), то есть \forall p\lnot w_1(`σ\textasciitilde{}, p), то есть
по корректности последнее выражение И, но это и есть в точности определение
σ(`σ\textasciitilde{}).

Ненормальное д-во общезначимости:
Итого мы доказали, что если формальная арифметика ω-непротиворечива,
то в ней не доказуемо ни σ(`σ\textasciitilde{}) ни \lnot σ(`σ\textasciitilde{}). Одно из них точно тавтология
(в формуле нет свободных переменных). Тогда ФА неполна при условии
ω-непротиворечивости.

Другое доказательство общезначимости:
\lnot σ(`σ\textasciitilde{}) недоказуема
[σ(`σ\textasciitilde{})] = [\forall p\lnot w_1(`σ\textasciitilde{}, p)] =
\begin{enumerate}
\item И если [\lnot w_1(`σ\textasciitilde{}, a)] = И для какого-то а
\item Л иначе
\end{enumerate}

Это значит, что
И если [w_1(`σ\textasciitilde{}, a)] = Л
[w_1(`σ\textasciitilde{}, a)] = Л, докажем от противного
Пусть [σ(`σ\textasciitilde{})] = Л,
[\forall p\lnot w_1(`σ\textasciitilde{}, p)] = Л
[\lnot \forall p\lnot w_1(`σ\textasciitilde{}, p)] = И
[\exists p.w_1(`σ\textasciitilde{}, p)] = И
[w_1(`σ\textasciitilde{}, a)] = И для какого-то а
то есть a доказывает σ(`σ\textasciitilde{})
???

тогда по определению w_1 существует
доказательство σ(`σ\textasciitilde{}),
\subsection{Пример w-противоречивой, но непротиворечивой теории (при усл. непрот. ФА)}
\label{sec-13-3}
Добавим в ФА аксиому Г: \lnot σ(`σ\textasciitilde{})
Тогда по контрпозиции 1п2 она w-противоречива.
Если бы мы могли доказать противоречивость нашей системы, то
ФА была бы противоречива, тогда хз
$\lnot σ(`σ\textasciitilde{}) \vdash σ(`σ\textasciitilde{})\&\lnot σ(`σ\textasciitilde{})$
$\vdash σ(`σ\textasciitilde{})$
Но мы предположили что \lnot σ(`σ\textasciitilde{})
\subsection{Форма Россера}
\label{sec-13-4}
Если формальная арифметика непротиворечива, то в ней найдется
такая формула \phi, что ⊬\phi и ⊬\lnot \phi
\section{Ticket 12: 2т о неполноте}
\label{sec-14}
\subsection{Consis, Условия выводимости Гильберта-Бернайса}
\label{sec-14-1}
Определим Consis как утверждение, показывающее
непротиворечивость ФА - отсутствие $\phi : \vdash \phi, \lnot \phi$. Поскольку
в противоречивой теории выводится что угодно, возьмем что-то
недоказуемое, типа 1=0.
Consis = \forall p(\lnot Proof('(1=0)\textasciitilde{}, p))

Определим отношение Sub(a, b, c) истинно, если
a, b - `a, `b \& c = `a[x:=b] или
a \lor b не геделев номер и c = 0

Пусть Sub(a, b, c) выражает τ(a, b, c)

\begin{itemize}
\item Лемма о самоприменении
a(x) - формула, тогда \exists b такой что
\begin{enumerate}
\item $\vdash a(`b\textasciitilde{}) \to b$
\item $\vdash \beta \to a(`b\textasciitilde{})$
\end{enumerate}
b₀(x) = \forall t(τ(x, x, t) \to a(t))
b = b₀(`b₀\textasciitilde{})
\begin{enumerate}
\item $a(`b\textasciitilde{}) \vdash a(`b\textasciitilde{})$
$a(`b\textasciitilde{}) \vdash τ(`b₀\textasciitilde{}, `b₀\textasciitilde{}, `b\textasciitilde{}) \to a(`b\textasciitilde{})$    акс 1 + MP
$a(`b\textasciitilde{}) \vdash \top \to (τ(`b₀\textasciitilde{}, `b₀\textasciitilde{}, `b\textasciitilde{}) \to a(`b\textasciitilde{}))$
$a(`b\textasciitilde{}) \vdash \top \to \forall t(τ(`b₀\textasciitilde{}, `b₀\textasciitilde{}, t) \to a(t))$
$a(`b\textasciitilde{}) \vdash \forall t(τ(`b₀\textasciitilde{}, `b₀\textasciitilde{}, t) \to a(t))$
$a(`b\textasciitilde{}) \vdash b$
\item $b \vdash \forall t(τ(`b₀\textasciitilde{}, `b₀\textasciitilde{}, t) \to a(t))$    тут почти $a \vdash a$ написано
$b \vdash τ(`b₀\textasciitilde{}, `b₀\textasciitilde{}, `b\textasciitilde{})$             по выразимости
$b \vdash τ(`b₀\textasciitilde{}, `b₀\textasciitilde{}, `b\textasciitilde{}) \to a(`b\textasciitilde{})$    сняли квантор с 1
$b \to a(`b\textasciitilde{})$
\end{enumerate}

\item Условия Гильберта-Бернайса
\end{itemize}
Пусть πg(x, p) выражает Proof(x, p)
πr(x) = \exists t πg(x, t) тогда если
\begin{enumerate}
\item $\vdash a$ , то $\vdash πr(`a\textasciitilde{})$
\item $\vdash πr(`a\textasciitilde{}) \to πr(`πr(`a\textasciitilde{})\textasciitilde{})$
\item $\vdash πr(`a\textasciitilde{}) \to πr(`(a \to b)\textasciitilde{}) \to πr(`b\textasciitilde{})$
\end{enumerate}
\subsection{Вторая теорема о неполноте}
\label{sec-14-2}
\subsubsection{Рукомашеское доказательство без условий Г-Б}
\label{sec-14-2-1}
\begin{itemize}
\item Если арифметика непротиворечива, в ней нет д-ва Consis
рассмотрим Consis \to σ(`σ\textasciitilde{}).
Тогда если Consis доказуемо, то σ(`σ\textasciitilde{}) недоказуемо.
То есть это формулировка 1.1 Гёделя о неполноте.
Тогда если у нас будет Consis, мы сможем доказать
σ(`σ), тогда 1.1 фейлится. Значит Consis недоказуемо.

\item Доказательство того, что Consis недостаточно формален
Заменим Consis в д-ве на
Proof1(a, x) = Proof(a, x)\&\lnot Proof(`(1=0),x)
Consis1 = \forall x\lnot Proof1(`(1=0),x)
Если арифметика непротиворечива, то Proof1(a, x) = Proof(a, x)
Если арифметика противоречива, то Consis1 доказуема как и все
остальное.
Ну давайте менять.

Поменяли. Смотрим. хехехе, давайте докажем Consis1:
\lnot (π(x) \& \lnot π(x))              доказуемо в ИВ
\top
\top \to \lnot (π(x) \& \lnot π(x))          1 акс, MP
\lnot (π(x) \& \lnot π(x))
\forall x(\lnot (π(x) \& \lnot π(x)))

Тогда выходит, что мы можем доказать противоречивость арифметики.
Но это не так, бага вот в чем:
Замена consis на consis1 неоправдана - в consis1 есть
формула 1=0, на которой ее результат не вычисляется, а
постулируется.
Чтобы выражать Consis абстрактно, существуют условия выводимости
Гильберта-Бернайса.

Докажем, что consis1 не удовлетворяет 3 свойству Г-Б
Пусть Proof1(x,p) выражает π1.
$\vdash π1(`a\textasciitilde{}) \to π1(`a\to b\textasciitilde{}) \to π1(`b\textasciitilde{})$ оценим при a=(2=0), b=(1=0)
?       \to (   true  \to  false)
?       \to false
Если эта формула верна, то $\vdash π1(`a\textasciitilde{}$)
Тогда если π1(`a\textasciitilde{}), то Proof(2=0, x)\&\lnot Proof(`1=0, x) = И
Это значит что теория противоречива, потому что в ней выводимо 2=0,
но она непротиворечива, потому что недоказуемо 1=0. \to ←
\end{itemize}
\subsubsection{Доказательство 2 теоремы Гёделя о неполноте}
\label{sec-14-2-2}
Пусть π удовлетворяет условиям Г-Б
Consis = \lnot π(1=0)
ФА непротиворечива
Тогда ⊬ Consis

\begin{enumerate}
\item По лемме о самоприменении
\begin{enumerate}
\item \lnot π(γ) \to γ
\item γ \to \lnot π(γ)
\item \lnot γ \to π(γ)                            контрпозиция
\item π(γ) \to \lnot γ
\end{enumerate}
\item π(γ) \to π(\lnot γ)
\begin{enumerate}
\item $π(γ) \vdash π(`π(γ)\textasciitilde{})$                     ГБ2
\item $\vdash π(π(γ) \to \lnot γ)$                       ГБ1 от 1.4
\item $\vdash π(π(γ)) \to π(π(γ) \to \lnot γ) \to π(\lnot (γ))$   ГБ3
\item $π(γ) \vdash π(\lnot γ)$                         2MP (2.1, 2.2)
\end{enumerate}
\item $\vdash π(α \to \beta \to γ) and \vdash π(α) \to π(\beta) => \vdash π(α) \to π(γ)$
\begin{enumerate}
\item π(α \to \beta \to γ) \to π(α) \to π(b \to γ)        ГБ3
\item π(\beta \to γ) \to π(\beta) \to π(γ)                ГБ3
\item π(α) \to π(\beta \to γ)                       MP 1, given
\item π(α) \to π(\beta)                           given
\item π(α) \to π(γ)                           занести под дедукцию, ГБ3
\end{enumerate}
\item $\vdash π(γ) \to π(1=0)$
\begin{enumerate}
\item γ \to \lnot γ \to (1=0)                        10i в ИИВ, выводима в предикатах
\item $\vdash π(γ \to \lnot γ \to (1=0))$                   ГБ1
\item π(γ) \to π(\lnot γ)                          2
\item $\vdash π(γ) \to π(1=0)$                       MP 4.2 4.3
\end{enumerate}
\item ⊬ Consis
$\vdash \lnot π(1=0) \to \lnot π(γ)$                        контрапозиция 4
$\vdash Consis \to \lnot π(γ)$                         the same
$] \vdash Consis$, тогда $\vdash \lnot π(γ)$
$\vdash \lnot π(γ) \to γ => \vdash γ => \vdash π(γ)$            1.1, ГБ1
$\vdash \lnot π(γ), \vdash π(γ)  \to ←$
\end{enumerate}
\section{Ticket 13: ТМ}
\label{sec-15}
\subsection{Теория множеств}
\label{sec-15-1}
   Значит это такая теория первого порядка.
   В сигнатуре модели есть один пред.символ - \in
   Добавляем связку a \leftrightarrow b = (a \to b) \& (b \to a)
   σ \in Θ => \forall x(x \in σ \to x \in Θ)
o   σ = Θ => σ \in Θ \& Θ \in σ
   ∅ : \forall x(\lnot x \in ∅)
   x ∩ y = z, тогда \forall t(t \in z \leftrightarrow t \in x \& t \in y)
   Dj(x) \forall a\forall b(a \in x \& b \in x \& a \ne  b \to a ∩ b = ∅)
   X(a) - мн-во всех x пересекающихся ровно в одном эл-те с каждым из а
   и содержащих элементы из ∪a.
   X(\{\{1, 2\}, \{2', 3\}\}) = \{\{2, 3\}, \{1, 2'\}\}
\subsubsection{Если существует мн-во, то существует пустое мн-во}
\label{sec-15-1-1}
Аксиома выделения:
\forall x\exists b\forall y(y \in b \leftrightarrow (y \in x \& \phi(y)))
Возьмем наше существующее мн-во x
\exists b\forall y(y \in b \leftrightarrow (y \in x \& \phi(y)))
Пусть \phi(y) = \bot
тогда подставим ∅ вместо b:
\forall y(y \in ∅ \leftrightarrow (y \in x \& \bot))
Это выполняется вроде.
\subsubsection{Если x, то найдется \{x\}}
\label{sec-15-1-2}
\forall x\exists \{x\}\forall y(y \in \{x\} \to y = x)
\begin{enumerate}
\item Пусть x \ne  ∅
\{x\} = \{y | y \in \{x, y\} \& y \ne  ∅\}
по аксиоме объединения \exists p\forall y(y \in p \leftrightarrow \exists s(y \in s \& s \in x))
\forall y(y \in \{x, ∅\} \leftrightarrow \exists s(y \in s \& s \in x))

или по аксиоме пары
\exists p(x \in p \& ∅ \in p \& \forall z(z \in p \to (x = z \lor y = z)))
x \in \{x, ∅\} \& ∅ \in \{x, ∅\} \& \forall z(z \in \{x, ∅\} \to \dots \}

ДГ руками помахал тут, ну и я помахаю по причине
отсутствия времени доказывать

А, нет, вот, кажется:
По аксиоме степени \forall x\exists \{x, ∅\}\forall y(y \in \{x, ∅\} \leftrightarrow y \in x)
\forall x\exists \{x, ∅\}\forall y((y \in \{x, ∅\} \to y \in x)\&(y \in x \to y \in \{x, ∅\}))
\lnot y \in ∅, значит (y \in x \to y \in x) = \top
\forall x\exists \{x, ∅\}\forall y((y \in \{x, ∅\} \to y \in x)\&\top)
\forall x\exists \{x, ∅\}\forall y(y \in \{x, ∅\} \to y = x)   более слабое условие
\item x = ∅
P(∅) = \{∅\}
\end{enumerate}
\subsubsection{\exists !x(\forall y.\lnot (y \in x))}
\label{sec-15-1-3}
\exists x(\forall y.\lnot (y \in x)) \& \forall a\forall b((\forall y.\lnot (y \in a)) \& (\forall y.\lnot (y \in b)) \to a = b)
Первое по определению пустого множества и аксиоме выделения с \bot
\forall y.(\lnot y \in \{\}) \& \forall y.(\lnot y \in \{\}) \to \forall p((p\in x \to p\in y) \& (p\in y \to p\in x))
Второе как-то через ∅_1 \in ∅₂ и обратное включение
На основании того, что мы подставляем наши пустые множества, импликация
вырождается в \top \to \top
\subsubsection{x ∩ y существует}
\label{sec-15-1-4}
по теореме выделения
\forall x\exists b\forall y(y \in b \leftrightarrow (y \in x \& \phi(y)))
\forall y(y \in x ∩ y \leftrightarrow (y \in x \& t \in y))
\subsection{Аксиоматика ZFC}
\label{sec-15-2}
\subsubsection{Аксиома равенства}
\label{sec-15-2-1}
\forall x\forall y\forall z((x = y \& y \in z) \to x \in z)
Eсли два множества равны, то любой элемент лежащий в первом,
лежит и во втором
\subsubsection{Аксиома пары}
\label{sec-15-2-2}
\forall x\forall y(\lnot (x=y) \to \exists p(x \in p \& y \in p \& \forall z(z \in p \to (x = z \lor y = z))))
x \ne  y, тогда сущ. \{x, y\}
\subsubsection{Аксиома объединений}
\label{sec-15-2-3}
\forall x(\exists y(y\in x) \to \exists p\forall y(y \in p \leftrightarrow \exists s(y \in s \& s \in x)))
Eсли x не пусто, то из любого семейства множеств можно
образовать „кучу-малу“, то есть такое множество p,
каждый элемент y которого принадлежит по меньшей мере
одному множеству s данного семейства s x
\subsubsection{Аксиома степени}
\label{sec-15-2-4}
\forall x\exists p\forall y(y \in p \leftrightarrow y \in x)
P(x) - множество степени x (не путать с 2ˣ - булеаном)
Это типа мы взяли наш x, и из его элементов объединением и
пересечением например понаобразовывали кучу множеств, а потом
положили их в p.
\subsubsection{Схема аксиом выделения}
\label{sec-15-2-5}
\forall x\exists b\forall y(y \in b \leftrightarrow (y \in x \& \phi(y)))
Для нашего множества x мы можем подобрать множество побольше,
на котором для всех элементов, являющихся подмножеством x
выполняется предикат.
\subsubsection{Аксиома выбора (не входит в ZF по дефолту)}
\label{sec-15-2-6}
Если a = Dj(x) и a \ne  0, то x \in a \ne  0
\subsubsection{Аксиома бесконечности}
\label{sec-15-2-7}
\exists N(∅ \in N \& \forall x(x \in N \to x ∪ \{x\} \in N))
\subsubsection{Аксиома фундирования}
\label{sec-15-2-8}
\forall x(x = ∅ \lor \exists y(y \in x \& y ∩ x = ∅))
\forall x(x \ne  ∅ \to \exists y(y \in x \& y ∩ x = ∅))
Равноценные формулы.

Я бы сказал, что это звучит как-то типа
"не существует бесконечно вложенных множеств"
\subsubsection{Схема аксиом подстановки}
\label{sec-15-2-9}
\forall x\exists !y.\phi(x,y) \to \forall a\exists b\forall c(c \in b \leftrightarrow (\exists d.(d \in a \& \phi(d, c))))
Пусть формула \phi такова, что для при любом x найдется единственный y
такой, чтобы она была истинна на x, y, тогда для любого a
найдется множество b, каждому элементу которого c можно сопоставить
подмножество a и наша функция будет верна на нем и на c
Типа для хороших функций мы можем найти множество с отображением из
его элементов в подмножество нашего по предикату.
\section{Ticket 14: oрдиналы}
\label{sec-16}
\subsection{Ординальные числа}
\label{sec-16-1}
\begin{itemize}
\item Определение вполне упорядоченного множества (фундированное
с линейныи порядком).
\item Определение транзитивного множества
Множество X транзитивно, если
\forall a\forall b((a \in b \& b \in x) \to a \in x)
\item Ординал - транзитивное вполне упорядоченное отношением \in мн-во
\item Верхняя грань множества ординалов S
C | \{C = min(X) \& C \in X | X = \{z | \forall (y\in S)(z ≥ y)\}\}
C = Upb(S)
Upb(\{∅\}) = \{∅\}
\item Successor ordinal (сакцессорный ординал?)
Это b = a' = a ∪ \{a\}
\item Предельны ординал
Ординал, не являющийся ни 0 ни successor'ом.
\item Недостижимый ординал\\
$\epsilon$ - такой ординал, что $\epsilon = w^{\text{\epsilon }}$
$\epsilon_0$ = Upb(w, w$^{\text{w}}$, w$^{\text{w}}$$^{\text{w}}$, w$^{\text{w}}$$^{\text{w}}$$^{\text{w}}$, \dots ) - минимальный из $\epsilon$
\item Канторова форма - форма вида ∑$(a*w^b+c)$, где $b$ - ординал, последовательность
строго убывает по $b$. Есть слабая канторова форма, где вместо a (a \in N)
пишут $a$ раз $w^b$. В канторовой форме приятно заниматься сложениями и
прочим, потому что всякие upb - слишком ниочем.
\end{itemize}
\subsection{Операции над ординальными числами}
\label{sec-16-2}
\subsubsection{Стабилизация убывающей последовательности}
\label{sec-16-2-1}
Допустим, что есть убывающая последовательность ординалов x_1,x₂\dots
Возьмем ординал x_1 + 1 = x₀. Тогда \{x_1, x₂, \dots \} \in x₀. x₀ не пусто,
значит там есть минимальный элемент по определению порядка на ординале.
Пусть этот элемент - m. Тогда поскольку m \in x₀, то m = x_i для какого-то
i нашей убывающей последовательности.
\begin{enumerate}
\item Последовательность убывает нестрого.
Тогда все xₖ \le m, для k > i. Это выполняется, если xₖ = x_i, тогда
последовательность стабилизируется в m.
\item Последовательность убывает строго.
Тогда все xₖ < m для k > i, но m - минимум множества. Противоречие.
Убывающей строго последовательности ординалов не существует.
\end{enumerate}
\subsubsection{Арифметические операции через Upb}
\label{sec-16-2-2}
Пусть lim(a) = предельный ординал a
\begin{align*}
$x + 0      &= x$ \\
$x + c'     &= (x + c)'$ \\
$x + lim(a) &= Upb\{x + c | c < a\}$ \\
 \\
$x * 0      &= 0$ \\
$x * c'     &= x * c + x$ \\
$x * lim(a) &= Upb\{x * c | c < a\}$ \\
 \\
$x \^{} 0      &= 1$ \\
$x \^{} c'     &= x^c * x$ \\
$x \^{} lim(a) &= Upb\{x^c | c < a\}$
\end{align*}

Ну вот короче можно так, только приходится много думать
как реализовывать Upb. Или только у меня так.

2$^{\text{w}}$ = Upb(2, 4, 8, \dots ) = w
\subsubsection{Арифметические операции через Канторову форму}
\label{sec-16-2-3}
Хорошо описано в этой статье:
\href{http://www.google.ru/url?sa=t&rct=j&q=&esrc=s&source=web&cd=1&ved=0CB4QFjAA&url=http://www.ccs.neu.edu/home/pete/pub/cade-algorithms-ordinal-arithmetic.pdf&ei=FDW6\lor JOYNuvXyQPd0ILQBQ&usg=AFQjCNENBOBOdKbbqBYN3iFhmAu_jFD2Sw&sig2=1UISFzJ_21I8f1YScX7Tkw&bvm=bv.83829542,d.bGQ&cad=rjt}{Algorithms for Ordinal Arithmetic}
Переписывать довольно громоздко, учитывая количество вспомогательных
функций. Есть в моем гитхабе (\emph{volhovm/mathlogic}) реализованное
\section{Ticket 15: кардиналы}
\label{sec-17}
\subsection{Кардинальные числа}
\label{sec-17-1}
Будем называть множества равномощными, если найдется биекция.
Будем называть A не превышающим по мощности B, если найдется
инъекция A \to B (|A| \le |B|)
Будем называть А меньше по мощности, чем B, если |A| \le |B| \& |A| \ne  |B|
Кардинальное число - число, оценивающее мощность множества.
Кардинальное число ℵ - это ординальное число a, такое что
\forall  x \le a |x| \le |a|
ℵ₀ = w по определению; ℵ_1 = минимальный кардинал, следующий за ℵ₀
Кардинальное число ℶ - это ординальное число а, такое что
ℶ_i = P(ℶ_i₋_1)
ℶ₀ = ℵ₀
Континуум-гипотеза формулируется таким образом: |P(ℵ₀)| = ℵ_1 или ℶ_1 = ℵ_1
В 40 году Гёдель доказал недоказуемость отрицания Континуум-гипотезы
в терминах ZFC, в 60 Коэн сделал то же самое но без отрицания. Это все
в условиях непротиворечивости ФА. То есть в ZFC нельзя доказать или
опровергнуть континуум-гипотезу.
Сложение кардинальных чисел - |A| + |B| = |A∪B| если в них
нету общих элементов, иначе max(|A|, |B|), поскольку мы можем
построить двумерную таблицу из перес. элементов.
Остальное есть на вики и вряд ли нужно вообще.
\subsection{Диагональный метод Кантора}
\label{sec-17-2}
Докажем, что для любого множества |x| < |P(x)|
Воспользуемся диагональным методом Kантора
Пусть |x|=|P(x)|
Выпишем таблицу, в которой  столбцу p и строке q соответствует
1, если в множестве X лежит p, а в множестве P(X) лежит
множество, содержащее в себе p. Построим ключевое мн-во t:
элемент лежит в t, если на i-й диагональной позиции не стоит 1
и наоборот. То есть это множество всех таких элементов из X,
которым по биекции соответствует множество о чем угодно, но не
о самом элементе (не включающее элемент).
t состоит из подмножеств X, тогда оно должно лежать в P(X).
Докажем, что строка t не присутствует в таблице, сравнив ее с
каждой другой строкой - от каждой n-й строки отличается в n-й
столбце по построению.
Противоречие - t нет в таблице, но t \in P(X).
\subsection{Теорема Лёвингейма-Скулема}
\label{sec-17-3}
\begin{itemize}
\item Назовем мощностью модели мощность ее носителя (\lor или P или \lor ∪P).
M - модель, |M| - ее мощность, ну ясно.
\item Элементарная подмодель
Пусть M - модель фс первого порядка с носителем D. Пусть определено
D_1 ⊂ D, тогда структура M_1 построенная на D_1 так, что в ее интерпретации
лежит все, что и в интерпретации M, кроме элементов, взаимодействующих
с M $\backslash$ M_1 (сужение области определения на D_1), называется \textbf{элементарной}
\textbf{подмоделью}, если:
\begin{enumerate}
\item Любая функция ФС, над которой рассматривается M, замкнута на
D_1 (то есть если a \in D_1, b \in D_1, \dots  то f(a, b, \dots ) \in D_1)
\item Любая формула A(x_1\dots x_n) теории при любых аргументах из D_1,
истинная в M истинна и в M_1.
\end{enumerate}
\item Элементарная подмодель теории - модель теории
Рассмотрим формулу А, она общазначима в М, значит и в М_1, тогда M_1 корректна.
\item Счетно-аксиоматизируемая теория - множество аксиом ФС имеет мощность ℵ₀
\item ФА и ТМ счетно-аксоиматизируемые
\item Пусть M - модель, T - мн-во формул теории. Тогда \exists M_1 : |M_1| = max(|T|, ℵ₀)
Нужно построить необходимое предметное множество и доказать,
что модель на нем - это подмодель.
\begin{enumerate}
\item Построение множества
Пусть у нас есть множество D', тогда D'' = D' ∪ P, где P - некоторое
множество формул добавленное при рассмотрении формул D' по одной.
A(y, x_1\dots x_n) - n-местная формула из Т. Фиксируем x_1\dots x_n из D'.
\begin{itemize}
\item Если А = И или А = Л (тождественно) при любом y \in D - пропустим формулу
\item Если А = И или А = Л при каких-то y \in D' - пропустим формулу
\item \exists y: A(y,..) = И, но при этом \forall y\in D' A(y,..) = Л - тогда добавим один из
тех у, на которых формула истинна, в D''. Добавим еще констант, которые
нужны для вычисления А. Типа если В D' не хватает переменных для того,
чтобы показать что A может принимать истинностное значение, сгенерим
и добавим такое.
\end{itemize}
Переход от предыдущего множества к текущему увеличивает его не более чем на
ℵ₀ * |T| * |D'| - max(ℵ₀, T)
Рассмотрим D₀, D₀ ⊂ D такое, что в него входят те элементы носителя,
соответствующие константам, упоминающимся в Т. Если оно пустое -- добавим
какую-нибудь константу из D. Оно ляжет в начало счетной последовательности
D₀ ⊂ D_1 ⊂ \dots  (каждый переход описан выше). D* = ∪D_i.
D* - нужное нам множество. |D*| = max(ℵ₀, T)
\item Проверка структуры
Индукция по структуре.
\begin{itemize}
\item База. Предикат.
P(f_1(x_1\dots x_n), \dots , fₖ(x_1\dots x_n)). Если x_1\dots x_n взяты из D*, то они были
добавлены на некотором шаге, значит \exists t | x_i \in Dₜ. Тогда на шаге Dₜ₊_1
лежат результаты функций f_1\dots fₖ. по построению. Тогда оценка формулы
сохраняется.
\item Переход
Связки X\&Y, X\lor Y, X\to Y, \lnot X работают на сужении модели и оценка сохр.
\begin{itemize}
\item \exists yB(y, x_1\dots x_n). Фиксируем x_1\dots x_n из D*.
\begin{enumerate}
\item A была тождественно истинна или ложна - все ОК
\item А принимала значения разных знаков
Каждый x_i добавлен на каком-то шаге, тогда возьмем максимальный
шаг t, в Dₜ₊_1 уже лежат все эти x_i.
Тогда по построению Dₜ₊_1 мы добавили нужный y такой что B(y, x_1\dots x_n)
определено и выполнено в M.
Значит B выполнено в M* по индукции, тогда A истинна в M*.
\end{enumerate}
\end{itemize}
\item \forall yB(y, x_1\dots x_n).
\begin{enumerate}
\item Тождественно - ОК
\item Принимает значения разных знаков
Если оно истинно в M, тогда оно истинно в M* по 1 пункту перехода.
Если \forall yB(y, x_1\dots x_n) было ложно на t-шаге, тогда на t+1 шаге мы
здоровски исправили ситуацию, положив в мир y на котором оно истинно.
Если оно было истинно, то по пункту 2 пошло дальше.
\end{enumerate}
Таким образом, D* - подмодель нашего множества, |D*| = ℵ₀ + |T|
\end{itemize}
\end{enumerate}
\end{itemize}
\subsection{Парадокс Скулема}
\label{sec-17-4}
Мнимый парадокс Сколема формулируется так:
Возьмем теорию, прикрутим модель с аксиоматикой ZF. Модель
будет счетно-аксиоматизируемой потому что ZF.
Утверждается, что в ZF $\vdash \exists x(|x| = ℶ_1)$ - это доказывает диагон.
метод Кантора.
Тогда получается что по теореме Лёвингейма-Скулема у нашей
модели есть подмодель размером ℵ₀ + ℵ₀(счетно-акс) = ℵ₀, но
мы можем взять то самое x | |x| = ℶ_1, и его занумеровать, выходит.

Формальный подход не допускает этого конфликта ввиду одного
простого факта:
Рассмотрим отношение существования несчетного мн-ва в R.
$ZF \vdash \lnot \exists f$(f - биецкия между w и P(w)) \& \exists f(f - биецкия между w и w∪w)
// второе гарантирует счетность
Первый аргумент конъюнкции - \lnot \exists f(\forall x\forall y(<x,y>\in f \leftrightarrow x\in A\&x\in B))
<x,y> - пара (типа \{x, y, \{x\}\})
Тогда это значит, что в носителе модели нет такого f, что он
бы представлял собой объединение пар.
Собственно, по теореме Лёвингейма-Скулема у нас любая подмодель
будет иметь счетный носитель. Нет никакого противоречия, потому
что мы все еще работаем со счетным количеством множеств, а
отсутствие биекции все так же выражается отсутствием множества
в носителе.
\section{Ticket 16: неполнота ФА}
\label{sec-18}
\subsection{Теорема о трансфинитной индукции}
\label{sec-18-1}
Пусть есть формула с одной свободной переменной a(x)
a истинна, если
\begin{enumerate}
\item a(0)
\item Если для любого конечного p - ординала мы можем
показать следование \{ q < p => a(p) \}, то a(p) истинно.
\end{enumerate}

Без док-ва, не требуется.

\subsection{Построение S∞}
\label{sec-18-2}
Мы строим еще одну теорию I порядка.
По сути, мы вкладываем ФА в нашу теорию так, что
любое доказательство ФА работает в S∞ и мы можем доказать
доказать непротиворечивость любого "импортированного" д-ва

\begin{enumerate}
\item Формулы:
Оставим связки \forall x, \lor , \lnot
Заметим, что \{\lor , \lnot \} полно для \{0, 1\}.
\item Доказательство
Доказательством является дерево утверждений, в
узлах которого правли, причем если дерево
растет вверх, то правила действуют сверху вниз.
\item Аксиомы:
\begin{enumerate}
\item все термы ФА без переменных типа θ_1=θ₂ (корректные)
\item все термы вида \lnot (θ_1=θ₂) если [θ_1]\ne [θ₂] (некорректные, все остальн.)
\end{enumerate}
\item Правила:
Примечание: в правилах используются боковые формулы,
они могут отсутствовать. Это сделано для формализации
того факта, что мы можем применять правило для двух любых
элементов нашей дизъюнкции или вроде того.
Примечение: org-mode подчеркивает a, если "$_{\text{a}}$\_"
\begin{enumerate}
\item Структурные
\begin{enumerate}
\item Перестановка
\uline{a\lor b\lor γ\lor σ}
a\lor γ\lor b\lor σ
\item Сокращение
\uline{a\lor b\lor b\lor γ}
a\lor b\lor γ
\end{enumerate}
\item Сильные
\begin{enumerate}
\item Ослабление
\uline{γ}
a\lor γ
\item Де-Морган
\uline{(\lnot a)\lor γ$_{\text{(\lnot b)}}$\lor γ}
\lnot (a\lor b)\lor γ
\item Отрицание
\uline{a\lor γ\_\_}
(\lnot \lnot a)\lor γ
\item Квантификация
\uline{\lnot a(t)\lor γ}
\lnot \forall x.a(x)\lor γ
\item Бесконечная индукция
\uline{a(0\textasciitilde{})\lor γ$_{\text{a}}$(1\textasciitilde{})\lor γ} \dots  \uline{a(r\textasciitilde{})\lor γ}\dots
\forall xa(x)\lor γ
\end{enumerate}
\item Сечение (для облегчения жизни)
\uline{γ\lor a$_{\text{\lnot a\lor δ}}$}
γ\lor δ
\end{enumerate}
\item Порядки
Каждой формуле в дереве соответствует порядок, причем
посылке и заключению (выше и ниже \uline{\_}) слабого правила
вывода соответствует один порядок, а порядковое число,
отнесенное заключению сильного правила или сечения, больше
порядковых чисел, отнесенных соотвтетствующим посылкам.
Порядковые числа - это ординалы, они могут быть достижимыми,
но не конечными - пусть формула какая-нибудь околорекурсивная
типа Ф(x): Ф(0) = A \to A, Ф(1) = A \to A \to A, Ф(2) = A \to A \to A \to A.
Тогда пусть мы хотим доказать \forall x.Ф(x) - по бесконечной
индукции порядок термов будет увеличиваться, а порядок
\forall x.Ф(x) будет w. Именно из-за этого факта мы используем
в доказательстве теоремы об устранении сечений трансфинитную
индукцию по порядку - ведь обычной индукции мало для порядков
больших w.
\item Степень
Степень сечения - количество связок в \lnot a.
Степень доказательства - наибольшая степень сечения в дереве.
Степень всегда конечна - любая формула в ФА содержит конечное
число связок, а при трансляции нет возможности увеличить
их количество. Тогда трансфинитная индукция по термам, в
которых в сечении количество связок растет до бесконечности,
невозможно.

\item Нитью называется последовательность формул от начальной до
конечной. Все нити в доказательстве конечны, поскольку если в начальной
формуле стоит ординал, последовательность в нити не возрастает, а эти
числа убывают с применением строгого правила или сечения. Мы знаем,
что строго убывающая бесконечная последовательность ординалов не существует.
Добавим правило, что последовательность применения слабых правил подряд
была всегда конечна
\item Теорема в S∞ - выражение, которое может стоять в заключительной формуле
вывода
\end{enumerate}
\subsection{Теоремы об эквивалентности ФА и S∞}
\label{sec-18-3}
\subsubsection{Лемма 1: В S∞ выводимо A\lor \lnot A}
\label{sec-18-3-1}
А либо корректна, либо некорректна, тогда
\uline{A}
\lnot A\lor A     ослабление

\uline{\lnot A}
\uline{A\lor \lnot A}   ослабление
\uline{\lnot A\lor A}   перестановка
\subsubsection{Лемма 2: В S∞ выводимо s\ne t\lor \lnot A(s)\lor A(t)}
\label{sec-18-3-2}
Если выводимо A(s), s = t, то выводимо A(t) (все вхождения меняем)
если s=t, то выводимо \lnot A(t)\lor A(t), потом сделаем ослабление
если s\ne t, то она аксиома (некорректная) и ослабим.
\subsubsection{Лемма 3: всякая выводимая в S замкнутая формула А является теоремой S∞}
\label{sec-18-3-3}
Докажем, что если что-то доказуемо в ФА, то его эквивалент
доказуем и в S∞.
$\vdash ₚₐA => \vdash ₛA'$
Схема док-ва
Рассмотрим доказательство в ФА, оно состоит из
\beta_1'\dots \beta_n', оттранслируем каждое в \beta_i \in S∞ (по полноте \{\lnot , \lor \} это возможно)
Тогда можно сделать дерево, в котором начальные формулы - аксиомы S,
а правила вывода - MP, GEN.

Рассмотрим формулу A = \beta_i
\begin{enumerate}
\item B \to C \to B, те \lnot B \lor (\lnot  C \lor B)
Из замкнутости A следует замкнутость B
\lnot B \lor B выводима по Л1, тогда ослабим с \lnot C, переставим.
\item (B \to C) \to (B \to C \to D) \to B \to D
\lnot (\lnot B \lor C) \lor (\lnot (\lnot B \lor (\lnot C \lor D)) \lor (\lnot B \lor D))
По Л1 выводимо \lnot (\lnot B \lor C) \lor (\lnot B \lor C),
(\lnot B \lor \lnot C \lor D) \lor \lnot (\lnot B \lor \lnot C \lor D)
Тогда можно по перестановке, сечению (с С) и сокращению
доказать (B \to C \to D) \to (B \to C) \to (B \to D)
что одно и то же, см дедукцию в предикатах
\item (B \to C) \to (B \to \lnot C) \to \lnot B
\lnot (\lnot B \lor C) \lor \lnot (\lnot B \lor \lnot C) \lor \lnot B
\begin{enumerate}
\item \lnot B \lor B                        л1
\item \lnot \lnot \lnot B \lor B                      отрицание
\item \lnot (\lnot \lnot B \lor C) \lor \lnot \lnot \lnot B \lor B         ослабление
\item \lnot \lnot \lnot B \lor \lnot (\lnot \lnot B \lor C) \lor B         перестановка
\item \lnot \lnot \lnot B \lor B \lor \lnot \lnot C                аналогично + еще перестановка
\item \lnot C \lor \lnot \lnot C                      лемма
\item \lnot C \lor B \lor \lnot \lnot C                  ослабление + перестановка
\item \lnot (\lnot \lnot B \lor C) \lor B \lor \lnot \lnot C          де-морган от 6 и 8
\item \lnot \lnot C \lor \lnot (\lnot \lnot B \lor C) \lor B          перестановка 8
\item \lnot (\lnot \lnot B \lor \lnot C) \lor \lnot (\lnot \lnot B \lor C) \lor B де-морган от 4 и 9
\end{enumerate}
Ну вот мы доказали что-то очень похожее на то, что нужно было.
Там контрпозиция, шмяк шмяк, готово.
\item Видимо, примерно все формулы так доказываются.
\item \forall x.B(x) \to B(t)
\lnot \forall x.B(x) \lor B(t)
По л1 \lnot B(t) \lor B(t), потом квантификация по 1 элем.
\item B(t) \to \exists x.B(x)
B(t) \to \lnot \forall x.B(x) (что заметно отличается от \forall x.\lnot B(x))
\lnot B(t) \lor \lnot \forall x.B(x)
Не, я не знаю. Но точно можно! По бесконечной индукции мож как-то.
Или там сечение хитрое.
\end{enumerate}

А еще есть аксиомы ФА
\begin{enumerate}
\item a = b \to a' = b'
\lnot (a = b) \lor a' = b'
\begin{itemize}
\item если a = b, то a' = b', это аксиома S∞,
тогда по ослаблению добавим \lnot (a = b)
\item если a \ne  b, то она же и аксиома
\end{itemize}
\item a = b \to a = c \to b = c
a \ne  b \lor a \ne  c \lor b = c
a \ne  b \lor \lnot (x = c) @ b \lor (x = c) @ c     по лемме 2
\item a' = b' \to a = b
Аналогично 1
\item \lnot (a' = 0)
Аксиома, поскольку a' всегда имеет с 0 разные значения
\item a + b' = (a + b)'
TODO
\item a + 0 = a
Аксиома, поскольку это вседа равенство
\item a * 0 = 0
аналогично 6
\item a * b' = a * b + a
TODO
\item \phi[x:=0] \& \forall x.(\phi \to \phi[x:=x']) \to \phi
\lnot B(0) \lor \lnot \forall x(\lnot B(x)\lor B(x')) \lor B(0)         лемма 1 и перестановка
\lnot B(0) \lor \lnot (\lnot B(0) \lor B(1)) \lor \dots            можно показать по индукции
\dots  \lor \lnot (\lnot B(k) \lor B(k')) \lor B(k')   (ослабление, перестановка, де-морган)
\lnot B(0) \lor \lnot (\forall x(\lnot B(x) \lor B(x'))) \lor B(k')    k раз квантификация, перестановки, сокращ.
Применим бесконечную индукцию относительно первого
и третьего терма и получим что надо.
\end{enumerate}

Окей, с аксиомами разобрались.
И еще есть два правила вывода
\begin{enumerate}
\item MP
B            условие
\lnot B \lor A       условие
A            сечение
\item GEN
B(x)         условие
Продвигаясь от этой формулы вверх можно поменять все
x на k, тогда верно
$$\vdash B(k)$$
На основании принципа бесконечной индукции доказываем \forall xB(x)
\item A \to B(t) => A \to \forall x.B(x)
\lnot A \lor B
Заменим все вхождения перменной в доказательстве в
ФА формулы \lnot A \lor B на 0, 1, 2\dots ,
тогда по бесконечной индукции:
$\lnot A \lor B(0), \lnot A \lor B(1), \dots  \vdash \lnot A \lor \forall x.B$   (только B(0) \lor \lnot A везде)
\item A(t) \to B => \exists x.A(x) \to B
$\lnot A \lor B \vdash \lnot \lnot \forall x.(\lnot A(x)) \to B$
Заменим все вхождения свободной перменной t в ФА на конкретные.
Получим счетное мн-во д-в \lnot A(0) \lor B, \lnot A(1) \lor B, \dots
по беск. индукции
\forall x.\lnot A(x) \lor B
\lnot \lnot \forall x.\lnot A(x) \lor B -  навесили двойное отрицание
\end{enumerate}
\subsubsection{Следствие: непротиворечивость S∞ влечет непротиворечивость S}
\label{sec-18-3-4}
Пусть в S доказуемо \lnot (0=0), тогда оно доказуемо и в
S∞, тогда
\uline{A\lor 0=0$_{\text{\lnot }}$(0=0)\lor A}          Аргументы получаются по ослаблению
\uline{A\lor A}
A
\subsection{Теорема Генцена об устранении сечений}
\label{sec-18-4}
\subsubsection{Лемма: сильные правила 2, 3, 5 обратимы}
\label{sec-18-4-1}
Правила обратимы и их д-во имеет порядок и степень не больше,
чем первоначальное
Нам дают доказательство формулы, мы строим новое дерево,
в котором из результата следуют посылки.
\begin{enumerate}
\item Правило Де-Моргана
\lnot (B \lor E) \lor D
Рассмотрим вывод формулы. Проследим вхождения
подформул \lnot (B \lor E)  формулы, которые соответствуют
вхождению в конечную формулу (то есть не те, которые
исчезают в сечении). Мы пройдем через всякий случай
применения слабого правила и сильные, когда \lnot (B \lor E)
является боковой формулой. Остановка произойдет
либо в ослаблении $( F \vdash \lnot (B \lor E) \lor F )$ либо в Де-Моргане
$( \lnot B \lor F, \lnot E \lor F \vdash \lnot (B \lor E) \lor F )$. Совокупность таких
вхождений формулы назовем ее \textbf{историей}.
Если все вхождения формулы \lnot (B \lor E) в ее истории заменить
на \lnot B, то в результате получится вывод \lnot B \lor D. Аналогично
можем вывести \lnot E \lor D.
\item Правило отрицания
\lnot \lnot B \lor D
Заменим все вхождения \lnot \lnot B в ее истории на B, получим
вывод B \lor D
\item Бесконечная индукция
\forall xB(x) \lor D
Заменим все вхождения \forall xB(x) в ее истории (1 элемент) на
B(k) (причем если мы уткнулись в бесконечную индукцию,
выберем только ту ветку, которая соответствует B(k))
Тогда для любого k получим вывод B(k)
\end{enumerate}
\subsubsection{Теорема: устранение сечения}
\label{sec-18-4-2}
Если для A в S∞ существует вывод (m, a), то существует
вывод в S∞ (n, 2ᵃ), где n < m
Докажем через трансфинитную индукцию по порядку a вывода A
\begin{itemize}
\item База: порядок вывода 0
Вывод не содержит сечений и его степень 0
\item Переход: пусть теорема верна для выводов порядков меньших a
Будем продвигаться вверх по выводу, пока не встретим первое
применение сильного правила или сечения.
\begin{enumerate}
\item Сильное правило
Пусть его посылки занумерованы порядковыми числами aₗ
Согласно индуктивному предположению, для этих посылок
существует дерево вывода F со степенью < m и порядком 2$^{\text{(a_i)}}$.
Заменим таким деревом то поддерево данного дерева вывода,
заключительной формулой которого служит рассматриваемое
вхождение F. Сделав так со всеми посылками мы получим новое
дерево для A, отнесем ему порядковое число 2ᵃ > 2$^{\text{(a_i)}}$
(пояснение от меня, потому что я чет долго доганял - каждую
посылку заменяем по индукции на ее модное новое дерево
от нее самой же).
\item Сечение
Значит имеем что-то на уровне
\uline{C\lor B\_$_{\text{\lnot B\lor D}}$}
C\lor D
Согласно индуктивному предположению для C\lor B и \lnot B\lor D существуют
выводы степеней меньших m и порядков 2$^{\text{a_1}}$, 2$^{\text{a₂}}$. Рассмотрим
разные случаи строения B:
\begin{enumerate}
\item B - это элементарная формула. Одна из формул B и \lnot B есть
аксиома. Пусть K - та, которая не аксиома.
По предположениию поддерево основного дерева вывода с K
может быть заменено другим со степенью n и порядка 2$^{\text{(a}_{\text{i}}\text{)}}$
(i = 1 или 2, смотря в какой посылке K).
В этом новом дереве расмотрим историю K, начальные формулы
в которой могут возникнуть только по ослаблению (а исчезать
по сечению). Поэтому удаление всех вхождений K из истории
приводит к построению дерева вывода для D или C порядка
2$^{\text{a_i}}$. Отсюда с помощью ослабления получаем дерево вывода
для C \lor D порядка 2$^{\text{a}}$. Степень меньше m (одно сечение убрали).
\item B - это \lnot E, тогда посылки выглядят как
C\lor \lnot E, \lnot \lnot E\lor D
Существует дерево вывода для \lnot \lnot E\lor D степени < m и порядка 2$^{\text{a₂}}$
В силу леммы об обратимости можно построить дерево вывода
E\lor D порядка 2$^{\text{a₂}}$ степени < m
Кроме того существует дерево вывода степени < m и порядка 2$^{\text{a_1}}$
для левой посылки, тогда построим из них новое дерево
вывода
\uline{E\lor D}   \uline{C\lor \lnot E}
\uline{D\lor E\_$_{\text{\lnot E\lor C}}$}
\uline{D\lor C}
C\lor D
Степень выделенного здесь сечения на единицу меньше общего
числа связок и кванторов в \lnot E, которое само по себе \le m.
Формуле C\lor D можно присвоить порядковое число C \lor D по свойству
посылок новог сечения.
\item B - это E\lor F, посылки:
C\lor E\lor F, \lnot (E\lor F)\lor D
Существует дерево вывода для правой посылки < m и порядка 2$^{\text{a₂}}$,
по лемме об обратимости существуют выводы степеней < m и порядка
2$^{\text{a₂}}$ для \lnot E\lor D и \lnot F\lor D (по Де-Моргану). По предположению индукции
есть еще дерево вывода для левой посылки.
Из последних трех построим такое:
\uline{C\lor E\lor F\_$_{\text{\lnot F\lor D}}$}
   \uline{C\lor E\lor D}
   \uline{C\lor D\lor E\_\_\_$_{\text{\lnot E\lor D}}$}
       \uline{C\lor D\lor D}
C\lor D
Степень сечения уменьшили (в каждом на 1), в формуле
C\lor E\lor D можно дать порядок $2^{\max(a_1, a₂)}$+1, а остальным - $2^a$
\item B - это \forall xE, посылки:
C\lor \forall xE  (\lnot \forall xE)\lor D
По индуктивному предположению для левой посылки можно построить
дерево вывода (< m,2$^{\text{a_1}}$). В силу указания в начале 2 леммы о
эквивалентности S∞ и ФА (если $\vdash A(t), t=s, то \vdash A(s)$) и леммы
об обратимости для любого постоянного терма z существует вывод
C\lor E(z).
Можем и правую посылку заменить (< m, 2$^{\text{a₂}}$) по инд.предположению
Тогда в правой посылке история \lnot (\forall xE) может начинаться либо
с ослабления либо с квантификации.
Заменим на C все такие начала - если это ослабление, то просто
подменим вместо \lnot (\forall xE) новое C, если квантификация, то
\uline{\lnot E(t)\lor F\_\_}         \uline{C\lor E(t)\_$_{\text{\lnot E}}$(t)\lor F}   //первое мы взяли из левой посылки
\lnot (\forall xE(x))\lor F   =>          C\lor F
В результате мы получили дерево вывода C\lor D
Степень меньше m, поскольку связку мы одну убрали
\textbf{тут еще ОЧЕНЬ АДОВЫЕ оценки порядка C\lor D, но он 2$^{\text{a}}$}
\end{enumerate}
\end{enumerate}
\end{itemize}
\subsubsection{Следствие: устранение всех сечений}
\label{sec-18-4-3}
Воспользуемся леммой об устранении сечения, пока степень вывода
не станет равна нулю. Тогда порядок будет что-то на уровне степенной
башни из двоек, в вершине которой первоначальная степень, а количество
двоек -- количество применения леммы.
\subsubsection{Следствие: S∞ непротивроечива}
\label{sec-18-4-4}
Если S∞ противоречива, в ней докажется $0\ne 0\lor 0\ne 0\lor \dotsb \lor 0\ne 0$.
В ее истории появление 0\ne 0 может быть только из-за ослабления,
но тогда посылкой ослабления тоже будет 0\ne 0. Если в д-ве есть
сечения, мы можем перестроить его, устранив все сечения, а затем провести
рассуждения вновь и увидеть, что такая формула недоказуема, а значит
S∞ непротиворечива, значит и ФА тоже непротиворечива.
\section{Ключевые фигуры}
\label{sec-19}
\begin{itemize}
\item Станислав Яськовски - 1906
Aлгебра Яськовского, один из первых исследователей ИИВ
\item Герхард Генцен - 1909
Теорема об устранении сечения (1935)
\item Курт Гёдель - 1906
Теоремы о непротиворечивости - 1930
Чуть ли не все остальное
\item Сол Крипке - 1940
Семантика Крипке - 1960-1970
\item Дэвид Гильберт - 1862, Пауль Бернайс - 1888
Основания математики - 1934, 39
\item Их достаточно много, а времени мало.
\end{itemize}
% Emacs 24.4.1 (Org mode 8.2.10)
\end{document}
