\section{Полнота исчисления высказываний}
\label{sec-4}
\subsection{Полнота исчисления высказываний относительно алгебры Яськовского}
\label{sec-4-1}
Кстати полноту можно доказывать маханием руками как для предикатов,
и я не могу утверждать, что при таком подходе ИВ не будет полно
относительно любой модели.
\subsubsection{Контрапозиция}
\label{sec-4-1-1}
\begin{lemma}
    $(\alpha \to \beta) \to (\lnot \beta \to \lnot \alpha)$
\end{lemma}
\begin{proof}
    Докажем, что $(\alpha \to \beta), \lnot \beta \vdash \lnot \alpha$:\\
    \begin{tabular}{lll}
    (1) & $\alpha \to \beta$& Допущение\\
    (2) & $(\alpha \to \beta) \to (\alpha \to \lnot \beta) \to \lnot \alpha$& Сх. акс. 9\\
    (3) & $(\alpha \to \lnot \beta) \to \lnot \alpha$& M.P. 1,2\\
    (4) & $\lnot \beta \to \alpha \to \lnot \beta$& Сх. акс. 1\\
    (5) & $\lnot \beta$& Допущение\\
    (6) & $\alpha \to \lnot \beta$& M.P. 5,4\\
    (7) & $\lnot \alpha$& M.P. 6,3\\
    \end{tabular}\\
    После применения теоремы о дедукции 2 раза получим как раз то, что нужно
\end{proof}
\subsubsection{Правило исключененного третьего}
\label{sec-4-1-2}
С помощью контрапозиции доказываем два утверждения:\\
$\lnot (A|\lnot A)\to \lnot A$ (один раз контрапозицию от этого обратную, там $A\to (A|\lnot A)$ акс) \\
$\lnot (A|\lnot A)\to \lnot \lnot A$
Потом девятую аксиому и снимаем двойное отрицание
\subsubsection{Всякие очевидные вещи типа если выводится из А и из Б то из А и Б тоже}
\label{sec-4-1-3}
\subsubsection{Правило со звездочкой (14 доказательств)}
\label{sec-4-1-4}
\begin{enumerate}
\item $\alpha, \beta \vdash \alpha \lor \beta$ \\
$\alpha$ \\
$\alpha \to \alpha \lor \beta$ \\
$\alpha \lor \beta$
\item $\alpha, \lnot \beta \vdash \alpha \lor \beta$ \\
$\alpha$ \\
$\alpha \to \alpha \lor \beta$ \\
$\alpha \lor \beta$
\item $\lnot \alpha, \beta \vdash \alpha \lor \beta$ \\
$\beta$ \\
$\beta \to \alpha \lor \beta$ \\
$\alpha \lor \beta$
\item $\lnot \alpha, \lnot \beta \vdash \lnot (\alpha \lor \beta)$ \\
$\lnot \alpha$ \\
$\lnot \beta$ \\
$(\alpha \lor \beta \to \alpha) \to (\alpha \lor \beta \to \lnot \alpha) \to \lnot (\alpha \lor \beta)$ \\
$\lnot \alpha \to \alpha \lor \beta \to \lnot \alpha$ \\
$\alpha \lor \beta \to \lnot \alpha$ \\
$\lnot \alpha, \lnot \beta, \alpha \lor \beta \vdash \alpha$ \\
$\lnot \alpha$ \\
$\lnot \beta$ \\
$\alpha \lor \beta$ \\
$\alpha \to \alpha$ \\
\ldots{} //д-во $\lnot \beta, \lnot \alpha \vdash \beta \to \alpha$ \\
$\beta \to \alpha$ \\
$(\alpha \to \alpha) \to ((\beta \to \alpha) \to (\alpha \lor \beta \to \alpha))$ \\
$(\beta \to \alpha) \to (\alpha \lor \beta \to \alpha)$ \\
$\alpha \lor \beta \to \alpha$ \\
$\alpha$ \\
$\alpha \lor \beta \to \alpha$ \\
$(\alpha \lor \beta \to \lnot \alpha) \to \lnot (\alpha \lor \beta)$ \\
$\lnot (\alpha \lor \beta)$
\item $\alpha, \beta \vdash \alpha \land \beta$ \\
$\alpha$ \\
$\beta$ \\
$\alpha \to \beta \to \alpha \land \beta$ \\
$\beta \to \alpha \land \beta$ \\
$\alpha \land \beta$
\item $\alpha, \lnot \beta \vdash \lnot (\alpha \land \beta)$ \\
$\lnot \beta$ \\
$((\alpha \land \beta) \to \beta) \to ((\alpha \land \beta) \to \lnot \beta) \to \lnot (\alpha \land \beta)$ \\
$\alpha \land \beta \to \beta$ \\
$(\alpha \land \beta \to \lnot \beta) \to \lnot (\alpha \land \beta)$ \\
$\lnot \beta \to \alpha \land \beta \to \lnot \beta$ \\
$\alpha \land \beta \to \lnot \beta$ \\
$\lnot (\alpha \land \beta)$
\item $\lnot \alpha, \beta \vdash \lnot (\alpha \land \beta)$ \\
аналогично
\item $\lnot \alpha, \lnot \beta \vdash \lnot (\alpha \land \beta)$ \\
аналогично
\item $\alpha, \beta \vdash \alpha \to \beta$ \\
$\beta$ \\
$\beta \to \alpha \to \beta$ \\
$\alpha \to \beta$
\item $\alpha, \lnot \beta \vdash \lnot (\alpha \to \beta)$ \\
$\alpha$ \\
$\lnot \beta$ \\
$\lnot \beta \to ((\alpha \to \beta) \to \lnot \beta)$ \\
$(\alpha \to \beta) \to \lnot \beta$ \\
$\alpha, \lnot \beta, \alpha \to \beta \vdash \beta$ \\
$\alpha$ \\
$\alpha \to \beta$ \\
$\beta$ \\
$(\alpha \to \beta) \to \beta$ \\
$((\alpha \to \beta) \to \beta) \to ((\alpha \to \beta) \to \lnot \beta) \to \lnot (\alpha \to \beta)$ \\
$((\alpha \to \beta) \to \lnot \beta) \to \lnot (\alpha \to \beta)$ \\
$\lnot  \beta \to (\alpha \to \beta) \to \lnot \beta$ \\
$(\alpha \to \beta) \to \lnot \beta$ \\
$\lnot (\alpha \to \beta)$
\item $\lnot \alpha, \beta \vdash \alpha \to \beta$ \\
$\beta$ \\
$\beta \to \alpha \to \beta$ \\
$\alpha \to \beta$
\item $\lnot \alpha, \lnot \beta \vdash \alpha \to \beta$\\
Ну тут типо очевидно (на самом деле тут боль и страдания)
\item $\alpha \vdash \lnot \lnot \alpha$\\
Схема аксиом 9
\item $\lnot \alpha \vdash \lnot \alpha$\\
$\lnot \alpha$
\end{enumerate}
